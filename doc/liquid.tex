% ------------------------------------------------------------------
%
%   TITLE: LIQUID documentation
% AUTHORS: Joseph D. Gaeddert et al.
% CREATED: February 7, 2009
% REVISED: 
%     URL: http://computing.ece.vt.edu/~jgaeddert/
%          https://ganymede.ece.vt.edu/trac/liquid/
%
% ------------------------------------------------------------------
 
\documentclass[11pt,twoside]{report}
% ------------------- DOCUMENT VARIABLES -------------------
\setlength{\textwidth}{6.5in}
\setlength{\textheight}{8.5in}
\setlength{\evensidemargin}{0in}
\setlength{\oddsidemargin}{0in}
\setlength{\topmargin}{0in}
\setlength{\parindent}{0pt}
\setlength{\parskip}{0.1in}

% ------------------- BIBLIOGRAPHY STYLE -------------------
\usepackage{natbib}
%\bibliographystyle{plain} % note the change here
\bibliographystyle{IEEEtran}
\bibpunct{[}{]}{,}{n}{}{;}  % define bibliography punctuation
% The six mandatory arguments for \bibpunct are:
%   1. opening bracket: '(', '[', '{', or '<'
%   2. closing bracket: ')', ']', '}', or '>'
%   3. separator between multiple citations: ';' or ','
%   4. citation style: 'n' for numerical style, 's' for numerical superscript
%      style, or 'a' for author year style
%   5. punctuation between the author names and the year
%   6. punctuation between years or numbers when common author lists are suppressed: ',' or ';' 

% ------------------- GRAPHICS PACKAGES -------------------
%\usepackage{epsf}
%\usepackage{graphicx}
\ifx\pdfoutput\undefined
\usepackage{graphicx}
\else
\usepackage[pdftex]{graphicx}
\fi
\usepackage{epsfig}
\usepackage{epstopdf}
\usepackage{colortbl}
\usepackage{color}

%\usepackage{tabularx}
\usepackage{ctable} % \toprule, \midrule, \bottomrule
\setlength{\heavyrulewidth}{0.1em} % modify thickness of \toprule, \bottomrule
\newcommand{\otoprule}{\midrule[\heavyrulewidth]}
\usepackage{subfigure}
\usepackage{multirow} % for tables
%\usepackage{fancyhdr}
\usepackage{amsmath}
\usepackage{listings}
\usepackage{fancyvrb}
\usepackage{acronym}

% tabular: \hline (thin) \hline\hline (thick)
% from: http://www.faqs.org/faqs/tex-faq/ (see #44)
\setlength{\doublerulesep}{\arrayrulewidth}

%--------- NEW COMMANDS -------------------
\newcommand{\NA}{\scriptsize{NA}}
\newcommand{\E}{\scriptsize{E}}
\newcommand{\tRMS}{\ensuremath{\tau_{rms}}}
\newcommand{\tmED}{\ensuremath{\bar{\tau}}}
\newcommand{\Np}{\ensuremath{N_p}}
\newcommand{\ua}{\ensuremath{\uparrow}}
\newcommand{\da}{\ensuremath{\downarrow}}
\newcommand{\sinc}{\textup{sinc}}
\newcommand{\etal}{{\it et al.}}
\newcommand{\wavt}{W@VT}
\renewcommand{\vec}[1]{\boldsymbol{#1}}
\newcommand{\ciren}{C{\sc iren}}

%--------- CAPTION OPTIONS -------------------
\usepackage[small,bf]{caption}
\setcaptionwidth{15cm}
\setlength{\belowcaptionskip}{0.5cm}

%%%%%%%%%%%%%%%%%%%%%%%%%%%%%%%%%%%%%%%%%%%%%%%%%%%%%%%%%%%%%%%%%%%%%
%
%             MAIN DOCUMENT
%
%%%%%%%%%%%%%%%%%%%%%%%%%%%%%%%%%%%%%%%%%%%%%%%%%%%%%%%%%%%%%%%%%%%%%

\begin{document}

% ------------------- DEFINE LISTINGS -------------------
\input{highlight.sty}


%%%%%%%%%%%%%%%%%%%%%%%%%%%%%%%%%%%%%%%%%%%%%%%%%%%%%%%%%%%%%%%%%%%%%
%
%             TITLE PAGE
%
%%%%%%%%%%%%%%%%%%%%%%%%%%%%%%%%%%%%%%%%%%%%%%%%%%%%%%%%%%%%%%%%%%%%%
\thispagestyle{empty}
\pagenumbering{roman}
\begin{center}

{\huge liquid} \\
Software-defined radio digital signal processing library

\vfill

Joseph D. Gaeddert

\vfill

February 7, 2009 \\
Blacksburg, Virginia

\vfill

{\it Keywords:}
polyphase filterbanks,
OFDM/OQAM,
power consumption,
cognitive radio,
software-defined radio,
dynamic spectrum access
\\

\end{center}

\pagebreak
%%%%%%%%%%%%%%%%%%%%%%%%%%%%%%%%%%%%%%%%%%%%%%%%%%%%%%%%%%%%%%%%%%%%%
%
%             TABLE OF CONTENTS
%
%%%%%%%%%%%%%%%%%%%%%%%%%%%%%%%%%%%%%%%%%%%%%%%%%%%%%%%%%%%%%%%%%%%%%
\tableofcontents
\pagebreak

%\listoffigures
%\pagebreak

%\listoftables
%\pagebreak

\section*{List of Abbreviations}
\begin{acronym}
  \acro{3G}{$3^{rd}$ generation}
  \acro{API}{application program interface}
  \acro{AWGN}{additive white Gauss noise}
  \acro{BER}{bit error rate}
  \acro{CE}{cognitive engine}
  \acro{CIREN}{cognitively intrepid radio emergency network}
  \acro{CPU}{central processing unit}
  \acro{CR}{cognitive radio}
  \acro{DSA}{dynamic spectrum access}
  \acro{DSP}{digital signal processing}
  \acro{EQ}{equalizer}
  \acro{FEC}{forward error correction}
  \acro{FIR}{finite impulse response}
  \acro{FRS}{family radio service}
  \acro{GPP}{general purpose processor}
  \acro{OFDM}{orthogonal frequency division multiplexing}
  \acro{OQAM}{offset quadrature amplitude modulation}
  \acro{QAM}{quadrature amplitude modulation}
  \acro{QoS}{quality of service}
  \acro{PSK}{phase-shift keying}
  \acro{PTP}{point-to-point}
  \acro{SDR}{software-defined radio}
  \acro{SNR}{signal-to-noise ratio}
  \acro{VB}{voice broadcast}
  \acro{VLSI}{very large scale integration}
  \acro{WCDMA}{wideband code divisional multiple access}
\end{acronym}

\pagenumbering{arabic}

%%%%%%%%%%%%%%%%%%%%%%%%%%%%%%%%%%%%%%%%%%%%%%%%%%%%%%%%%%%%%%%%%%%%%
%
%             SECTIONS
%
%%%%%%%%%%%%%%%%%%%%%%%%%%%%%%%%%%%%%%%%%%%%%%%%%%%%%%%%%%%%%%%%%%%%%

% modem, filter, etc.
\section{Introduction}
Key points
\begin{itemize}
\item open-source software-defined radio DSP algorithms
\item minimal dependence on external libraries
\item portable to embedded platforms
\item flexible configuration
\item targets cognitive radios and enabling technologies through
      flexible algorithmic development
\end{itemize}

\subsection{Features}
\begin{itemize}
\item automatic test scripts for validation and accuracy
\item benchmark tool for estimating computational speed on your machine
\end{itemize}


\section{Quick Start Guide}
To build:
\begin{verbatim}
cd /path/to/liquid/
./reconf
./configure
make
sudo make install
\end{verbatim}
You may also want to build and run the optional validation program via
\begin{verbatim}
make check
\end{verbatim}
and the benchmarking tool,
\begin{verbatim}
make bench
\end{verbatim}

\section{Tutorial}

\section{Sandbox}

\subsection{Why C?}
A commonly asked question is ``why C and not C++?''
The answer is simple: {\em portability}.
Our aim is to provide a lightweight DSP library for software-defined radio
that does not rely on a myriad of dependencies.
While C++ is a fine language for many purposes (and theoretically runs just as
fast as C), it is not as portable to embedded platforms as C.
Furthermore, the majority of functions simply perform complex operations on a
data sequence and do not require a high-level object-oriented programming
interface.
This we will leave to framework developers and interface builders.

\subsection{Building/Linking with C++}
Here is an example:
\input{listings/nco.c++.tex}

\section{History}
{\em liquid} was created by J. Gaeddert out of necessity to perform complex
digital signal processing algorithms on embedded devices
without relying on dealing
with proprietary and otherwise cumbersome frameworks.
This was a critical step in his PhD thesis to adapt DSP algorithms in
cognitive dynamic-spectrum radios to optimally manage finite radio resources.
Was he successful?
Put it this way: at the time this document was written he still has not
graduated.\footnote{Come back in a year and ask again... {\em sigh}}

%%%%%%%%%%%%%%%%%%%%%%%%%%%%%%%%%%%%%%%%%%%%%%%%%%%%%%%%%%%%%%%%%%%%%
%
%             Modules
%
%%%%%%%%%%%%%%%%%%%%%%%%%%%%%%%%%%%%%%%%%%%%%%%%%%%%%%%%%%%%%%%%%%%%%
\chapter{Modules}
\label{ch:modules}


\section{agc: automatic gain control}
\section{ann: artificial neural networks}
\section{audio}
source audio encoders/decoders: cvsd, filterbanks (sub-channel coding)
\section{buffer}
internal buffering, circular/static, ports (threaded)
\section{channel}
communications channel modeling: additive noise, multipath fading...
\section{dotprod: dot products (real, complex)}
\section{equalization}
adaptive equalizers: LMS, RLS, blind...
\section{estimation}
parametric estimation: frequency offset, timing offset, fading...
\section{fec: forward error correction}
(basic), checksum, crc, Hamming block codes...
\section{fft: fast Fourier transform}
(basic), simple implementation
\section{filter}
fir, iir, polyphase, hilbert, interp, decim, Farrow, design, remez, etc.
\section{interleaver}
bit- and symbol-level
\section{math}
transcendental functions not in the C standard library (gamma, besseli, etc.)
and polynomial operations
\section{matrix}
basic math, lu-decomp, inversion, gauss-jordan elimination
\section{modem}
modulate, demodulate, psk, dpsk, qam, oqam, aqam, msk, fsk
\section{multicarrier}
OFDM/OQAM, OFDM...
\section{nco}
numerically-controlled oscillator: mix, pll
\section{optim}
newton-raphson, ga, gradient
\section{quantization}
analog/digital converters, companding...
\section{random}
random number generators
\section{sequence}
lfsr, complementary codes
\section{utility}
useful miscellany: bit packing, etc. 


% 
% MODULE : agc (automatic gain control)
%

\section{agc (automatic gain control)}
\label{module:agc}
Normalizing the incoming signal level is a critical step for many wireless
communications protocols, particularly in digital modulation schemes which
encode information in the signal amplitude (e.g. see {\tt MOD\_QAM}).
Furthermore, loop filters for tracking carrier and symbol timing are highly
sensitive to signal levels.
For these reasons and more does gain control play a crucial role in SDR.

\liquid\ implements automatic gain controlling with the {\tt agc\_xxxt}
object.
Operating one sample at a time, the {\tt agc} object makes an estimate
$\hat{e}$ of the signal energy and updates the internal gain $\hat{g}$,
applying it to the input to produce an output with the target energy
$\bar{e}$.

Problem areas: signals that have fluctuating gain...

Loop filter coefficients for a bandwidth $w$,
\[  \alpha = \sqrt{w}   \]
\[  \beta = 1 - \alpha  \]

\subsection{{\tt LIQUID\_AGC\_DEFAULT}}
The default {\tt agc} type is a... For a given input sample $x$ at time $n$,
the filtered signal energy estimate $\hat{e}$ is
\[  q(n) = \zeta\|x\|^2 + (1-\zeta)q(n-1)   \]
\[  \hat{e}(n) = \sqrt{q(n)}                \]
where $q(0)=0$ and $\zeta=0.1$ is a smoothing factor.%
\footnote{This smoothing factor is necessary to help prevent wild fluctuations
in $\hat{e}(n)$ which can occur if the input signal's instantaneous amplitude
has significant variation.}
The instantaneous ideal gain is
\[
    \bar{g}(n) = \bar{e}(n) / \hat{e}(n)
\]
and the filtered gain estimate is
\[
    \hat{g}(n+1) = \beta \hat{g}(n) + \alpha \bar{g}(n)
\]

\subsection{{\tt LIQUID\_AGC\_LOG}}
As in the default case, the log {\tt agc} type makes an estimate of the
instantaneous gain,
\[
    \bar{g}(n) = \bar{e}(n) / \hat{e}(n)
\]
however the loop control operates on the log of the gain error, viz
\[
    \gamma(n) = \ln\bigl( \bar{g}(n) / \hat{g}(n) \bigr)
\]
The {\tt agc} then updates its gain estimate according to the magnitude of the
error.
\[
    \hat{g}(n+1) = \hat{g}(n) e^{ \alpha \gamma(n) }
\]
Notice that when $\gamma(n)=0$, the gain is not updated
(i.e. $\hat{g}(n+1) = \hat{g}(n)$).


\subsection{{\tt LIQUID\_AGC\_EXP}}
The exponential gain update method does not...
The instantaneous output energy is
\[  \|y(n)\| = \hat{g}(n)\|x(n)\|      \]
and the gain error is
\[
    \gamma(n) = \|y(n)\| - \bar{e}
\]
The gain estimate is updated proportional to the gain error, viz
\[
    \hat{g}(n+1) =
        \hat{g}(n) \left(1 -
         \frac{
            \gamma(n)\sqrt{\alpha}
        }{
            \text{argmax}\left\{\bar{e}(n),\|y(n)\|\right\}
        }
    \right)
\]
The multiplier in to the right of the above equation is always positive
and proportional to the gain error.
Notice that when the error is zero ($\|y(n)\| = \bar{e}$),
$\hat{g}(n+1) = \hat{g}$

\subsection{Squelch}
The {\tt agc} module contains internal squelch control to allow the
controlling unit the ability to disable signal processing when the signal
level is too low.
In traditional radio design, the squelch circuit was used to suppress the
output of a receiver when the signal strength falls below a certain level,
primarily used to disable annoying static due to noise when no other operators
were transmitting.
Having said that, he squelch control in \liquid\ is actually somewhat of a
misnomer as it doesn't actually control the AGC, but rather just monitors the
dynamics of the signal level and returns its status to the controlling unit.
The squelch control follows six states
(enabled, rising edge trigger, signal high, falling edge trigger,
signal low, and timeout)
as depicted in
Figure~\ref{fig:module:agc:squelch} and
Table~\ref{tab:module:agc:squelch_codes}.
These states give the user flexibility in programming networks where packets
are transmitted in short bursts and the receiver needs to synchronize quickly.
The status of the squelch control is retrieved via the
{\tt agc\_crcf\_squelch\_get\_status()} method.

The typical control cycle for the AGC squelch is depicted in
Figure~\ref{fig:module:agc:squelch}.
Initially, squelch is enabled (code {\tt 0}) as the signal has been low for
quite some time.
When the beginning of a packet is received, the RSSI increases beyond the
squelch threshold (code {\tt 1}).
All subsequent samples above this threshold return a ``signal high'' status
(code {\tt 2}).
Once the signal level falls below the threshold, the squelch returns a
``falling edge trigger'' status (code {\tt 3}).
All subsequent samples below the threshold until timing out return a ``signal
low'' status (code {\tt 4}).
When the signal has been low for a sufficient period of time (defined by the
user), the squelch will return a ``timeout'' status (code {\tt 5}).
All subsequent samples below the threshold will return a ``squelch enabled''
status.

\subsubsection{methodology}
The reason for all six states (as opposed to just ``squelch on'' and ``squelch
off'') are to allow for the AGC to adjust to complex signal dynamics.
The default operation for the AGC is to {\it disable} the squelch.
For example if the AGC squelch control is in ``signal low'' mode
(state {\tt 4}) and the signal increases above the threshold before timeout,
the AGC will move back to the ``signal high'' mode (state {\tt 2}).
This is particularly useful for weak signals whose received signal strength is
hovering around the squelch threshold; it would be undesireable for the AGC to
enable the squelch in the middle of receiving a packet!

\subsubsection{auto-squelch}
The AGC module also allows for an auto-squelch mechanism which attempts to
track the signal threshold to the noise floor of the receiver.
This is accomplished by monitoring the signal level when squelch is enabled.
The auto-squelch mechanism has a 4dB headroom; if the signal level drops below
4dB beneath the squelch threshold, the threshold will be decremented.
This is useful for receiving weak signals slightly above the noise floor,
particularly when the exact noise floor is not known or varies slightly over
time.
Auto-squelch is enabled/disabled using the
{\tt agc\_crcf\_squelch\_enable\_auto()} and 
{\tt agc\_crcf\_squelch\_disable\_auto()} methods respectively.

\begin{figure}
\centering
  \includegraphics[trim = 0mm 0mm 0mm 0mm, clip, width=16cm]{figures.gen/agc_squelch_pgf}
\caption{{\tt agc\_crcf} squelch}
\label{fig:module:agc:squelch}
\end{figure}


% ------------ TABLE: AGC SQUELCH CODES ------------
\begin{table}[!ht]
\caption{{\tt agc} squelch codes}
\label{tab:module:agc:squelch_codes}
\centering
\begin{tabular*}{0.95\textwidth}{@{\extracolsep{\fill}}lll}

\hline\hline \\[-6pt]
{\bf code} & {\bf id} & {\bf description} \\[6pt]
\hline \\[-6pt]
{\tt 0} & {\tt LIQUID\_AGC\_SQUELCH\_ENABLED}    & squelch enabled \\
{\tt 1} & {\tt LIQUID\_AGC\_SQUELCH\_RISE}       & rising edge trigger \\
{\tt 2} & {\tt LIQUID\_AGC\_SQUELCH\_SIGNALHI}   & signal level high \\
{\tt 3} & {\tt LIQUID\_AGC\_SQUELCH\_FALL}       & falling edge trigger \\
{\tt 4} & {\tt LIQUID\_AGC\_SQUELCH\_SIGNALLO}   & signal level low, but no timeout \\
{\tt 5} & {\tt LIQUID\_AGC\_SQUELCH\_TIMEOUT}    & signal level low, timeout \\ \\[-6pt]

\hline\hline
\end{tabular*}
\end{table}%
% ------------------------

\subsection{Usage}
Basic usage:
\input{listings/agc.example.c.tex}


    
%%%%%%%%%%%%%%%%%%%%%%%%%%%%%%%%%%%%%%%%%%%%%%%%%%%%%%%%%%%%%%%%%%%%%
%
%             BIBLIOGRAPHY
%
%%%%%%%%%%%%%%%%%%%%%%%%%%%%%%%%%%%%%%%%%%%%%%%%%%%%%%%%%%%%%%%%%%%%%
\cleardoublepage
%\bibliography{../IEEEtranBST/IEEEabrv,liquid}


% In LaTeX, each appendix is a "chapter"
\appendix
%%%%%%%%%%%%%%%%%%%%%%%%%%%%%%%%%%%%%%%%%%%%%%%%%%%%%%%%%%%%%%%%%%%%%
%
%             APPENDIX A
%
%%%%%%%%%%%%%%%%%%%%%%%%%%%%%%%%%%%%%%%%%%%%%%%%%%%%%%%%%%%%%%%%%%%%%
\end{document}

