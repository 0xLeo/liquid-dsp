% ------------------------------------------------------------------
%
%   TITLE: LIQUID documentation
% AUTHORS: Joseph D. Gaeddert et al.
% CREATED: February 7, 2009
% REVISED: 
%     URL: http://computing.ece.vt.edu/~jgaeddert/
%          https://ganymede.ece.vt.edu/trac/liquid/
%
% ------------------------------------------------------------------
 
\documentclass[11pt,twoside]{report}
% ------------------- DOCUMENT VARIABLES -------------------
\setlength{\textwidth}{6.5in}
\setlength{\textheight}{8.5in}
\setlength{\evensidemargin}{0in}
\setlength{\oddsidemargin}{0in}
\setlength{\topmargin}{0in}
\setlength{\parindent}{0pt}
\setlength{\parskip}{0.1in}

% ------------------- BIBLIOGRAPHY STYLE -------------------
\usepackage{natbib}
%\bibliographystyle{plain} % note the change here
\bibliographystyle{IEEEtran}
\bibpunct{[}{]}{,}{n}{}{;}  % define bibliography punctuation
% The six mandatory arguments for \bibpunct are:
%   1. opening bracket: '(', '[', '{', or '<'
%   2. closing bracket: ')', ']', '}', or '>'
%   3. separator between multiple citations: ';' or ','
%   4. citation style: 'n' for numerical style, 's' for numerical superscript style, or 'a' for author�year style
%   5. punctuation between the author names and the year
%   6. punctuation between years or numbers when common author lists are suppressed: ',' or ';' 

% ------------------- GRAPHICS PACKAGES -------------------
%\usepackage{epsf}
%\usepackage{graphicx}
\ifx\pdfoutput\undefined
\usepackage{graphicx}
\else
\usepackage[pdftex]{graphicx}
\fi
\usepackage{epsfig}
\usepackage{epstopdf}
\usepackage{colortbl}
\usepackage{color}

%\usepackage{tabularx}
\usepackage{ctable} % \toprule, \midrule, \bottomrule
\setlength{\heavyrulewidth}{0.1em} % modify thickness of \toprule, \bottomrule
\newcommand{\otoprule}{\midrule[\heavyrulewidth]}
\usepackage{subfigure}
\usepackage{multirow} % for tables
%\usepackage{fancyhdr}
\usepackage{amsmath}
\usepackage{listings}
\usepackage{fancyvrb}
\usepackage{acronym}

% tabular: \hline (thin) \hline\hline (thick)
% from: http://www.faqs.org/faqs/tex-faq/ (see #44)
\setlength{\doublerulesep}{\arrayrulewidth}

%--------- NEW COMMANDS -------------------
\newcommand{\NA}{\scriptsize{NA}}
\newcommand{\E}{\scriptsize{E}}
\newcommand{\tRMS}{\ensuremath{\tau_{rms}}}
\newcommand{\tmED}{\ensuremath{\bar{\tau}}}
\newcommand{\Np}{\ensuremath{N_p}}
\newcommand{\ua}{\ensuremath{\uparrow}}
\newcommand{\da}{\ensuremath{\downarrow}}
\newcommand{\sinc}{\textup{sinc}}
\newcommand{\etal}{{\it et al.}}
\newcommand{\wavt}{W@VT}
\renewcommand{\vec}[1]{\boldsymbol{#1}}
\newcommand{\ciren}{C{\sc iren}}

%--------- CAPTION OPTIONS -------------------
\usepackage[small,bf]{caption}
\setcaptionwidth{15cm}
\setlength{\belowcaptionskip}{0.5cm}

%%%%%%%%%%%%%%%%%%%%%%%%%%%%%%%%%%%%%%%%%%%%%%%%%%%%%%%%%%%%%%%%%%%%%
%
%             MAIN DOCUMENT
%
%%%%%%%%%%%%%%%%%%%%%%%%%%%%%%%%%%%%%%%%%%%%%%%%%%%%%%%%%%%%%%%%%%%%%

\begin{document}

% ------------------- DEFINE LISTINGS -------------------
\input{highlight.sty}


%%%%%%%%%%%%%%%%%%%%%%%%%%%%%%%%%%%%%%%%%%%%%%%%%%%%%%%%%%%%%%%%%%%%%
%
%             TITLE PAGE
%
%%%%%%%%%%%%%%%%%%%%%%%%%%%%%%%%%%%%%%%%%%%%%%%%%%%%%%%%%%%%%%%%%%%%%
\thispagestyle{empty}
\pagenumbering{roman}
\begin{center}

{\huge liquid} \\
Software-defined radio digital signal processing library

\vfill

Joseph D. Gaeddert

\vfill

February 7, 2009 \\
Blacksburg, Virginia

\vfill

{\it Keywords:}
polyphase filterbanks,
OFDM/OQAM,
power consumption,
cognitive radio,
software-defined radio,
dynamic spectrum access
\\

\end{center}

\pagebreak
%%%%%%%%%%%%%%%%%%%%%%%%%%%%%%%%%%%%%%%%%%%%%%%%%%%%%%%%%%%%%%%%%%%%%
%
%             TABLE OF CONTENTS
%
%%%%%%%%%%%%%%%%%%%%%%%%%%%%%%%%%%%%%%%%%%%%%%%%%%%%%%%%%%%%%%%%%%%%%
\tableofcontents
\pagebreak

%\listoffigures
%\pagebreak

%\listoftables
%\pagebreak

\section*{List of Abbreviations}
\begin{acronym}
  \acro{3G}{$3^{rd}$ generation}
  \acro{API}{application program interface}
  \acro{AWGN}{additive white Gauss noise}
  \acro{BER}{bit error rate}
  \acro{CE}{cognitive engine}
  \acro{CIREN}{cognitively intrepid radio emergency network}
  \acro{CPU}{central processing unit}
  \acro{CR}{cognitive radio}
  \acro{DSA}{dynamic spectrum access}
  \acro{DSP}{digital signal processing}
  \acro{EQ}{equalizer}
  \acro{FEC}{forward error correction}
  \acro{FIR}{finite impulse response}
  \acro{FRS}{family radio service}
  \acro{GPP}{general purpose processor}
  \acro{OFDM}{orthogonal frequency division multiplexing}
  \acro{OQAM}{offset quadrature amplitude modulation}
  \acro{QAM}{quadrature amplitude modulation}
  \acro{QoS}{quality of service}
  \acro{PSK}{phase-shift keying}
  \acro{PTP}{point-to-point}
  \acro{SDR}{software-defined radio}
  \acro{SNR}{signal-to-noise ratio}
  \acro{VB}{voice broadcast}
  \acro{VLSI}{very large scale integration}
  \acro{WCDMA}{wideband code divisional multiple access}
\end{acronym}

\pagenumbering{arabic}

%%%%%%%%%%%%%%%%%%%%%%%%%%%%%%%%%%%%%%%%%%%%%%%%%%%%%%%%%%%%%%%%%%%%%
%
%             SECTIONS
%
%%%%%%%%%%%%%%%%%%%%%%%%%%%%%%%%%%%%%%%%%%%%%%%%%%%%%%%%%%%%%%%%%%%%%

% modem, filter, etc.
\section{Introduction}
Key points
\begin{itemize}
\item open-source software-defined radio DSP algorithms
\item minimal dependence on external libraries
\item portable to embedded platforms
\item flexible configuration
\end{itemize}

\subsection{Features}
\begin{itemize}
\item automatic test scripts for validation and accuracy
\item benchmark tool for estimating computational speed on your machine
\end{itemize}


\section{Quick Start Guide}

\section{Tutorial}

\section{Sandbox}

\subsection{Why C?}
A commonly asked question is ``why C and not C++?''
The answer is simple: {\em portability}.
Our aim is to provide a lightweight DSP library for software-defined radio
that does not rely on a miriad of dependencies.
While C++ is a fine language for many purposes (and theoretically runs just as
fast as C), it is not as portable to embedded platforms as C.
Furthermore, the majority of functions simply perform complex operations on a
data sequence and do not require a high-level object-oriented programming
interface.
This we will leave to framework developers and interface builders.

\subsection{Building/Linking with C++}
Here is an example:
\input{listings/nco.c++.tex}

%%%%%%%%%%%%%%%%%%%%%%%%%%%%%%%%%%%%%%%%%%%%%%%%%%%%%%%%%%%%%%%%%%%%%
%
%             BIBLIOGRAPHY
%
%%%%%%%%%%%%%%%%%%%%%%%%%%%%%%%%%%%%%%%%%%%%%%%%%%%%%%%%%%%%%%%%%%%%%
\cleardoublepage
%\bibliography{../IEEEtranBST/IEEEabrv,liquid}


% In LaTeX, each appendix is a "chapter"
\appendix
%%%%%%%%%%%%%%%%%%%%%%%%%%%%%%%%%%%%%%%%%%%%%%%%%%%%%%%%%%%%%%%%%%%%%
%
%             APPENDIX A
%
%%%%%%%%%%%%%%%%%%%%%%%%%%%%%%%%%%%%%%%%%%%%%%%%%%%%%%%%%%%%%%%%%%%%%
\end{document}

