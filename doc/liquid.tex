% ------------------------------------------------------------------
%
%   TITLE: LIQUID documentation
% AUTHORS: Joseph D. Gaeddert et al.
% CREATED: February 7, 2009
% REVISED: 
%     URL: http://computing.ece.vt.edu/~jgaeddert/
%          https://ganymede.ece.vt.edu/trac/liquid/
%
% ------------------------------------------------------------------
 
\documentclass[11pt,twoside]{report}
% ------------------- DOCUMENT VARIABLES -------------------
\setlength{\textwidth}{6.5in}
\setlength{\textheight}{8.5in}
\setlength{\evensidemargin}{0in}
\setlength{\oddsidemargin}{0in}
\setlength{\topmargin}{0in}
\setlength{\parindent}{0pt}
\setlength{\parskip}{0.1in}

% ------------------- BIBLIOGRAPHY STYLE -------------------
\usepackage{natbib}
%\bibliographystyle{plain} % note the change here
\bibliographystyle{IEEEtran}
\bibpunct{[}{]}{,}{n}{}{;}  % define bibliography punctuation
% The six mandatory arguments for \bibpunct are:
%   1. opening bracket: '(', '[', '{', or '<'
%   2. closing bracket: ')', ']', '}', or '>'
%   3. separator between multiple citations: ';' or ','
%   4. citation style: 'n' for numerical style, 's' for numerical superscript
%      style, or 'a' for author year style
%   5. punctuation between the author names and the year
%   6. punctuation between years or numbers when common author lists are suppressed: ',' or ';' 

% ------------------- GRAPHICS PACKAGES -------------------
%\usepackage{epsf}
%\usepackage{graphicx}
\ifx\pdfoutput\undefined
\usepackage{graphicx}
\else
\usepackage[pdftex]{graphicx}
\fi
\usepackage{epsfig}
\usepackage{epstopdf}
\usepackage{colortbl}
\usepackage{color}

%\usepackage{tabularx}
\usepackage{ctable} % \toprule, \midrule, \bottomrule
\setlength{\heavyrulewidth}{0.1em} % modify thickness of \toprule, \bottomrule
\newcommand{\otoprule}{\midrule[\heavyrulewidth]}
\usepackage{subfigure}
\usepackage{multirow} % for tables
%\usepackage{fancyhdr}
\usepackage{amsmath}
\usepackage{listings}
\usepackage{fancyvrb}
\usepackage{acronym}

% tabular: \hline (thin) \hline\hline (thick)
% from: http://www.faqs.org/faqs/tex-faq/ (see #44)
\setlength{\doublerulesep}{\arrayrulewidth}

%--------- NEW COMMANDS -------------------
\newcommand{\NA}{\scriptsize{NA}}
\newcommand{\E}{\scriptsize{E}}
\newcommand{\tRMS}{\ensuremath{\tau_{rms}}}
\newcommand{\tmED}{\ensuremath{\bar{\tau}}}
\newcommand{\Np}{\ensuremath{N_p}}
\newcommand{\ua}{\ensuremath{\uparrow}}
\newcommand{\da}{\ensuremath{\downarrow}}
\newcommand{\sinc}{\textup{sinc}}
\newcommand{\etal}{{\it et al.}}
\newcommand{\wavt}{W@VT}
\renewcommand{\vec}[1]{\boldsymbol{#1}}
\newcommand{\ciren}{C{\sc iren}}

%--------- CAPTION OPTIONS -------------------
\usepackage[small,bf]{caption}
\setcaptionwidth{15cm}
\setlength{\belowcaptionskip}{0.5cm}

%%%%%%%%%%%%%%%%%%%%%%%%%%%%%%%%%%%%%%%%%%%%%%%%%%%%%%%%%%%%%%%%%%%%%
%
%             MAIN DOCUMENT
%
%%%%%%%%%%%%%%%%%%%%%%%%%%%%%%%%%%%%%%%%%%%%%%%%%%%%%%%%%%%%%%%%%%%%%

\begin{document}

% ------------------- DEFINE LISTINGS -------------------
\input{highlight.sty}


%%%%%%%%%%%%%%%%%%%%%%%%%%%%%%%%%%%%%%%%%%%%%%%%%%%%%%%%%%%%%%%%%%%%%
%
%             TITLE PAGE
%
%%%%%%%%%%%%%%%%%%%%%%%%%%%%%%%%%%%%%%%%%%%%%%%%%%%%%%%%%%%%%%%%%%%%%
\thispagestyle{empty}
\pagenumbering{roman}
\begin{center}

{\huge liquid} \\
Software-defined radio digital signal processing library

\vfill

Joseph D. Gaeddert

\vfill

February 7, 2009 \\
Blacksburg, Virginia

\vfill

{\it Keywords:}
polyphase filterbanks,
OFDM/OQAM,
power consumption,
cognitive radio,
software-defined radio,
dynamic spectrum access
\\

\end{center}

\pagebreak
%%%%%%%%%%%%%%%%%%%%%%%%%%%%%%%%%%%%%%%%%%%%%%%%%%%%%%%%%%%%%%%%%%%%%
%
%             TABLE OF CONTENTS
%
%%%%%%%%%%%%%%%%%%%%%%%%%%%%%%%%%%%%%%%%%%%%%%%%%%%%%%%%%%%%%%%%%%%%%
\tableofcontents
\pagebreak

%\listoffigures
%\pagebreak

%\listoftables
%\pagebreak

\section*{List of Abbreviations}
\begin{acronym}
  \acro{3G}{$3^{rd}$ generation}
  \acro{API}{application program interface}
  \acro{AWGN}{additive white Gauss noise}
  \acro{BER}{bit error rate}
  \acro{CE}{cognitive engine}
  \acro{CIREN}{cognitively intrepid radio emergency network}
  \acro{CPU}{central processing unit}
  \acro{CR}{cognitive radio}
  \acro{DSA}{dynamic spectrum access}
  \acro{DSP}{digital signal processing}
  \acro{EQ}{equalizer}
  \acro{FEC}{forward error correction}
  \acro{FIR}{finite impulse response}
  \acro{FRS}{family radio service}
  \acro{GPP}{general purpose processor}
  \acro{OFDM}{orthogonal frequency division multiplexing}
  \acro{OQAM}{offset quadrature amplitude modulation}
  \acro{QAM}{quadrature amplitude modulation}
  \acro{QoS}{quality of service}
  \acro{PSK}{phase-shift keying}
  \acro{PTP}{point-to-point}
  \acro{SDR}{software-defined radio}
  \acro{SNR}{signal-to-noise ratio}
  \acro{VB}{voice broadcast}
  \acro{VLSI}{very large scale integration}
  \acro{WCDMA}{wideband code divisional multiple access}
\end{acronym}

\pagenumbering{arabic}

%%%%%%%%%%%%%%%%%%%%%%%%%%%%%%%%%%%%%%%%%%%%%%%%%%%%%%%%%%%%%%%%%%%%%
%
%             SECTIONS
%
%%%%%%%%%%%%%%%%%%%%%%%%%%%%%%%%%%%%%%%%%%%%%%%%%%%%%%%%%%%%%%%%%%%%%

% modem, filter, etc.
\section{Introduction}
Key points
\begin{itemize}
\item open-source software-defined radio DSP algorithms
\item minimal dependence on external libraries
\item portable to embedded platforms
\item flexible configuration
\item targets cognitive radios and enabling technologies through
      flexible algorithmic development
\end{itemize}

\subsection{Features}
\begin{itemize}
\item automatic test scripts for validation and accuracy
\item benchmark tool for estimating computational speed on your machine
\end{itemize}


\section{Quick Start Guide}
To build:
\begin{verbatim}
cd /path/to/liquid/
./reconf
./configure
make
sudo make install
\end{verbatim}
You may also want to build and run the optional validation program via
\begin{verbatim}
make check
\end{verbatim}
and the benchmarking tool,
\begin{verbatim}
make bench
\end{verbatim}

\section{Tutorial}

\section{Sandbox}

\subsection{Why C?}
A commonly asked question is ``why C and not C++?''
The answer is simple: {\em portability}.
Our aim is to provide a lightweight DSP library for software-defined radio
that does not rely on a myriad of dependencies.
While C++ is a fine language for many purposes (and theoretically runs just as
fast as C), it is not as portable to embedded platforms as C.
Furthermore, the majority of functions simply perform complex operations on a
data sequence and do not require a high-level object-oriented programming
interface.
This we will leave to framework developers and interface builders.

\subsection{Building/Linking with C++}
Here is an example:
\input{listings/nco.c++.tex}

\section{History}
{\em liquid} was created by J. Gaeddert out of necessity to perform complex
digital signal processing algorithms on embedded devices
without relying on dealing
with proprietary and otherwise cumbersome frameworks.
This was a critical step in his PhD thesis to adapt DSP algorithms in
cognitive dynamic-spectrum radios to optimally manage finite radio resources.
Was he successful?
Put it this way: at the time this document was written he still has not
graduated.\footnote{Come back in a year and ask again... {\em sigh}}

%%%%%%%%%%%%%%%%%%%%%%%%%%%%%%%%%%%%%%%%%%%%%%%%%%%%%%%%%%%%%%%%%%%%%
%
%             Modules
%
%%%%%%%%%%%%%%%%%%%%%%%%%%%%%%%%%%%%%%%%%%%%%%%%%%%%%%%%%%%%%%%%%%%%%
\chapter{Modules}
\label{ch:modules}


% 
% MODULE : agc (automatic gain control)
%

\section{agc (automatic gain control)}
\label{module:agc}
Normalizing the incoming signal level is a critical step for many wireless
communications protocols, particularly in digital modulation schemes which
encode information in the signal amplitude (e.g. see {\tt MOD\_QAM}).
Furthermore, loop filters for tracking carrier and symbol timing are highly
sensitive to signal levels.
For these reasons and more does gain control play a crucial role in SDR.

\liquid\ implements automatic gain controlling with the {\tt agc\_xxxt}
object.
Operating one sample at a time, the {\tt agc} object makes an estimate
$\hat{e}$ of the signal energy and updates the internal gain $\hat{g}$,
applying it to the input to produce an output with the target energy
$\bar{e}$.

Problem areas: signals that have fluctuating gain...

Loop filter coefficients for a bandwidth $w$,
\[  \alpha = \sqrt{w}   \]
\[  \beta = 1 - \alpha  \]

\subsection{{\tt LIQUID\_AGC\_DEFAULT}}
The default {\tt agc} type is a... For a given input sample $x$ at time $n$,
the filtered signal energy estimate $\hat{e}$ is
\[  q(n) = \zeta\|x\|^2 + (1-\zeta)q(n-1)   \]
\[  \hat{e}(n) = \sqrt{q(n)}                \]
where $q(0)=0$ and $\zeta=0.1$ is a smoothing factor.%
\footnote{This smoothing factor is necessary to help prevent wild fluctuations
in $\hat{e}(n)$ which can occur if the input signal's instantaneous amplitude
has significant variation.}
The instantaneous ideal gain is
\[
    \bar{g}(n) = \bar{e}(n) / \hat{e}(n)
\]
and the filtered gain estimate is
\[
    \hat{g}(n+1) = \beta \hat{g}(n) + \alpha \bar{g}(n)
\]

\subsection{{\tt LIQUID\_AGC\_LOG}}
As in the default case, the log {\tt agc} type makes an estimate of the
instantaneous gain,
\[
    \bar{g}(n) = \bar{e}(n) / \hat{e}(n)
\]
however the loop control operates on the log of the gain error, viz
\[
    \gamma(n) = \ln\bigl( \bar{g}(n) / \hat{g}(n) \bigr)
\]
The {\tt agc} then updates its gain estimate according to the magnitude of the
error.
\[
    \hat{g}(n+1) = \hat{g}(n) e^{ \alpha \gamma(n) }
\]
Notice that when $\gamma(n)=0$, the gain is not updated
(i.e. $\hat{g}(n+1) = \hat{g}(n)$).


\subsection{{\tt LIQUID\_AGC\_EXP}}
The exponential gain update method does not...
The instantaneous output energy is
\[  \|y(n)\| = \hat{g}(n)\|x(n)\|      \]
and the gain error is
\[
    \gamma(n) = \|y(n)\| - \bar{e}
\]
The gain estimate is updated proportional to the gain error, viz
\[
    \hat{g}(n+1) =
        \hat{g}(n) \left(1 -
         \frac{
            \gamma(n)\sqrt{\alpha}
        }{
            \text{argmax}\left\{\bar{e}(n),\|y(n)\|\right\}
        }
    \right)
\]
The multiplier in to the right of the above equation is always positive
and proportional to the gain error.
Notice that when the error is zero ($\|y(n)\| = \bar{e}$),
$\hat{g}(n+1) = \hat{g}$

\subsection{Squelch}
The {\tt agc} module contains internal squelch control to allow the
controlling unit the ability to disable signal processing when the signal
level is too low.
In traditional radio design, the squelch circuit was used to suppress the
output of a receiver when the signal strength falls below a certain level,
primarily used to disable annoying static due to noise when no other operators
were transmitting.
Having said that, he squelch control in \liquid\ is actually somewhat of a
misnomer as it doesn't actually control the AGC, but rather just monitors the
dynamics of the signal level and returns its status to the controlling unit.
The squelch control follows six states
(enabled, rising edge trigger, signal high, falling edge trigger,
signal low, and timeout)
as depicted in
Figure~\ref{fig:module:agc:squelch} and
Table~\ref{tab:module:agc:squelch_codes}.
These states give the user flexibility in programming networks where packets
are transmitted in short bursts and the receiver needs to synchronize quickly.
The status of the squelch control is retrieved via the
{\tt agc\_crcf\_squelch\_get\_status()} method.

The typical control cycle for the AGC squelch is depicted in
Figure~\ref{fig:module:agc:squelch}.
Initially, squelch is enabled (code {\tt 0}) as the signal has been low for
quite some time.
When the beginning of a packet is received, the RSSI increases beyond the
squelch threshold (code {\tt 1}).
All subsequent samples above this threshold return a ``signal high'' status
(code {\tt 2}).
Once the signal level falls below the threshold, the squelch returns a
``falling edge trigger'' status (code {\tt 3}).
All subsequent samples below the threshold until timing out return a ``signal
low'' status (code {\tt 4}).
When the signal has been low for a sufficient period of time (defined by the
user), the squelch will return a ``timeout'' status (code {\tt 5}).
All subsequent samples below the threshold will return a ``squelch enabled''
status.

\subsubsection{methodology}
The reason for all six states (as opposed to just ``squelch on'' and ``squelch
off'') are to allow for the AGC to adjust to complex signal dynamics.
The default operation for the AGC is to {\it disable} the squelch.
For example if the AGC squelch control is in ``signal low'' mode
(state {\tt 4}) and the signal increases above the threshold before timeout,
the AGC will move back to the ``signal high'' mode (state {\tt 2}).
This is particularly useful for weak signals whose received signal strength is
hovering around the squelch threshold; it would be undesireable for the AGC to
enable the squelch in the middle of receiving a packet!

\subsubsection{auto-squelch}
The AGC module also allows for an auto-squelch mechanism which attempts to
track the signal threshold to the noise floor of the receiver.
This is accomplished by monitoring the signal level when squelch is enabled.
The auto-squelch mechanism has a 4dB headroom; if the signal level drops below
4dB beneath the squelch threshold, the threshold will be decremented.
This is useful for receiving weak signals slightly above the noise floor,
particularly when the exact noise floor is not known or varies slightly over
time.
Auto-squelch is enabled/disabled using the
{\tt agc\_crcf\_squelch\_enable\_auto()} and 
{\tt agc\_crcf\_squelch\_disable\_auto()} methods respectively.

\begin{figure}
\centering
  \includegraphics[trim = 0mm 0mm 0mm 0mm, clip, width=16cm]{figures.gen/agc_squelch_pgf}
\caption{{\tt agc\_crcf} squelch}
\label{fig:module:agc:squelch}
\end{figure}


% ------------ TABLE: AGC SQUELCH CODES ------------
\begin{table}[!ht]
\caption{{\tt agc} squelch codes}
\label{tab:module:agc:squelch_codes}
\centering
\begin{tabular*}{0.95\textwidth}{@{\extracolsep{\fill}}lll}

\hline\hline \\[-6pt]
{\bf code} & {\bf id} & {\bf description} \\[6pt]
\hline \\[-6pt]
{\tt 0} & {\tt LIQUID\_AGC\_SQUELCH\_ENABLED}    & squelch enabled \\
{\tt 1} & {\tt LIQUID\_AGC\_SQUELCH\_RISE}       & rising edge trigger \\
{\tt 2} & {\tt LIQUID\_AGC\_SQUELCH\_SIGNALHI}   & signal level high \\
{\tt 3} & {\tt LIQUID\_AGC\_SQUELCH\_FALL}       & falling edge trigger \\
{\tt 4} & {\tt LIQUID\_AGC\_SQUELCH\_SIGNALLO}   & signal level low, but no timeout \\
{\tt 5} & {\tt LIQUID\_AGC\_SQUELCH\_TIMEOUT}    & signal level low, timeout \\ \\[-6pt]

\hline\hline
\end{tabular*}
\end{table}%
% ------------------------

\subsection{Usage}
Basic usage:
\input{listings/agc.example.c.tex}



\section{ann: artificial neural networks}

% Audio documentation / sandbox
% 
% MODULE : audio
%

\newpage
\section{audio}
\label{module:audio}
The audio module in \liquid\ provides several objects and functions for
compressing, digitizing, and manipulating audio signals.
This is particularly useful for encoding audio data for wireless
communications.

\subsection{{\tt cvsd} (continuously variable slope delta)}
\label{module:audio:cvsd}
Continuously variable slope delta (CVSD) source encoding is used for data
compression of audio signals.
CVSD is a lossy compression whose quality is directly related to the sampling
frequency and is generally most practical for speech applications.
It is a form of delta modulation where $\Delta$ (the step size) is changed
continuously to minimize slope-overload distortion \cite[p. 131]{Proakis:2001}.
The output bit stream has a rate equal to that of the sampling frequency.
It is considered to be a moderate compromise between quality and complexity.

\subsubsection{Theory}
\label{module:audio:cvsd:theory}
The algorithm attempts to dynamically adjust $\Delta$ value to track
to the input signal.
As with regular delta modulation algorithms,
if the decoded reference signal exceeds the input (the error signal is
negative), a binary {\tt 0} is sent and $\Delta$ is subtracted from the
reference, otherwise a binary {\tt 1} is sent and $\Delta$ is added.
However CVSD observes the previous $N$ transmitted bits are stored in memory.
$\Delta$ is increased by $\zeta$ if they are equal, and decreased otherwise.
This improves the dynamic range of the encoder over fixed-delta modulation
encoders.


\subsubsection{Pre-/Post-Filtering}
\label{module:audio:cvsd:filtering}
To preserve the signal's integrity the encoder applies a pre-filter
to emphasize the high-frequency information of the signal before the
encoding process.
The pre-filter is a simple 2-tap FIR filter defined as
%
\begin{equation}
    H_{pre}(z) = 1 - \alpha z^{-1}
\end{equation}
%
where $\alpha$ controls the amount of emphasis applied.
Typical values fore pre-emphasis are $0.92 < \alpha < 0.98$;
setting $\alpha=0$ completely disables this emphasis.
%
This process is reversed on the decoder by applying the inverse of
$H_{pre}(z)$ as a low-pass de-emphasis filter:
%
\begin{equation}
    H_{pre}^{-1}(z) = 
        \frac{ 1 }{ 1 - \alpha z^{-1} }
\end{equation}
%
Additionally, the decoder adds a DC-blocking filter to reject any
residual offset caused by the decoding process.
By itself the DC-blocking filter has a transfer function
%
\begin{equation}
    H_{0}(z) = 
        \frac{ 1 - z^{-1} }{ 1 - \beta z^{-1} }
\end{equation}
%
where $\beta$ controls the cut-off frequency of the filter and is
typically set very close to 1.
The default value for $\beta$ in \liquid\ is 0.99.
The full post-emphasis filter is therefore
%
\begin{equation}
    H_{post}(z) = 
    H_{pre}^{-1}(z) H_0(z) =
        \frac{
            1 - z^{-1}
        }{
            1 - (\alpha + \beta) z^{-1} + \alpha\beta z^{-2}
        }
\end{equation}
%

\subsubsection{Interface}
\label{module:audio:cvsd:interface}


The {\tt cvsd} object in \liquid\ allows the user to select both $\zeta$
as well as $N$, the number of repeated bits observed before $\Delta$ is
updated.
The combination of these values with the sampling rate yields a speech
compression algorithm with moderate quality.
Listed below is the full interface to the {\tt cvsd} object:
%
\begin{description}
\item[{\tt cvsd\_create(N,zeta,alpha)}]
    creates an {\tt agc} object with parameters $N$, $\zeta$, and
    $\alpha$.
\item[{\tt cvsd\_destroy(q)}]
    destroys a {\tt cvsd} object, freeing all internally-allocated
    memory and objects.
\item[{\tt cvsd\_print(q)}]
    prints the {\tt cvsd} object's internal parameters to the standard
    output.
\item[{\tt cvsd\_encode(q,sample)}]
    encodes a single audio sample, returning the encoded bit.
\item[{\tt cvsd\_decode(q,bit)}]
    decodes and returns a single audio sample from an input bit.
\item[{\tt cvsd\_encode8(q,samples,byte)}]
    encodes a block of 8 samples returning the result in a single byte.
\item[{\tt cvsd\_decode8(q,byte,samples)}]
    decodes a block of 8 samples from an encoded byte.
\end{description}

\subsubsection{Example}
\label{module:audio:cvsd:example}

Here is a basic example of the {\tt cvsd} object in \liquid:
%
\input{listings/cvsd.example.c.tex}
%
A demonstration of the algorithm can be seen in
Figure~\ref{fig:module:audio:cvsd} where the encoder attempts to track to an
input sinusoid.
Notice that the encoder sometimes overshoots the reference signal.
This distortion results in degradations, particularly in the upper frequency
bands.
%
\begin{figure}
\centering
  \includegraphics[trim = 0mm 0mm 0mm 0mm, clip, width=13cm]{figures.gen/audio_cvsd}
\caption{
    {\tt cvsd} example encoding a windowed sine function
    with $\zeta=1.5$, $N=3$, and $\alpha=0.95$.}
\label{fig:module:audio:cvsd}
\end{figure}
%
A more detailed example is given in
{\tt examples/cvsd\_example.c}
under the main \liquid\ project directory.

%source audio encoders/decoders: cvsd, filterbanks (sub-channel coding)

% 
% MODULE : buffer
%

\newpage
\section{buffer}
\label{module:buffer}
The buffer module includes objects for storing, retrieving, and
interfacing with buffered data samples.


\subsection{{\tt window} buffer}
\label{module:buffer:window}
The {\tt window} object is used to implement a sliding window buffer.
It is essentially a first-in, first-out queue but with the constraint that a
fixed number of elements is always available, and the ability to read the
entire queue at once.
This is particularly useful for filtering objects which use time-domain
convolution of a fixed length to compute its outputs.
Unlike the {\tt gport} object, {\tt window} objects operate on a known data
type, e.g.
{\it float} ({\tt windowf}), and
{\it float complex} ({\tt windowcf}).
%{\it unsigned int} ({\tt uiwindow}).

The buffer has a fixed number of elements which are initially zeros.
Values may be pushed into the end of the buffer (into the ``right'' side)
using the {\tt push()} method, or written in blocks via {\tt write()}.
In both cases the oldest data samples are removed from the buffer (out of the
``left'' side).
When it is necessary to read the contents of the buffer, the {\tt read()}
method returns a pointer to its contents.
\liquid\ implements this shifting method in the same manner as a ring buffer,
and linearizes the data very efficiently, without performing any unnecessary
data memory copies.
Effectively, the window looks like:

\begin{centering}
\includegraphics[width=16cm]{figures.pgf/window}
\end{centering}

Listed below is the full interface for the {\tt window} family of
objects.
While each method is listed for {\tt windowcf}
(a window with {\tt float complex} elements),
the same functionality applies to the {\tt windowf} object.
%
\begin{description}
\item[{\tt windowcf\_create(n)}]
    creates a new window with an internal length of $n$ samples.
\item[{\tt windowcf\_recreate(q,n)}]
    extends an existing window's size, similar to the standard C library's
    {\tt realloc()} to $n$ samples.
    If the size of the new window is larger than the old one, the newest
    values are retained at the beginning of the buffer and the oldest
    values are truncated.
    % see \ref{listing:buffer:window}~line~23
    If the size of the new window is smaller than the old one, the
    oldest values are truncated.
    % see \ref{listing:buffer:window}~line~27
\item[{\tt windowcf\_destroy(q)}]
    destroys the object, freeing all internally-allocated memory.
\item[{\tt windowcf\_clear(q)}]
    clears the contents of the buffer by setting all internal values to zero.
\item[{\tt windowcf\_index(q,i,*v)}]
    retrieves the $i^{th}$ sample in the window, storing the output
    value in $v$.
    This is equivalent to first invoking {\tt read()} and then indexing
    on the resulting pointer;
    however the result is obtained much faster.
    Therefore invoking {\tt windowcf\_index(q,0,*v)} returns the
    {\em oldest} value in the window.
\item[{\tt windowcf\_read(q,**r)}]
    reads the contents of the window by returning a pointer to the
    aligned internal memory array.
    This method guarantees that the elements are linearized.
    This method should {\em only} be used for reading; writing values to
    the buffer has unspecified results.
\item[{\tt windowcf\_push(q,v)}]
    shifts a single sample $v$ into the right side of the window,
    pushing the oldest (left-most) sample out of the end.
    Unlike stacks, the {\tt windowcf} object has no equivalent ``pop''
    method, as values are retained in memory until they are overwritten.
\item[{\tt windowcf\_write(q,*v,n)}]
    writes a block of $n$ samples in the array $\vec{v}$ to the window.
    Effectively, it is equivalent to pushing each sample one at a time,
    but executes much faster.
\end{description}

Here is an example demonstrating the basic functionality of the window object.
The comments show the internal state of the window after each function call as
if the window were a simple C array.
%
\input{listings/window.example.c.tex}


\subsection{{\tt wdelay} delay buffer}
\label{module:buffer:wdelay}
The {\tt wdelay} object in \liquid\ implements a an efficient digital
delay line with a minimal amount of memory.
Specifically, the transfer function is just
%
\begin{equation}
\label{eqn:buffer:wdelay}
    H_d(z) = z^{-k}
\end{equation}
%
where $k$ is the number of samples of delay.
%
The interface for the {\tt wdelay} family of objects is listed below.
While the interface is given for {\tt wdelayf} for floating-point
precision, equivalent interfaces exist for
{\tt float complex} with {\tt wdelaycf}.
%
\begin{description}
\item[{\tt wdelayf\_create(k)}]
    creates a new {\tt wdelayf} object with a delay of $k$ samples.
\item[{\tt wdelayf\_recreate(q,k)}]
    adjusts the delay size, preserving the internal state of the object.
\item[{\tt wdelayf\_destroy(q)}]
    destroys the object, freeing all internally-allocated memory.
\item[{\tt wdelayf\_print(q)}]
    prints the object's properties internal state to the standard
    output.
\item[{\tt wdelayf\_clear(q)}]
    clears the contents of the internal buffer by setting all values to
    zero.
\item[{\tt wdelayf\_read(q,y)}]
    reads the sample at the head of the buffer and stores it to the
    output pointer.
\item[{\tt wdelayf\_push(q,x)}]
    pushes a sample into the buffer.
\end{description}

%internal buffering, circular/static, ports (threaded)

\section{channel}
communications channel modeling: additive noise, multipath fading...
\section{dotprod: dot products (real, complex)}
\section{equalization}
adaptive equalizers: LMS, RLS, blind...
\section{estimation}
parametric estimation: frequency offset, timing offset, fading...
\section{fec: forward error correction}
(basic), checksum, crc, Hamming block codes...
\section{fft: fast Fourier transform}
(basic), simple implementation

% 
% MODULE : filter
%

\newpage
\section{filter}
\label{module:filter}
The filter module is at the core of \liquid's digital signal processing
functionality.
Filter design and implementation is a significant portion of radio
engineering, and consumes a considerable portion of the baseband receiver's
energy.
For linear digital modulation techniques (see section~\ref{module:modem}),
the matched filter is responsible for limiting the occupied bandwidth of the
transmitted signal while eliminating inter-symbol interference.


% 
% autocorr
%
\subsection{{\tt autocorr} (auto-correlator)}
The {\tt autocorr} family of objects implement autocorrelation of
signals.
The discrete autocorrelation of a signal $\vec{x}$
is a delay, conjugate multiply, and accumulate operation defined as
%
% TODO check this equation
\begin{equation}
\label{eqn:filter:autocorr}
    r_{xx}(n) = \sum_{k=0}^{N-1} {x(n-k)x^*(n-k-d)}
\end{equation}
%
where $N$ is the window length, and $d$ is the overlap delay.
An example of the {\tt autocorr} interface is listed below.
%
\input{listings/autocorr.example.c.tex}
%
A more detailed example is given in
{\tt examples/autocorr\_cccf\_example.c}
in the main \liquid\ project directory.


% 
% decim
%
\subsection{{\tt decim} (decimator)}
The {\tt decim} object family implements a basic interpolator with an integer
output-to-input resampling ratio.
It is essentially just a {\tt firfilt} object which operates on a block of
samples at a time.
An example of the decimator can be seen in figure~\ref{fig:module:filter:decim_crcf}.

%The same result can be accomplished with a {\tt firfilt} object...

\begin{figure}
\centering
  \includegraphics[trim = 0mm 0mm 0mm 0mm, clip, width=13cm]{figures.gen/filter_decim_crcf}
\caption{{\tt decim\_crcf} (decimator) demonstration, $D=4$}
\label{fig:module:filter:decim_crcf}
\end{figure}



% firfilt
\subsection{{\tt firfilt} (finite impulse response filter)}
\label{module:filter:firfilt}
Finite impulse response (FIR) filter.
The output $y$ is the convolution of the input $x$ with the filter
coefficients (impulse response) $h$, viz
\[
    y(n) = \sum_{k=0}^{N-1}{ h(k) x(N-k-1) }
\]
where $\vec{h} = \{h_0,h_1,\ldots,h_{N-1}\}$ is the filter impulse response.

% firhilb
\subsection{{\tt firhilb} (finite impulse response Hilbert transform)}
The {\tt firhilb} object is a finite impulse response Hilbert transform which
converts between real and complex time series.
The interpolator takes a complex time series and produces real-valued samples
at twice the sample rate.
The decimator reverses the process by halving the sample rate of a real-valued
time series to a complex-valued one.

Typical trade-offs between filter length, side-lobe suppression, and
transition bandwidth apply.
The {\tt firhilb} object uses a half-band filter to implement the transform as
efficiently as possible.
While any filter length can be accepted, the {\tt firhilb} object internally
forces the length to be of the form $n=4m+1$.
A halfband filter of this length has $2m$ zeros and $2m+1$ non-zero
coefficients.
Of these non-zero coefficients, the center is exactly $1$ while the other $2m$
are even symmetric, and therefore only $m$ computations are needed...
[TODO: flesh out this section]

{\it See also:} {\tt resamp2} (section~\ref{module:filter:resamp2}),
FIR filter design (section~\ref{module:filter:firdes}).

\input{listings/firhilb.example.c.tex}

\begin{figure}
\centering
\subfigure[time] {
  \includegraphics[trim = 0mm 0mm 0mm 0mm, clip, width=13cm]{figures.gen/filter_firhilb_decim_crcf_time}
}
\subfigure[PSD] {
  \includegraphics[trim = 0mm 0mm 0mm 0mm, clip, width=13cm]{figures.gen/filter_firhilb_decim_crcf_freq}
}
\caption{{\tt firhilb\_crcf} (Hilbert transform) decimator demonstration. The
small signal at $f=0.13$ is due to aliasing as a result of imperfect image
rejection}
\label{fig:module:filter:firhilb_crcf}
\end{figure}

% firpfb
\subsection{{\tt firpfb} (finite impulse response polyphase filter bank)}

% interp
\subsection{{\tt interp} (interpolator)}
The {\tt interp} object implements a basic interpolator with an integer
output-to-input resampling ratio.

\input{listings/interp.example.c.tex}

\begin{figure}
\centering
  \includegraphics[trim = 0mm 0mm 0mm 0mm, clip, width=13cm]{figures.gen/filter_interp_crcf}
\caption{{\tt interp\_crcf} (interpolator) demonstration, $k=4$}
\label{fig:module:filter:interp_crcf}
\end{figure}


% iirfilt
\subsection{{\tt iirfilt} (infinite impulse response filter)}
\label{module:filter:iirfilt}
The {\tt iirfilt\_crcf} object and family implement the infinite impulse
response (IIR) filters.
Also known as recursive filters, IIR filters allow a portion of the output to
be fed back into the input, thus creating an impulse response which is
non-zero for an infinite amount of time.
Formally, the output signal $y[n]$ may be written in terms of the input signal
$x[n]$ as
\[
    y[n] = \frac{1}{a_0} \left(
           \sum_{j=0}^{n_b-1}{ b_j x[n-j] } -
           \sum_{k=1}^{n_a-1}{ a_k y[n-k] }
           \right)
\]
where $\vec{b} = [b_0,b_1,\ldots,b_{n_b-1}]^T$ are the feed-forward parameters
and   $\vec{a} = [a_0,a_1,\ldots,a_{n_a-1}]^T$ are the feed-back parameters
of length $n_b$ and $n_a$, respectively.
The $z$-transform of the transfer function is therefore
\[
    H(z) = \frac{Y(z)}{X(z)}
         = \frac{\sum\limits_{j=0}^{n_b-1}{b_j z^{-j}}}
                {\sum\limits_{k=0}^{n_a-1}{a_k z^{-k}}}
         = \frac{ b_0 + b_1 z^{-1} + \cdots + b_{n_b-1} z^{n_b-1}}
                { a_0 + a_1 z^{-1} + \cdots + a_{n_a-1} z^{n_a-1}}
\]
Typically, $H(z)$ is normalized such that $a_0=1$.

For larger order filters (even as small as $n\approx 8$) the filter can become
unstable due to finite machine precision.
It is often therefore useful to express $H(z)$ in terms of second-order
sections.
For a filter of order $n$, these sections are denoted by the two
$(L+r)\times 3$ matrices $\vec{B}$ and $\vec{A}$
where $r=n \mod 2$ (0 for odd $n$, 1 for even $n$) and $L=(n-r)/2$.
\[
    H_d(z) = 
             \left[
                \frac{B_{r,0} + B_{r,1}z^{-1}}
                     {1       + A_{r,1}z^{-1}}
             \right]^r
             \prod_{k=1}^{L} {\left[
                \frac{B_{k,0} + B_{k,1}z^{-1} + B_{k,2}z^{-2}}
                     {1       + A_{k,1}z^{-1} + A_{k,2}z^{-2}}
             \right]}
\]
Notice that $H(z)$ is now a series of cascaded second-order IIR filters.
The `sos' form is practical when filters are designed from analog prototypes
where the poles and zeros are known.
\liquid\ implements second-order sections efficiently with the internal
{\tt iirfiltsos\_xxxt} object.
For a cascaded second-order section IIR filter, use
{\tt iirfilt\_xxxt\_create\_sos(B,A,n)}.

See also: {\tt iirdes} (IIR filter design).

\subsubsection{interface}
\label{module:filter:iirfilt:interface}
The interface to the {\tt iirfilt} object follows the convention of other
\liquid\ signal processing objects in that the ``{\tt xxxt}'' extension in
this section is only a placeholder for ``{\tt crcf}'' etc.

\begin{description}
\item[{\tt iirfilt\_xxxt\_create()}]
    creates a new {\tt iirfilt} object
\item[{\tt iirfilt\_xxxt\_create\_sos()}]
    creates a new {\tt iirfilt} object using second-order sections.
\item[{\tt iirfilt\_xxxt\_destroy()}]
    destroys an {\tt iirfilt} object, freeing all internally-allocated memory
    arrays and buffers.
\item[{\tt iirfilt\_xxxt\_print()}]
    prints the internals of an {\tt iirfilt} object.
\item[{\tt iirfilt\_xxxt\_clear()}]
    clears the filter's internal state.
\item[{\tt iirfilt\_xxxt\_execute()}]
    accepts one input sample and computes one output sample (i.e. executes one
    iteration of the filter).
\item[{\tt iirfilt\_xxxt\_get\_length()}]
    returns the order of the filter.
\end{description}

Listed below is a basic example of the interface.
For more detailed and extensive examples, refer to the {\tt examples/}
source directory.
\input{listings/iirfilt.example.c.tex}

\begin{figure}
\centering
  \includegraphics[trim = 0mm 0mm 0mm 0mm, clip, width=13cm]{figures.gen/filter_iirfilt_crcf_time}
\caption{{\tt iirfilt\_crcf} (infinite impulse response filter) demonstration}
\label{fig:module:filter:iirfilt_crcf}
\end{figure}


\subsection{{\tt iirdes} (infinite impulse response filter design)}
\label{module:filter:iirdes}
\liquid\ implements infinite impulse respone (IIR) filter design for the four
major classes of filters (Butterworth, Chebyshev, elliptic, and Bessel) by
first computing their analog low-pass prototypes, performing a bilinear
$z$-transform to convert to the digital domain, then transforming to the
appropriate band type (e.g. high pass) if necessary.
Externally, the user may abstract the entire process by using the
{\tt iirdes()} method.

\subsubsection{{\tt iirdes()}, the simplified method}
\label{module:filter:iirdes:iirdes}
The {\tt iirdes()} method implements...

\begin{verbatim}
iirdes(_ftype, _btype, _format, _n, _fc, _f0, _Ap, _As, _B, _A);
\end{verbatim}

\begin{itemize}
\item[{\tt \_ftype}]
    is the analog filter prototype, e.g. {\tt LIQUID\_IIRDES\_BUTTER}
\item[{\tt \_btype}]
    is the band type, e.g. {\tt LIQUID\_IIRDES\_BANDPASS}
\item[{\tt \_format}]
    is the output format of the coefficients, e.g. {\tt LIQUID\_IIRDES\_SOS}
\item[{\tt \_n}]
    is the filter order
\item[{\tt \_fc}]
    is the normalized cutoff frequency of the analog prototype
\item[{\tt \_f0}]
    is the normalized center frequency of the analog prototype (only
    applicable to bandpass and bandstop filter designs, ignored for lowpass
    and highpass filter designs)
\item[{\tt \_Ap}]
    is the passband ripple (only applicable to Chebyshev Type-I and elliptic
    filter designs, ignored for Butterworth, Chebyshev Type-II, and Bessel
    designs)
\item[{\tt \_As}]
    is the stopband ripple (only applicable to Chebyshev Type-II and elliptic
    filter designs, ignored for Butterworth, Chebyshev Type-I, and Bessel
    designs)
\item[{\tt \_B}, {\tt \_A}]
    are the output feed-forward (numerator) and feed-back (denominator)
    coefficients, respectively.
    The format and size of these arrays depends on the value of the
    {\tt \_format} and {\tt \_btype} parameters.
\end{itemize}

\subsubsection{internal description}
\label{module:filter:iirdes:internal}
The internal IIR filter design process is described here:
\begin{enumerate}
\item Use {\tt butterf()}, {\tt cheby1f()}, {\tt cheby2f()}, {\tt ellipf()},
      {\tt besself()} to design a low-pass analog prototype in terms of
      complex zeros, poles, and gain.
      The {\tt azpkf} extension stands for ``analog zeros, poles, gain
      (floating-point).''

    \begin{description}
    \item[{\tt butter\_azpkf()}] Butterworth (maximally flat in the passband)
    \item[{\tt cheby1\_azpkf()}] Chebyshev Type-I (equiripple in the passband)
    \item[{\tt cheby2\_azpkf()}] Chebyshev Type-II (equiripple in the stopband)
    \item[{\tt ellip\_azpkf() }] elliptic filter (equiripple in the pass- and
        stopbands)
    \item[{\tt bessel\_azpkf()}] Bessel (maximally flat group delay)
    \end{description}

\item Compute frequency pre-warping factor, $m$, to set cutoff frequency (and
      center frequency if designing a band-pass or band-stop filter) using the
      {\tt iirdes\_freqprewarp()} method.

\item Convert the low-pass analog prototype to its digital equivalent (also in
      terms of zeros, poles, and gain) using the bilinear $z$-transform using
      the {\tt bilinear\_zpkf()} method.

\item Transform the low-pass digital prototype to high-pass, band-pass, or
      band-stop using the {\tt iirdes\_dzpk\_lp2bp()} method.
      For the band-pass and band-stop cases, the number of poles and zeros
      will need to be doubled.
    \begin{description}
    \item[LP] low-pass filter   : $s = m (1+z^{-1}) / (1-z^{-1})$
    \item[HP] high-pass filter  : $s = m (1-z^{-1}) / (1+z^{-1})$
    \item[BP] band-pass filter  : $s = m (1-c_0 z^{-1}+z^{-2}) / (1-z^{-2})$
    \item[BS] band-stop filter  : $s = m (1-z^{-2}) / (1-c_0 z^{-1}+z^{-2})$
    \end{description}

\item Transform the digital $z/p/k$ form of the filter to one of the two forms:
    \begin{itemize}
    \item[tf]  typical transfer function for digital iir filters of the form
        $B(z)/A(z)$, {\tt iirdes\_dzpk2tff()}
    \item[sos] second-order sections form : $\prod_k{ B_k(z)/A_k(z) }$, 
        {\tt iirdes\_dzpk2sosf()}.
        This is the preferred method.
    \end{itemize}

\item Create the filter object (e.g. iirfilt) from the resulting
      structure using either {\tt iirfilt\_crcf\_create()} or
      {\tt iirfilt\_crcf\_create\_sosf()}

\end{enumerate}

%An extensive example is given in {\tt sandbox/iirdes\_example.c} while
%{\tt examples/iirdes\_example.c} gives the simplified interface.
A simplified example is given in {\tt examples/iirdes\_example.c}.

\subsubsection{{\tt iirdes} filter types}
\label{module:filter:iirdes:types}
There are currently five low-pass prototypes available for inifinite impulse
response filter design in \liquid,
\begin{description}
\item[{\tt LIQUID\_IIRDES\_BUTTER}]
    is a Butterworth filter.
    This is an all-pole analog design that has a maximally flat magnitude
    response in the passband.
    The analog prototype interface is {\tt butter\_azpkf()} which computes the
    roots $s_0,s_1,\ldots,s_{n-1}$ of the $n^{th}$-order Butterworth
    polynomial,
    \[
        s_k = \omega_c \exp\left\{
                    j \frac{\left(2k+n+1\right)\pi}{2n}
              \right\}
    \]
    for $k=0,1,\ldots,n-1$.
    Note that this results in a set of complex conjugate pairs such that
    $(-1)^n s_0 s_1 \cdots s_{n-1} = 1$.
    An example of a digital filter response can be found in
    figure~\ref{fig:module:filter:butter}
\item[{\tt LIQUID\_IIRDES\_CHEBY1}]
    is a Chebyshev Type-I filter.
    This design uses Chebyshev polynomials to create a filter with a sharper
    transition band than the Butterworth design by allowing ripples in the
    passband.
    The analog prototype interface is {\tt cheby1\_azpkf()}.
    An example of a digital filter response can be found in
    figure~\ref{fig:module:filter:cheby1}
\item[{\tt LIQUID\_IIRDES\_CHEBY2}]
    is a Chebyshev Type-II filter.
    This design is similar to that of Chebyshev Type-I, except that the
    Chebyshev polynomial is inverted.
    This inverts the magnitude response of the filter and exhibits an
    equiripple behavior in the stopband, rather than the passband.
    The analog prototype interface is {\tt cheby2\_azpkf()}.
    An example of a digital filter response can be found in
    figure~\ref{fig:module:filter:cheby2}
\item[{\tt LIQUID\_IIRDES\_ELLIP}]
    is an elliptic (Cauer) filter.
    This design allows ripples in both the passband and stopbands to create a
    filter with a very sharp transition band.
    The design process is somewhat more involved than the Butterworth and
    Chebyshev prototypes and requires solving the elliptic integral of
    different moduli.
    For a more detailed description we refer the interested reader to
    \cite{Orfanidis:2006}.
    The analog prototype interface is {\tt ellip\_azpkf()}.
    An example of a digital filter response can be found in
    figure~\ref{fig:module:filter:ellip}
\item[{\tt LIQUID\_IIRDES\_BESSEL}]
    is a Bessel filter.
    This is an all-pole analog design that has a maximally flat group delay
    response (maximally linear phase response).
    The solution to the design happens to be the roots to the Bessel
    polynomials of equal order.
    Computing the roots to the polynomial is, again, somewhat complex.
    For a more detailed description we refer the interested reader to
    \cite{Orchard:1965}.
    The analog prototype interface is {\tt bessel\_azpkf()}.
    An example of a digital filter response can be found in
    figure~\ref{fig:module:filter:bessel}
\end{description}

% 
% bilinear z-transform
%
\subsubsection{{\tt bilinear\_zpkf} (Bilinear $z$-transform)}
\label{module:filter:iirdes:bilinear}
The bilinear $z$-transform converts an analog prototype to its digital
counterpart.
Given a continuous time analog transfer function in zeros/poles/gain form
(``zpk''),
%\[
%    H(s) =  \frac{
%                r_0 + r_1 s + r_2 s^2 + \cdots + r_n s^n
%            }{
%                q_0 + q_1 s + q_2 s^2 + \cdots + q_m s^m
%            }
%\]
\[
    H_a(s) = k_a
            \frac{
                (s-z_{a0})(s-z_{a1})\cdots(s-z_{an-1})
            }{
                (s-p_{a0})(s-p_{a1})\cdots(s-p_{am-1})
            }
\]
the bilinear $z$-transform converts $H_a(s)$ into the discretized transfer
function $H_d(z)$ by mapping the $s$-plane onto the $z$-plane with the
approximation
\[
    s \approx \frac{2}{T}
              \frac{1-z^{-1}}{1 + z^{-1}}
\]
This maps $H_a(0) \rightarrow H_d(0)$ and
$H_a(\infty) \rightarrow H_d(\omega_s/2)$, however we are free to choose the
pre-warping factor which maps the cutoff frequency $\omega_c$.
\[
    s \rightarrow \omega_c
                  \cot\left(\frac{\pi \omega_c}{\omega_s}\right)
                  \frac{1-z^{-1}}{1+z^{-1}}
\]
Substituting this into $H_a(s)$ gives the discrete-time transfer function
\[
    H(z) = k_a \frac{
            \left(m\frac{1-z^{-1}}{1+z^{-1}}-z_{a0}\right)
            \left(m\frac{1-z^{-1}}{1+z^{-1}}-z_{a1}\right)
            \cdots
            \left(m\frac{1-z^{-1}}{1+z^{-1}}-z_{an-1}\right)
           }{
            \left(m\frac{1-z^{-1}}{1+z^{-1}}-p_{a0}\right)
            \left(m\frac{1-z^{-1}}{1+z^{-1}}-p_{a1}\right)
            \cdots
            \left(m\frac{1-z^{-1}}{1+z^{-1}}-p_{am-1}\right)
           }
\]
where $m=\omega_c \cot\left(\pi \omega_c / \omega_s\right)$ is the
frequency pre-warping factor, computed in \liquid\ via the method
{\tt iirdes\_freqprewarp()}.
Multiplying both the numerator an denominator by $(1+z^{-1})^{m-1}$...
...transforming the the analog zeros ($z_k$), poles ($p_k$), and gain ($k$) into
their digital equivalents. [reword]
...maps the analog zeros and poles $\{z_{ai},p_{ai}\}$
to digital $\{z_{i},p_{i}\}$...
\[
    H_d(s) = k_d
            \frac{
                (1-z_{0}z^{-1})(1-z_{1}z^{-1})\cdots(1-z_{n-1}z^{-1})
            }{
                (1-p_{0}z^{-1})(1-p_{1}z^{-1})\cdots(1-p_{n-1}z^{-1})
            }
\]
%
\liquid\ computes all of this with the {\tt bilinear\_zpk()} method.

% 
% filter transformations
%
\subsubsection{Filter transformations}
\label{module:filter:iirdes:transformations}
\begin{description}
\item[{\tt iirdes\_dzpk\_lp2hp()}]
    Converts a low-pass prototype to a high-pass prototype.
\item[{\tt iirdes\_dzpk\_lp2bp()}]
    Converts a low-pass prototype to a band-pass prototype.
\end{description}

...second-order sections form:
\[
    H_d(z) = H_0
             \left[
                G_0 \frac{1 + z^{-1}}
                         {1 - p_0 z^{-1}}
             \right]^r
             \prod_{k=1}^{L} {\left[
                G_i \frac{(1-z_iz^{-1})(1-z_i^*z^{-1})}
                         {(1-p_iz^{-1})(1-p_i^*z^{-1})}
             \right]}
\]
where $r=0$ when the filter order is odd, $r=1$ when the filter order is even,
and $L=(n-r)/2$.

% Butterworth IIR filter design
\begin{figure}
\centering
\subfigure[spectrum] {
  \includegraphics[trim = 2mm 5mm 0mm 5mm, clip, width=0.55\textwidth]{figures.gen/filter_butter_psd}
}
\subfigure[zeros, poles] {
  \includegraphics[trim = 15mm 0mm 15mm 0mm, clip, width=0.37\textwidth]{figures.gen/filter_butter_zpk}
}
\caption{{\tt butterf} (Butterworth filter design)}
\label{fig:module:filter:butter}
\end{figure}

% Chebyshev type-I IIR filter design
\begin{figure}
\centering
\subfigure[spectrum] {
  \includegraphics[trim = 2mm 5mm 0mm 5mm, clip, width=0.55\textwidth]{figures.gen/filter_cheby1_psd}
}
\subfigure[zeros, poles] {
  \includegraphics[trim = 15mm 0mm 15mm 0mm, clip, width=0.37\textwidth]{figures.gen/filter_cheby1_zpk}
}
\caption{{\tt cheby1f} (Chebyshev type-I filter design)}
\label{fig:module:filter:cheby1}
\end{figure}

% Chebyshev type-II IIR filter design
\begin{figure}
\centering
\subfigure[spectrum] {
  \includegraphics[trim = 2mm 5mm 0mm 5mm, clip, width=0.55\textwidth]{figures.gen/filter_cheby2_psd}
}
\subfigure[zeros, poles] {
  \includegraphics[trim = 15mm 0mm 15mm 0mm, clip, width=0.37\textwidth]{figures.gen/filter_cheby2_zpk}
}
\caption{{\tt cheby2f} (Chebyshev type-II filter design)}
\label{fig:module:filter:cheby2}
\end{figure}

% Elliptic IIR filter design
\begin{figure}
\centering
\subfigure[spectrum] {
  \includegraphics[trim = 2mm 5mm 0mm 5mm, clip, width=0.55\textwidth]{figures.gen/filter_ellip_psd}
}
\subfigure[zeros, poles] {
  \includegraphics[trim = 15mm 0mm 15mm 0mm, clip, width=0.37\textwidth]{figures.gen/filter_ellip_zpk}
}
\caption{{\tt ellipf} (Elliptic filter design)}
\label{fig:module:filter:ellip}
\end{figure}

% Bessel IIR filter design
\begin{figure}
\centering
\subfigure[spectrum] {
  \includegraphics[trim = 2mm 5mm 0mm 5mm, clip, width=0.55\textwidth]{figures.gen/filter_bessel_psd}
}
\subfigure[zeros, poles] {
  \includegraphics[trim = 15mm 0mm 15mm 0mm, clip, width=0.37\textwidth]{figures.gen/filter_bessel_zpk}
}
\caption{{\tt besself} (Bessel filter design)}
\label{fig:module:filter:bessel}
\end{figure}


\subsection{{\tt resamp2} (halfband resampler)}
\label{module:filter:resamp2}
{\tt resamp2} is a half-band resampler used for efficient interpolation and
decimation.
The internal filter of the {\tt resamp2} object is a Kaiser-windowed $\sinc$
(see {\tt firdes\_kaiser\_window}, section~\ref{module:filter:firdes}) with
$f_c = 1/2$.
This makes the filter half-band, and puts the half-power (6 dB) cutoff point
$\omega_c$ at $\pi/2$ (one quarter of the sampling frequency).
In fact, any FIR filter design using a windowed $\sinc$ function with
periodicity $f_c=1/2$ will generate a Nyquist half-band filter (zero
inter-symbol interference).
This is because \cite[(4.6.3)]{Vaidyanathan:1993}
\[
    h(Mn) = 
        \begin{cases}
        c & n=0 \\
        0 & \text{otherwise}
        \end{cases}
\]
which holds for $h(n) = w(n) \sin(\pi n/M) / (\pi n)$ since
$\sin(\pi n/M) = 0$ for $n=$ any non-zero multiple of M.
Additionally, $M=2$ is the special case of half-band filters.
In particular half-band filtering is computationally efficient because half
the coefficients of the filter are zero, and the remaining half are symmetric
(so long as $w(n)$ is also symmetric).
In theory, this means that for a filter length of $4m+1$ taps, only $m$
computations are necessary \cite{harris:2004}.
The {\tt resamp2} object in \liquid\ uses a Kaser window for $w(n)$ for
several reasons, but in particular because it is nearly optimum, and it is
easy to trade side-lobe attenuation for transition bandwidth.

\subsection{{\tt resamp} (arbitrary resampler)}
For arbitrary (e.g. irrational) resampling ratios, the {\tt resamp} object is
the ideal solution.
It makes no restrictions on the output-to-input resampling ratio (e.g.
irrational values are fair game).
The arbitrary resampler uses a polyphase filter bank for interpolation between
available input sample points.

Because the number of outputs for each input is not fixed, the interface needs
some explaining.
Over time the true resampling ratio will equal the value specified, however
from one input to the next, the number of outputs will change.
For example, if the resampling rate is $2$, every input will produce exactly
two output samples.
However, if the resampling rate is $\sqrt{2} \approx 1.4142$, an input sample
will usually produce one output, but sometimes two.
In the limit (on {\it average}) however, the ratio of output samples to input
samples will be exactly $\sqrt{2}$.
The {\tt resamp} object handles this internally by storing the accumulated
sampling phase and produces an output for each overflow (i.e. values where the
accumulated phase is equal to or exceeds 1).

Below is a code example demonstrating the {\tt resamp} interface.
Notice that the {\tt resamp\_crcf\_execute()} method also returns the number
of samples written to the buffer.
This number will never exceed $\lceil r \rceil$.

\input{listings/resamp_crcf.example.c.tex}

Figure~\ref{fig:module:filter:resamp_crcf} gives a graphical depiction of the
arbitrary resampler, in both the time and frequency domains.
The time series has been aligned (shifted by the filter delay and scaled by
the resampling rate) to show equivalence.
Additionally, the signal's power spectrum has been scaled by $r$ to reflect
the change in sampling rate.
%Notice how the input and output spectra align, despite its

It is important to understand how filter design impacts the performance of the
resampler.
The {\tt resamp} object interpolates between available sample points to
minimize aliasing effects on the output signal.
This is apparent in the power spectral density plot in
figure~\ref{fig:module:filter:resamp_crcf} which shows very little aliasing on
the output signal.
Aliasing can be reduced by increasing the filter length at the cost of
additional computational complexity.
For synchronization of digital receivers, it is always good practice to
preceed the resampler with an anti-aliasing filter to remove out-of-band
interference.

See also: {\tt resamp2}, {\tt firpfb}, {\tt symsync},
{\tt examples/resamp\_crcf\_example.c}

\begin{figure}
\centering
\subfigure[time] {
  \includegraphics[trim = 0mm 0mm 0mm 0mm, clip, width=13cm]{figures.gen/filter_resamp_crcf}
}
\subfigure[PSD] {
  \includegraphics[trim = 0mm 0mm 0mm 0mm, clip, width=13cm]{figures.gen/filter_resamp_crcf_psd}
}
\caption{{\tt resamp\_crcf} (arbitrary resampler) demonstration,
$r=1/\sqrt{2}\approx0.7071$}
\label{fig:module:filter:resamp_crcf}
\end{figure}

\subsection{{\tt symsync} (symbol synchronizer)}
The {\tt symsync} object is a multi-rate symbol timing synchronizer useful for
locking a received digital signal to the receiver's clock.
It is effectively the same as the {\tt resamp} object, but includes an
internal control mechanism for tracking to timing phase and frequency offsets.
The filter structure is a polyphase representation of a Nyquist matched
filter.

\subsubsection{Theory}

\subsubsection{Loop Architecture}
Timing error:
\[
    e_\tau(nT_s) = \tanh\left( y(nT_s)\dot{y}(nT_s) \right)
\]
Filtered timing error:
\[
    \hat{e}_\tau(n) = \beta e_\tau(n) + \alpha \hat{e}_\tau(n-1)
\]
For a given bandwidth $b$, the loop filter coefficients are computed as
$\alpha = 1-b$ and $\beta = 0.22b$.
While certainly not optimized, it is important to understand the difficulty in
computing loop filter coefficients when a delay is introduced into the control
loop.
This delay is a by-product of the matched filter itself, in the sense that the
timing error estimate $e_\tau(n)$ is computed on $k m T_s$...

\subsection{{\tt firfarrow} (finite impulse response Farrow filter)}
\begin{description}
\item[{\tt firfarrow\_crcf\_create()}]
    creates a {\tt firfarrow} object.
\item[{\tt firfarrow\_crcf\_destroy()}]
    destroy object, freeing all internally-allocated memory.
\item[{\tt firfarrow\_crcf\_clear()}]
    clear filter internal memory buffer.
    This does not reset the delay.
\item[{\tt firfarrow\_crcf\_print()}]
    prints the filter's internal state to {\tt stdout}.
\item[{\tt firfarrow\_crcf\_push()}]
    push a single sample into the filter's internal buffer.
\item[{\tt firfarrow\_crcf\_set\_delay()}]
    set fractional delay of filter.
\item[{\tt firfarrow\_crcf\_execute()}]
    computes sample output.
\item[{\tt firfarrow\_crcf\_get\_length()}]
    returns length of the filter (number of taps)
\item[{\tt firfarrow\_crcf\_get\_coefficients()}]
    returns the internal filter coefficients.
\end{description}

\subsection{{\tt firdes} (finite impulse response filter design)}
\label{module:filter:firdes}

\subsubsection{Window prototype}
Basic idea: apply windowing function to truncated sinc

\begin{equation}
h(t) = w_n(t) \sinc(\Delta f t)
\end{equation}

where $w_n(t)$ is chosen from...

The Kaiser window is defined as...
\begin{equation}
\label{eqn:kaiser_window}
    w[n] = \frac{
        I_0\left(\pi\beta\sqrt{1-\left(\frac{n}{N/2}^2\right)}\right)
    }{
        I_0\left(\pi\beta\right)
    }
\end{equation}
where $I_\nu(z)$ is the modified Bessel function of the first kind of order
$\nu$.
\begin{equation}
\label{eqn:besseli_infinite_sum}
    I_\nu(z) = \sum_{k=0}^{\infty}{
        \frac{
            \left(\frac{1}{4}z^2\right)^k
        }{
            k!\Gamma(k+\nu+1)
        }
    }
\end{equation}
For more approximations to $I_0(z)$, see section~\ref{module:math} on the math
module.

The sum (\ref{eqn:besseli_infinite_sum}) converges quickly due to the
denominator increasing rapidly, thus only a few terms are necessary for
sufficient approximation.
An approximation for the value of $\beta$ to give a particular sidelobe level
for the window is given by {\tt kaiser\_beta\_As()}, and
the length of the filter can be approximated with
{\tt estimate\_req\_filter\_len()} (see section~\ref{module:filter:misc} for
more detail on these methods).

TODO : add discussion of interface for {\tt firdes\_kaiser\_window}.

\begin{figure}
\centering
\subfigure[time] {
  \includegraphics[trim = 0mm 0mm 0mm 0mm, clip, width=13cm]{figures.gen/filter_kaiser_time}
}
\subfigure[PSD] {
  \includegraphics[trim = 0mm 0mm 0mm 0mm, clip, width=13cm]{figures.gen/filter_kaiser_freq}
}
\caption{{\tt fir\_kaiser\_window()} demonstration, $f_c=0.15$,
$\Delta f=0.05$, $A_s=60$dB}
\label{fig:module:filter:firdes_kaiser}
\end{figure}


\subsubsection{{\tt rcos} (raised-cosine filter)}
\label{module:filter:firdes:rcos}

\subsubsection{{\tt rrcos} (square-root raised-cosine filter)}
\label{module:filter:firdes:rrcos}
{\tt design\_rrc\_filter} calculates the coefficients for a square-root
raised-cosine (RRC) finite impulse response (FIR) filter commonly used in
digital communications.
The input parameters are as follows
\begin{itemize}
    \item[$k$] : samples per symbol
    \item[$m$] : sample delay
    \item[$\beta$] : excess bandwidth (rolloff) factor
\end{itemize}
The function returns the filter coefficients in the buffer, $h$, which has been
allocated externally.  The length of the filter is always
\[
    h_{len} = 2 k m + 1
\]

The filter coefficients themselves are derived from the following equation
\[ 
    h\left[z\right] =
      4\beta \frac{ \cos\left[(1+\beta)\pi z\right] +
                    \sin\left[(1-\beta)\pi z\right] / (4\beta z) }
                  { \pi \sqrt{T}\left[ 1-16\beta^2z^2\right] }
\]
where $z=n/k-m$, and $T=1$ for most cases.
The function compensates for the two cases where $h[n]$ might be
undefined in the above equation, viz
\[
    \mathop {\lim }\limits_{z \to 0 } h(z) = 1 - \beta + 4\beta/\pi
\]
and
\[
    \mathop {\lim }\limits_{z \to \pm \frac{1}{4\beta} } h(z) =
        \frac{\beta}{\sqrt{2}}
        \left[
            \left(1 + \frac{2}{\pi}\right)\sin\left(\frac{\pi}{4\beta}\right) +
            \left(1 - \frac{2}{\pi}\right)\cos\left(\frac{\pi}{4\beta}\right)
        \right]
\]


\subsubsection{{\tt firdespm}}
\label{module:filter:firdespm}
FIR filter desing using the Parks-McClellan algorithm.

\input{listings/firdespm.example.c.tex}


% firdespm
\begin{figure}
\centering
  \includegraphics[trim = 0mm 0mm 0mm 0mm, clip, width=13cm]{figures.gen/filter_firdespm}
\caption{{\tt firdespm} multi-passband filter design demonstration}
\label{fig:module:filter:firdespm}
\end{figure}

\subsubsection{Miscellaneous functions}
\label{module:filter:misc}
Here are several miscellaneous functions used in \liquid's filter module,
useful to filtering and filter design.

\begin{description}
\item[{\tt estimate\_req\_filter\_len()}]
    returns an estimate of the required filter length, given a transition
    bandwidth $\Delta f$ and stopband attenuation $A_s$.
    The estimate uses Hermann's formula \cite{Herrmann:1973}
    \[
        N \approx \frac{
                D_{\infty}(\delta_1,\delta_2) - f(\delta_1,\delta_2)(\Delta f)^2
            }{
                \Delta f
            }
    \]
    where
    \[
        D_{\infty} (\delta_1,\delta_2) =
            \Bigl(0.005309 t_1^2 + 0.07114 t_1 - 0.4761 \Bigr) t_2
           -\Bigl(0.00266  t_1^2 + 0.5941  t_1 + 0.4278 \Bigr)
    \]
    and
    \[
        f(\delta_1,\delta_2) = 11.012 + 0.51244\left(t_1 - t_2\right)
    \]
    with $t_1 = \log_{10}\delta_1$ and $t_2 = \log_{10}\delta_2$.

\item[{\tt estimate\_req\_filter\_As()}]
    returns an estimate of the filter's stop-band attenuation $A_s$
    given the filter's length $N$ and transition bandwidth $\Delta f$.
    The estimate uses an iterative binary search to find $A_s$ from
    {\tt extimate\_req\_filter\_As()}.

\item[{\tt estimate\_req\_filter\_df()}]
    returns an estimate of the filter's transition bandwidth $\Delta f$
    given the filter's length $N$ and stop-band attenuation $A_s$.
    The estimate uses an iterative binary search to find $\Delta f$ from
    {\tt extimate\_req\_filter\_As()}.

\item[{\tt kaiser\_beta\_As()}]
    returns an estimate of the Kaiser $\beta$ factor for a particular
    stop-band attenuation $A_s$.
    The estimate uses Kaiser's original formula \cite{Vaidyanathan:1993}, viz
    \[
    \beta =
    \begin{cases}
        0.1102 (A_s - 8.7)      &   A_s > 50 \\
        0.5842 (A_s - 21)^{0.4} &   21 < A_s \le 50 \\
        0                       &   \text{else}
    \end{cases}
    \]

\item[{\tt fir\_group\_delay()}]
    computes the group delay for a finite impulse-response filter.
    The group delay $\tau_g$ at frequency $f$
    for a finite impulse response filter of length $N$
    is computed as
    \[
        \tau_g = \Re\Biggl\{ \,\,
            \frac{
                \sum_{k=0}^{N-1}{h(k)e^{j 2 \pi f k} \cdot k}
            } {
                \sum_{k=0}^{N-1}{h(k)e^{j 2 \pi f k}}
            }\,\,
            \Biggr\}
    \]

\item[{\tt iir\_group\_delay()}]
    computes the group delay for an infinite impulse-response filter.
    The group delay $\tau_g$ at frequency $f$ for an infinite impulse
    response filter of order $N$ is computed as
    \[
        \tau_g = \Re\Biggl\{ \,\,
            \frac{
                \sum_{k=0}^{2(N+1)}{c(k)e^{j 2 \pi f k} \cdot k}
            } {
                \sum_{k=0}^{2(N+1)}{c(k)e^{j 2 \pi f k}}
            }\,\,
            \Biggr\}
            - N
    \]
    where $c(n) = \sum_{m=0}^{N-1}{a(m)^*b(m-n)}$
    for $n \in \{0,1,\ldots,2(N+1)\}$
    which can be described as the flipped convolution of $\vec{a}$ and
    $\vec{b}$.

\item[{\tt iirdes\_isstable()}]
    checks the stability of an infinite impulse-response filter by computing
    the roots of the denominator (poles) and ensuring that they lie within the
    unit circle.
    Notice that the poles in
    % using the cleveref package
%    \cref{fig:module:filter:butter,%
%          fig:module:filter:cheby1,%
%          fig:module:filter:cheby2,%
%          fig:module:filter:ellip,%
%          fig:module:filter:bessel}
    % using regular references
    Figure~\ref{fig:module:filter:butter}--%
           \ref{fig:module:filter:bessel}
    all have their poles within the unit circle and are therefore stable (as
    expected).

\item[{\tt liquid\_filter\_autocorr()}]
    computes the auto-correlation of a filter with an array of coefficients
    $\vec{h}$ of length $N$ at a specific lag $n$ as
    \[
        r_{hh}(n) = \sum_{k=n}^{N-1} {h(k)h^*(k-n)}
    \]

\item[{\tt liquid\_filter\_isi()}]
    computes a filter's inter-symbol interference (both mean-squared error and
    maximum error).
    This is useful in comparing the performance of root-Nyquist matched filter
    designs (e.g. root raised-cosine).
    %The inter-symbol interference of a filter is simply its auto-correlation
    %at a lag equal to its...
    %The total interference is the ...

\item[{\tt liquid\_filter\_energy()}]
    computes a finite impulse response filter's relative out-of-band
    energy for a cutoff frequency $f_c$, viz
    \[
        e =
            \frac{
                \int_{2 \pi f_c}^{\infty}{H(\omega)d\omega}
            } {
                \int_{0}^{\infty}{H(\omega)d\omega}
            }
    \]

\end{description}


%fir, iir, polyphase, hilbert, interp, decim, Farrow, design, remez, etc.

\section{interleaver}
bit- and symbol-level
\section{math}
transcendental functions not in the C standard library (gamma, besseli, etc.)
and polynomial operations
\section{matrix}
basic math, lu-decomp, inversion, gauss-jordan elimination
\section{modem}
modulate, demodulate, psk, dpsk, qam, oqam, aqam, msk, fsk

%
% MODULE : multicarrier
%

\newpage
\section{multicarrier}

\subsection{firpfbch}
finite impulse response polyphase filterbank channelizer (firpfbch)
\begin{itemize}
\item uses FFTW library (www.fftw.org) if available, internal FFT library
      otherwise
\item basis behind OFDM/OQAM
\end{itemize}



%OFDM/OQAM, OFDM...

\section{nco}
numerically-controlled oscillator: mix, pll
\section{optim}
newton-raphson, ga, gradient
\section{quantization}
analog/digital converters, companding...
\section{random}
random number generators
\section{sequence}
lfsr, complementary codes
\section{utility}
useful miscellany: bit packing, etc. 


%%%%%%%%%%%%%%%%%%%%%%%%%%%%%%%%%%%%%%%%%%%%%%%%%%%%%%%%%%%%%%%%%%%%%
%
%             BIBLIOGRAPHY
%
%%%%%%%%%%%%%%%%%%%%%%%%%%%%%%%%%%%%%%%%%%%%%%%%%%%%%%%%%%%%%%%%%%%%%
\cleardoublepage
%\bibliography{../IEEEtranBST/IEEEabrv,liquid}


% In LaTeX, each appendix is a "chapter"
\appendix
%%%%%%%%%%%%%%%%%%%%%%%%%%%%%%%%%%%%%%%%%%%%%%%%%%%%%%%%%%%%%%%%%%%%%
%
%             APPENDIX A
%
%%%%%%%%%%%%%%%%%%%%%%%%%%%%%%%%%%%%%%%%%%%%%%%%%%%%%%%%%%%%%%%%%%%%%
\end{document}

