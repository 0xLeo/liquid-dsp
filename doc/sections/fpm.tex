% 
% MODULE : fpm (fixed-point math)
%

\newpage
\section{Fixed-point types}
\label{section:fpm:types}
%
Fixed-point data types are represented simply as signed integers.
This is typically known as ``Q'' format.
In particular \liquid\ implements two data types:
{\tt q16\_t} and {\tt q32\_t}.

% 
% SECTION : BASICS
%
\section{Basics}
\label{section:fpm:basics}

% Q-Format
\subsection{The Q Format}
\label{section:fpm:basics:qformat}

Under the hood,
{\tt q16\_t} is really just the standard {\tt int16\_t}
(the standard 16-bit signed integer type) and
{\tt q32\_t} is really just {\tt int32\_t}, a standard 32-bit signed
integer.

Also defined are {\tt q16\_at} and {\tt q32\_at}: their double-precision
accumulator types.


% Angular representation
\subsection{Angular Representation}
\label{section:fpm:basics:angular}


% Primitive Constants
\subsection{Primitive Constants}
\label{section:fpm:basics:primitives}


{\small
    \begin{tabular*}{0.65\textwidth}{l@{\extracolsep{\fill}}ll}
    \toprule
    {\it name} &
    {\it float value} &
    {\it description}\\\otoprule
    %
    {\tt q16\_min}              &   & minimum representable value \\
    {\tt q16\_max}              &   & maximum representable value \\
    {\tt q16\_zero}             & 0 & zero \\
    {\tt q16\_one}              & 1 & one \\
    {\tt q16\_2pi}              &   & 2*pi (angular) \\
    {\tt q16\_pi}               &   & pi (angular) \\
    {\tt q16\_pi\_by\_2}        &   & pi/2 (angular) \\
    {\tt q16\_pi\_by\_4}        &   & pi/4 (angular) \\
    {\tt q16\_angle\_scalarf}   &   & angular/scalar relationship\\\bottomrule
    \end{tabular*}
} % small


% 
% SECTION : MATH
%
\section{Math}
\label{section:fpm:math}

\subsection{Conversion}
\label{section:fpm:math:conversion}

\subsection{Arithmetic}
\label{section:fpm:math:arithmetic}

\subsection{Trigonometric functions}
\label{section:fpm:math:trig}

\subsection{Hyperbolic functions}
\label{section:fpm:math:hyperbolic}

\subsection{Exponential and Logarithmic functions}
\label{section:fpm:math:explog}

\subsection{Power functions}
\label{section:fpm:math:power}

\subsection{Error and Gamma functions}
\label{section:fpm:math:error-gamma}

\subsection{Mathematical Constants}
\label{section:fpm:math:constants}

Mathematical constants (see: {\tt genlib/genlib.constants.c})

{\small
    \begin{tabular*}{0.65\textwidth}{l@{\extracolsep{\fill}}ll}
    \toprule
    {\it name} &
    {\it float value} &
    {\it description}\\\otoprule
    %
    {\tt q16\_E}            & {\tt 2.71828182845905}    & $e$\\
    {\tt q16\_LOG2E}        & {\tt 1.44269504088896}    & $\log_2(e)$\\
    {\tt q16\_LOG10E}       & {\tt 0.43429448190325}    & $\log_{10}(e)$\\
    {\tt q16\_LN2}          & {\tt 0.69314718055995}    & $\ln(2)$\\
    {\tt q16\_LN10}         & {\tt 2.30258509299405}    & $\ln(10)$\\
    {\tt q16\_PI}           & {\tt 3.14159265358979}    & $\pi$ (literal)\\
    {\tt q16\_PI\_2}        & {\tt 1.57079632679490}    & $\pi/2$\\
    {\tt q16\_PI\_4}        & {\tt 0.78539816339745}    & $\pi/4$\\
    {\tt q16\_1\_PI}        & {\tt 0.31830988618379}    & $1/\pi$\\
    {\tt q16\_2\_PI}        & {\tt 0.63661977236758}    & $2/\pi$\\
    {\tt q16\_2\_SQRTPI}    & {\tt 1.12837916709551}    & $2/\sqrt{\pi}$\\
    {\tt q16\_SQRT2}        & {\tt 1.41421356237310}    & $\sqrt{2}$\\
    {\tt q16\_SQRT1\_2}     & {\tt 0.70710678118655}    & $\sqrt{1/2}$\\\bottomrule
    \end{tabular*}
} % small


% 
% SECTION : COMPLEX MATH
%
\section{Complex Math}
\label{section:fpm:math-complex}

\subsection{Conversion}
\label{section:fpm:math-complex:conversion}

\subsection{Arithmetic}
\label{section:fpm:math-complex:arithmetic}

\subsection{Trigonometric functions}
\label{section:fpm:math-complex:trig}

\subsection{Hyperbolic functions}
\label{section:fpm:math-complex:hyperbolic}

\subsection{Exponential and Logarithmic functions}
\label{section:fpm:math-complex:explog}

\subsection{Power functions}
\label{section:fpm:math-complex:power}


