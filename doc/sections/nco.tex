% 
% MODULE : nco (numerically-controlled oscillator)
%

\section{nco (numerically-controlled oscillator)}
\label{module:nco}
This section describes the numerically-controlled oscillator (NCO) for carrier
synchronization.

\subsection{{nco}}
\label{module:nco:nco}
The {\tt nco} object implements an oscillator with two options for internal
phase precision: {\tt LIQUID\_NCO} and {\tt LIQUID\_VCO}.
The {\tt LIQUID\_NCO} implements a numerically-controlled oscillator that uses
a look-up table to generate a complex sinusoid while
the {\tt LIQUID\_VCO} implements a ``voltage-controlled'' oscillator that uses
the {\tt sinf} and {\tt cosf} standard math functions to generate a complex
sinusoid.

\subsubsection{Description of operation}
The {\tt nco} object maintains its phase and frequency states internally.
Various computations--such as mixing--use the phase state for generating
complex sinusoids.
The phase $\theta$ of the {\tt nco} object is updated using the
{\tt nco\_crcf\_step()} method which increments $\theta$ by $\Delta\theta$, the
frequency.
Both the phase and frequency of the {\tt nco} object can be manipulated using
the appropriate {\tt nco\_crcf\_set} and {\tt nco\_crcf\_adjust} methods.

\subsubsection{Interface}
\begin{description}
\item[{\tt nco\_crcf\_create()}] creates an {\tt nco} object
\item[{\tt nco\_crcf\_destroy()}] destroys an {\tt nco} object
\item[{\tt nco\_crcf\_reset()}] clears in internal state of an {\tt nco} object
\item[{\tt nco\_crcf\_set\_frequency()}] sets the frequency, $\Delta\theta$
\item[{\tt nco\_crcf\_adjust\_frequency()}] increments the frequency
\item[{\tt nco\_crcf\_set\_phase()}] sets the internal {\tt nco} phase, $\theta$
\item[{\tt nco\_crcf\_adjust\_phase()}] increments the internal {\tt nco} phase
\item[{\tt nco\_crcf\_step()}] increments the internal {\tt nco} phase by its
    internal frequency, $\theta \leftarrow \theta + \Delta\theta$
\item[{\tt nco\_crcf\_get\_phase()}] returns the internal phase
\item[{\tt nco\_crcf\_get\_frequency()}] returns the internal frequency (phase step
    size)
\item[{\tt nco\_crcf\_sin()}] returns $\sin(\theta)$
\item[{\tt nco\_crcf\_cos()}] returns $\cos(\theta)$
\item[{\tt nco\_crcf\_sincos()}] returns $\sin(\theta)$, $\cos(\theta)$
\item[{\tt nco\_crcf\_cexpf()}] returns $e^{j\theta}$
\item[{\tt nco\_crcf\_mix\_up()}] rotates an input vector by $e^{j\theta}$
\item[{\tt nco\_crcf\_mix\_down()}] rotates an input vector by $e^{-j\theta}$
\end{description}

\subsection{PLL (phase-locked loop)}
\label{module:nco:pll}
% NOTE : as of [21f7b4] PLL is internal to NCO object
The phase-locked loop object provides a method for synchronizing oscillators
on different platforms.
It uses a second-order integrating loop filter to adjust the frequency of its
{\tt nco} based on an instantaneous phase error input.
As its name implies, a PLL locks the phase of the {\tt nco} object to a
reference signal.
The PLL accepts a phase error and updates the frequency (phase step size) of
the {\tt nco} to track to the phase of the reference.
The reference signal can be another {\tt nco} object, or a signal whose
carrier is modulated with data.
%Frequency modulation (FM), for example, changes its carrier frequency relative
%to an analog audio signal.

% FIGURE : nco_pll_block diagram
\begin{figure}
\centering
  \includegraphics[width=10cm]{figures.pgf/nco_pll_diagram}
\caption{PLL block diagram}
\label{fig:module:nco:pll_diagram}
\end{figure}

The PLL consists of three components: the phase detector, the loop filter, and
the integrator.
A block diagram of the PLL can be seen in
Figure~\ref{fig:module:nco:pll_diagram} in which the phase detector is
represented by the summing node, the loop filter is $F(s)$, and the integrator
has a transfer function $G(s) = K/s$.
For a given loop filter $F(s)$, the closed-loop transfer function becomes
\[
    H(s) = \frac{ G(s)F(s) }{ 1 + G(s)F(s) }
         = \frac{ KF(s)    }{ s + KF(s)    }
\]
where the loop gain $K$ absorbs all the gains in the loop.

There are several well-known options for designing the loop filter $F(s)$,
which is, in general, a first-order low-pass filter.
In particular we are interested in getting the denominator of $H(s)$ to the
standard form $s^2 + 2\zeta\omega_n s + \omega_n^2$ where $\omega_n$ is the
natural frequency of the filter and $\zeta$ is the damping factor.
This simplifies analysis of the overall transfer function and allows the
parameters of $F(s)$ to ensure stability.

\subsubsection{Active lag design}
From \cite{Best:1997}.
\[
    F(s) = \frac{1 + \tau_2 s}{1 + \tau_1 s}
\]
This gives a closed-loop transfer function
\[
    H(s) = \frac{
                \frac{K}{\tau_1} (1 + s\tau_2)
           } {
                s^2 + s\frac{1 + K\tau_2}{\tau_1} + \frac{K}{\tau_1}
           }
\]
{\bf TODO : check these calculations}
\[
    \omega_n = \sqrt{\frac{K}{\tau_1}}
    \,\,\,\,\,\,
    \zeta = \frac{\omega_n}{2}\left(\tau_2 + \frac{1}{K}\right)
        \rightarrow
    \tau_1 = \frac{K}{\omega_n^2}
    \,\,\,\,\,\,
    \tau_2 = \frac{2\zeta}{\omega_n} - \frac{1}{K}
\]

The open-loop transfer function is
\[
    H'(s) = F(s)G(s) = K \frac{1 + \tau_2 s}{s + \tau_1 s^2}
\]
Taking the bilinear $z$-transform gives the digital filter:
\[
    H'(z) = H'(s)\Bigl.\Bigr|_{s = \frac{1}{2}\frac{1-z^{-1}}{1+z^{-1}}}
          = 2 K \frac{
                (1+\tau_2/2) + 2 z^{-1}     + ( 1 - \tau_2/2)z^{-2}
          } {
                (1+\tau_1/2) -\tau_1 z^{-1} + (-1 + \tau_1/2)z^{-2}
          }
\]



\subsubsection{Active PI design}
Active ``proportional plus integration''
\[
    F(s) = \frac{1 + \tau_2 s}{\tau_1 s}
\]
This gives a closed-loop transfer function
\[
    H(s) = \frac{
                \frac{K}{\tau_1} (1 + s\tau_2)
           } {
                s^2 + s\frac{K\tau_2}{\tau_1} + \frac{K}{\tau_1 + \tau_2}
           }
\]

\[
    \omega_n = \sqrt{\frac{K}{\tau_1}}
    \,\,\,\,\,\,
    \zeta = \frac{\omega_n \tau_2}{2}
        \rightarrow
    \tau_1 = \frac{K}{\omega_n^2}
    \,\,\,\,\,\,
    \tau_2 = \frac{2\zeta}{\omega_n}
\]

The open-loop transfer function is
\[
    H'(s) = F(s)G(s) = K \frac{1 + \tau_2 s}{\tau_1 s^2}
\]
Taking the bilinear $z$-transform gives the digital filter:
\[
    H'(z) = H'(s)\Bigl.\Bigr|_{s = \frac{1}{2}\frac{1-z^{-1}}{1+z^{-1}}}
          = 2 K \frac{
                (1+\tau_2/2) + 2 z^{-1}     + ( 1 - \tau_2/2)z^{-2}
          } {
                \tau_1/2 -\tau_1 z^{-1} + (\tau_1/2)z^{-2}
          }
\]

\begin{figure}
\centering
\subfigure[nco output] {
  \includegraphics[trim = 0mm 0mm 0mm 0mm, clip, width=13cm]{figures.gen/nco_pll_sincos}
}
\subfigure[phase error] {
  \includegraphics[trim = 0mm 0mm 0mm 0mm, clip, width=13cm]{figures.gen/nco_pll_error}
}
\caption{{\tt nco} phase-locked loop demonstration}
\label{fig:module:nco:pll}
\end{figure}

\subsubsection{PLL Interface}

% TODO : explain!
Internally, the {\tt nco\_crcf\_step()} method acts as the integrator, so the
implementation of the PLL in \liquid\ is a little skewed...

An example of the PLL can be seen in figure~\ref{fig:module:nco:pll}.

\begin{description}
\item[{\tt nco\_crcf\_pll\_set\_bandwidth()}]
    sets the bandwidth of the loop filter of the {\tt pll} internal to the
    {\tt nco} object.
\item[{\tt nco\_crcf\_pll\_step()}]
    advance the {\tt pll} phase with a phase error argument to the loop
    filter.
    This method only changes the frequency of the {\tt nco} object and does
    not update the phase until {\tt nco\_crcf\_step()} is invoked.
    This is useful if one wants to only run the PLL periodically and ignore
    several samples.
\end{description}

Here is a minimal example demonstrating the interface to the {\tt nco} object
and the internal phase-locked loop:
\input{listings/nco_pll.example.c.tex}

See also
{\tt examples/nco\_pll\_example.c},
{\tt examples/nco\_pll\_modem\_example.c}

