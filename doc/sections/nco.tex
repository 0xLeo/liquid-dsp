% 
% MODULE : nco (numerically-controlled oscillator)
%

\newpage
\section{nco (numerically-controlled oscillator)}
\label{module:nco}
This section describes the numerically-controlled oscillator (NCO) for carrier
synchronization.

\subsection{{\tt nco} object}
\label{module:nco:nco}
The {\tt nco} object implements an oscillator with two options for internal
phase precision: {\tt LIQUID\_NCO} and {\tt LIQUID\_VCO}.
The {\tt LIQUID\_NCO} implements a numerically-controlled oscillator that uses
a look-up table to generate a complex sinusoid while
the {\tt LIQUID\_VCO} implements a ``voltage-controlled'' oscillator that uses
the {\tt sinf} and {\tt cosf} standard math functions to generate a complex
sinusoid.

\subsubsection{Description of operation}
The {\tt nco} object maintains its phase and frequency states internally.
Various computations--such as mixing--use the phase state for generating
complex sinusoids.
The phase $\theta$ of the {\tt nco} object is updated using the
{\tt nco\_crcf\_step()} method which increments $\theta$ by $\Delta\theta$, the
frequency.
Both the phase and frequency of the {\tt nco} object can be manipulated using
the appropriate {\tt nco\_crcf\_set} and {\tt nco\_crcf\_adjust} methods.
%
Here is a minimal example demonstrating the interface to the {\tt nco}
object:
%
\input{listings/nco_pll.example.c.tex}
%

\subsubsection{Interface}
\label{module:nco:nco:interface}
%
Listed below is the full interface to the {\tt nco} family of objects.
%
\begin{description}
\item[{\tt nco\_crcf\_create(type)}]
    creates an {\tt nco} object of type
    {\tt LIQUID\_NCO} or 
    {\tt LIQUID\_VCO}.
\item[{\tt nco\_crcf\_destroy(q)}]
    destroys an {\tt nco} object, freeing all internally-allocated
    memory.
\item[{\tt nco\_crcf\_print(q)}]
    prints the internal state of the {\tt nco} object to the standard
    output.
\item[{\tt nco\_crcf\_reset(q)}]
    clears in internal state of an {\tt nco} object.
\item[{\tt nco\_crcf\_set\_frequency(q,f)}]
    sets the frequency $f$ (equal to the phase step size
    $\Delta\theta$).
\item[{\tt nco\_crcf\_adjust\_frequency(q,df)}]
    increments the frequency by $\Delta f$.
\item[{\tt nco\_crcf\_set\_phase(q,theta)}]
    sets the internal {\tt nco} phase to $\theta$.
\item[{\tt nco\_crcf\_adjust\_phase(q,dtheta)}]
    increments the internal {\tt nco} phase by $\Delta\theta$.
\item[{\tt nco\_crcf\_step(q)}]
    increments the internal {\tt nco} phase by its
    internal frequency, $\theta \leftarrow \theta + \Delta\theta$
\item[{\tt nco\_crcf\_get\_phase(q)}]
    returns the internal phase of the {\tt nco} object,
    $-\pi \leq \theta < \pi$.
\item[{\tt nco\_crcf\_get\_frequency(q)}]
    returns the internal frequency (phase step size)
\item[{\tt nco\_crcf\_sin(q)}] returns $\sin(\theta)$
\item[{\tt nco\_crcf\_cos(q)}] returns $\cos(\theta)$
\item[{\tt nco\_crcf\_sincos(q,*sine,*cosine)}]
    computes $\sin(\theta)$ and $\cos(\theta)$
\item[{\tt nco\_crcf\_cexpf(q,*y)}] computes $y=e^{j\theta}$
\item[{\tt nco\_crcf\_mix\_up(q,x,*y)}]
    rotates an input sample $x$ by $e^{j\theta}$, storing the result in
    the output sample $y$.
\item[{\tt nco\_crcf\_mix\_down(q,x,*y)}]
    rotates an input sample $x$ by $e^{-j\theta}$, storing the result in
    the output sample $y$.
\item[{\tt nco\_crcf\_mix\_block\_up(q,*x,*y,n)}]
    rotates an $n$-element input array $\vec{x}$ by $e^{j\theta k}$
    for $k \in \{0,1,\ldots,n-1\}$, storing the result in
    the output vector $\vec{y}$.
\item[{\tt nco\_crcf\_mix\_block\_down(q,*x,*y,n)}]
    rotates an $n$-element input array $\vec{x}$ by $e^{-j\theta k}$
    for $k \in \{0,1,\ldots,n-1\}$, storing the result in
    the output vector $\vec{y}$.
\end{description}

\subsection{PLL (phase-locked loop)}
\label{module:nco:pll}
% NOTE : as of [21f7b4] PLL is internal to NCO object
The phase-locked loop object provides a method for synchronizing oscillators
on different platforms.
It uses a second-order integrating loop filter to adjust the frequency of its
{\tt nco} based on an instantaneous phase error input.
As its name implies, a PLL locks the phase of the {\tt nco} object to a
reference signal.
The PLL accepts a phase error and updates the frequency (phase step size) of
the {\tt nco} to track to the phase of the reference.
The reference signal can be another {\tt nco} object, or a signal whose
carrier is modulated with data.
%Frequency modulation (FM), for example, changes its carrier frequency relative
%to an analog audio signal.
%
% FIGURE : nco_pll_block diagram
\begin{figure}
\centering
  \includegraphics[width=10cm]{figures.pgf/nco_pll_diagram}
\caption{PLL block diagram}
\label{fig:module:nco:pll_diagram}
\end{figure}
%
The PLL consists of three components: the phase detector, the loop filter, and
the integrator.
A block diagram of the PLL can be seen in
Figure~\ref{fig:module:nco:pll_diagram} in which the phase detector is
represented by the summing node, the loop filter is $F(s)$, and the integrator
has a transfer function $G(s) = K/s$.
For a given loop filter $F(s)$, the closed-loop transfer function becomes
%
\begin{equation}
    H(s) = \frac{ G(s)F(s) }{ 1 + G(s)F(s) }
         = \frac{ KF(s)    }{ s + KF(s)    }
\end{equation}
%
where the loop gain $K$ absorbs all the gains in the loop.
%
There are several well-known options for designing the loop filter $F(s)$,
which is, in general, a first-order low-pass filter.
In particular we are interested in getting the denominator of $H(s)$ to the
standard form $s^2 + 2\zeta\omega_n s + \omega_n^2$ where $\omega_n$ is the
natural frequency of the filter and $\zeta$ is the damping factor.
This simplifies analysis of the overall transfer function and allows the
parameters of $F(s)$ to ensure stability.


\subsubsection{Active lag design}
\label{module:nco:pll:active_lag}
The active lag PLL \cite{Best:1997} has a loop filter with a transfer
function
%$F(s) = \frac{1 + \tau_2 s}{1 + \tau_1 s}$
$F(s) = (1 + \tau_2 s)/(1 + \tau_1 s)$
where $\tau_1$ and $\tau_2$ are parameters relating to the damping
factor and natural frequency.
%
This gives a closed-loop transfer function
%
\begin{equation}
\label{eqn:nco:pll:H_active_lag}
    H(s) = \frac{
                \frac{K}{\tau_1} (1 + s\tau_2)
           } {
                s^2 + s\frac{1 + K\tau_2}{\tau_1} + \frac{K}{\tau_1}
           }
\end{equation}
%
Converting the denominator of (\ref{eqn:nco:pll:H_active_lag}) into
standard form yeilds the following equations for $\tau_1$ and $\tau_2$:
%
\begin{equation}
    \omega_n = \sqrt{\frac{K}{\tau_1}}
    \,\,\,\,\,\,
    \zeta = \frac{\omega_n}{2}\left(\tau_2 + \frac{1}{K}\right)
        \rightarrow
    \tau_1 = \frac{K}{\omega_n^2}
    \,\,\,\,\,\,
    \tau_2 = \frac{2\zeta}{\omega_n} - \frac{1}{K}
\end{equation}
%
The open-loop transfer function is therefore
\begin{equation}
    H'(s) = F(s)G(s) = K \frac{1 + \tau_2 s}{s + \tau_1 s^2}
\end{equation}
%
Taking the bilinear $z$-transform of $H'(s)$ gives the digital filter:
%
\begin{equation}
    H'(z) = H'(s)\Bigl.\Bigr|_{s = \frac{1}{2}\frac{1-z^{-1}}{1+z^{-1}}}
          = 2 K \frac{
                (1+\tau_2/2) + 2 z^{-1}     + ( 1 - \tau_2/2)z^{-2}
          } {
                (1+\tau_1/2) -\tau_1 z^{-1} + (-1 + \tau_1/2)z^{-2}
          }
\end{equation}
%
A simple 2$^{nd}$-order active lag IIR filter can be designed using the
following method:
%
\begin{Verbatim}[fontsize=\small]
  void iirdes_pll_active_lag(float _w,    // filter bandwidth
                             float _zeta, // damping factor
                             float _K,    // loop gain (1,000 suggested)
                             float * _b,  // output feed-forward coefficients [size: 3 x 1]
                             float * _a); // output feed-back coefficients [size: 3 x 1]
\end{Verbatim}
%



\subsubsection{Active PI design}
\label{module:nco:pll:active_PI}
Similar to the active lag PLL design is the
active ``proportional plus integration'' (PI)
which has a loop filter
%$F(s) = \frac{1 + \tau_2 s}{\tau_1 s}$
$F(s) = (1 + \tau_2 s)/(\tau_1 s)$
where $\tau_1$ and $\tau_2$ are also parameters relating to the damping
factor and natural frequency,
but are different from those in the active lag design.
The above loop filter yields a closed-loop transfer function
%
\begin{equation}
\label{eqn:nco:pll:H_active_PI}
    H(s) = \frac{
                \frac{K}{\tau_1} (1 + s\tau_2)
           } {
                s^2 + s\frac{K\tau_2}{\tau_1} + \frac{K}{\tau_1 + \tau_2}
           }
\end{equation}
%
Converting the denominator of (\ref{eqn:nco:pll:H_active_PI}) into
standard form yeilds the following equations for $\tau_1$ and $\tau_2$:
%
\begin{equation}
    \omega_n = \sqrt{\frac{K}{\tau_1}}
    \,\,\,\,\,\,
    \zeta = \frac{\omega_n \tau_2}{2}
        \rightarrow
    \tau_1 = \frac{K}{\omega_n^2}
    \,\,\,\,\,\,
    \tau_2 = \frac{2\zeta}{\omega_n}
\end{equation}
%
The open-loop transfer function is therefore
%
\begin{equation}
    H'(s) = F(s)G(s) = K \frac{1 + \tau_2 s}{\tau_1 s^2}
\end{equation}
%
Taking the bilinear $z$-transform of $H'(s)$ gives the digital filter
%
\begin{equation}
    H'(z) = H'(s)\Bigl.\Bigr|_{s = \frac{1}{2}\frac{1-z^{-1}}{1+z^{-1}}}
          = 2 K \frac{
                (1+\tau_2/2) + 2 z^{-1}     + ( 1 - \tau_2/2)z^{-2}
          } {
                \tau_1/2 -\tau_1 z^{-1} + (\tau_1/2)z^{-2}
          }
\end{equation}
%
A simple 2$^{nd}$-order active PI IIR filter can be designed using the
following method:
%
\begin{Verbatim}[fontsize=\small]
  void iirdes_pll_active_PI(float _w,    // filter bandwidth
                            float _zeta, // damping factor
                            float _K,    // loop gain (1,000 suggested)
                            float * _b,  // output feed-forward coefficients [size: 3 x 1]
                            float * _a); // output feed-back coefficients [size: 3 x 1]
\end{Verbatim}
%



\subsubsection{PLL Interface}
\label{module:nco:pll:interface}
The {\tt nco} object has an internal PLL interface which only needs to
be invoked before the {\tt nco\_crcf\_step()} method
(see \S\ref{module:nco:nco:interface}) with the appropriate phase
error estimate.
This will permit the {\tt nco} object to automatically track to a
carrier offset for an incoming signal.
%
The {\tt nco} object has the following PLL method extensions to enable a
simplified phase-locked loop interface.
%
\begin{description}
\item[{\tt nco\_crcf\_pll\_set\_bandwidth(q,w)}]
    sets the bandwidth of the loop filter of the {\tt nco} object's
    internal PLL to $\omega$.
\item[{\tt nco\_crcf\_pll\_step(q,dphi)}]
    advances the {\tt nco} object's internal phase with a phase error
    $\Delta\phi$ to the loop filter.
    This method only changes the frequency of the {\tt nco} object and does
    not update the phase until {\tt nco\_crcf\_step()} is invoked.
    This is useful if one wants to only run the PLL periodically and ignore
    several samples.
    See the example code below for help.
\end{description}
%
Here is a minimal example demonstrating the interface to the {\tt nco}
object and the internal phase-locked loop:
%
\input{listings/nco_pll.example.c.tex}
%
See also
{\tt examples/nco\_pll\_example.c} and
{\tt examples/nco\_pll\_modem\_example.c}
located in the main \liquid\ project directory.
%
\begin{figure}
\centering
\subfigure[nco output] {
  \includegraphics[trim = 0mm 0mm 0mm 0mm, clip, width=13cm]{figures.gen/nco_pll_sincos}
}
\subfigure[phase error] {
  \includegraphics[trim = 0mm 6mm 0mm 6mm, clip, width=13cm]{figures.gen/nco_pll_error}
}
\caption{{\tt nco} phase-locked loop demonstration}
\label{fig:module:nco:pll}
\end{figure}
%
An example of the PLL can be seen in Figure~\ref{fig:module:nco:pll}.
Notice that during the first 150 samples the NCO's output signal is
misaligned to the input;
eventually, however, the PLL acquires the phase of the input sinusoid
and the phase error of the NCO's output approaches zero.

