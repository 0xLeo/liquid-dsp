%
% introduction
%

\newpage
\section{Background and History}

\liquid\ is a free and open-source digital signal processing (DSP) library
designed specifically for software-defined radios on embedded platforms.
The aim is to provide a lightweight DSP library that does not rely on a myriad
of external dependencies or proprietary and otherwise cumbersome frameworks.
%
All signal processing elements are designed to be flexible, scalable, and
dynamic, including filters, filter design, oscillators, modems, synchronizers,
and complex mathematical operations.
The source for \liquid\ is written entirely in C so that it can be
compiled quickly with a low memory footprint
and easily deployed on embedded platforms.

\liquid\ was created by J. Gaeddert in 2007 out of necessity to perform
complex digital signal processing algorithms on embedded devices
with a low memory footprint and little computational overhead.
This was a critical step in his PhD thesis to adapt DSP algorithms in
cognitive dynamic-spectrum radios to optimally manage finite radio resources.
The project is not intended to compete with many other well-known and
excellent software radio packages freely available
(such as {\em GNU radio} \cite{gnuradio:web} and {\em OSSIE}
\cite{ossie:web})
but was created as a lightweight library which can be used to augment
these projects' capabilities
or be used in embedded platforms were minimizing overhead is critical.
You will notice that \liquid\ lacks any sort of underlying framework
%found in these packages
for connecting signal processing ``blocks'' or ``components.''
The design was chosen because each application requires the signal
processing block to be redesigned and recompiled for each application
anyway so the notion of a reconfigurable framework is, for the most
part, a flawed concept.
%A large number of researchers want the functionality found in scripted
%languages such as MATLAB \cite{mathworks:web} or GNU Octave
%\cite{octave:web} but need to speed of a real-time processing library so
%that they can test their algorithms in real wireless environments.

In \liquid\ there is
no model for passing data between structures,
no generic interface for data abstraction,
no customized/proprietary data types,
no framework for handling memory management;
this responsibility is left to the designer,
and as a consequence the library provides very little computational overhead.
% TODO : reword [provides] above
% ...
This package does {\em not} provide
graphical user interfaces,
component models, or
debugging tools;
\liquid\ is simply a collection raw signal processing modules
providing flexibility in algorithm development for wireless
communications at the physical layer.


% Key points
%   * open-source software-defined radio DSP algorithms
%   * minimal dependence on external libraries
%   * portable to embedded platforms
%   * flexible configuration
%   * targets cognitive radios and enabling technologies through
%     flexible algorithmic development
%
% Features
%   * automatic test scripts for validation and accuracy
%   * benchmark tool for estimating computational speed on your machine
%

\section{Quick Start Guide}
\label{section:quickstart}
A full description of installation procedures can be found in
Part~\ref{part:installation}.
%\S\ref{section:installation}.
The library can easily be built from source and is available from
several places.
The two most typical means of distribution are a compressed archive
(a {\em tarball}) and cloning the source repository.

\subsection{Building from a Tarball}
\label{section:quickstart:tarball}
If you are building from a tarball
download the compressed archive {\tt liquid-dsp-v.v.v.tar.gz} to your
local machine where {\tt v.v.v} denotes the version of the release
(e.g. {\tt liquid-dsp-\liquidversion.tar.gz}).
% 
% Download the tarball and md5 check:
%
%   $ wget http://ganymede.ece.vt.edu/downloads/liquid-dsp-version.tar.gz
%   $ wget http://ganymede.ece.vt.edu/downloads/liquid-dps.md5
%   $ md5sum --check liquid-dsp.md5
%
% Check the validity of the tarball using MD5...
%
% generate:
%   $ md5sum liquid-dsp-v.v.v.tar.gz > liquid-dsp.md5
%
% check:
%   $ md5sum --check liquid-dsp.md5
%
% You should see a message verifying the file:
%   liquid-dsp-v.v.v.tar.gz: OK
%
% If it fails, do no unpack.
Unpack the tarball
%
\begin{verbatim}
    $ tar -xvf liquid-dsp-v.v.v.tar.gz
\end{verbatim}
%
Move into the directory and run the configure script and make the
library.
%
\begin{verbatim}
    $ cd liquid-dsp-v.v.v
    $ ./configure
    $ make
    # make install
\end{verbatim}
%

\subsection{Cloning the Git Repository}
\label{section:quickstart:git}
%
You may also build the latest version of the code by cloning the
Git repository.
The main repository for \liquid\ is hosted online by {\em github}
\cite{github:web} and can be cloned on your local machine via
%
\begin{verbatim}
    $ git clone git://github.com/jgaeddert/liquid-dsp.git
\end{verbatim}
%
Move into the directory, check out a particular tag, and build as before
with the archive, but with the additional bootstrapping step:
%
\begin{verbatim}
    $ cd liquid-dsp
    $ git checkout v1.2.0
    $ ./bootstrap.sh
    $ ./configure
    $ make
    # make install
\end{verbatim}
%

\subsection{Additional {\tt make} Targets}
\label{section:quickstart:targets}
%
You might also want to build and run the optional validation program
(see \S\ref{section:installation:targets:autotests}) via
\begin{verbatim}
    $ make check
\end{verbatim}
and the benchmarking tool
(see \S\ref{section:installation:targets:benchmarks})
\begin{verbatim}
    $ make bench
\end{verbatim}
%
A comprehensive list of signal processing examples is given in the
{\tt examples} directory.
You may build all of the example binaries at one time by running
\begin{verbatim}
    $ make examples
\end{verbatim}
%
Sometimes, however, it is useful to build one example individually.
This can be accomplished by directly targeting its binary
(e.g. ``{\tt make examples/cvsd\_example}'').
The example then can be run at the command line
(e.g. ``{\tt ./examples/cvsd\_example}'').

%
% SECTION : Data Structures
%
\section{Data Structures in \liquid}
\label{section:data_structures}
Most of \liquid's signal processing elements are C structures which
retain the object's parameters, state, and other useful information.
The naming convention is
{\tt basename\_xxxt\_method} where
{\tt basename} is the base object name (e.g. {\tt interp}),
{\tt xxxt} is the type definition, and
{\tt method} is the object method.
The type definition describes respective output, internal, and input type.
Types are usually {\tt f} to denote standard 32-bit {\it floating point}
precision values and can either be represented as {\tt r} (real) or {\tt c}
(complex).
For example, a {\tt dotprod} (vector dot product) object with complex input
and output types but real internal coefficients operating on 32-bit
floating-point precision values is {\tt dotprod\_crcf}.

Most objects have at least four standard methods:
{\tt create()},
{\tt destroy()},
{\tt print()},
and
{\tt execute()}.
Certain objects also implement a {\tt recreate()} method which operates
similar to that of {\tt realloc()} in C and are used to restructure or
reconfigure an object without completely destroying it and creating it again.
Typically, the user will create the signal processing object independent of
the external (user-defined) data array.
The object will manage its own memory until its {\tt destroy()} method is
invoked.
A few points to note:
\begin{enumerate}
\item The object is only used to maintain the state of the signal processing
      algorithm.
      For example, a finite impulse response filter
      (\S\ref{module:filter:firfilt}) needs to retain the filter
      coefficients and a buffer of input samples.
      Certain algorithms which do not retain information (those which are
      memoryless) do not use objects.
      For example, {\tt design\_rnyquist\_filter()}
      (\S\ref{module:filter:firdes:rnyquist})
      calculates the coefficients of a square-root raised-cosine filter,
      a processing algorithm which does not need to maintain a state
      after its completion.
\item While the objects do retain internal memory, they typically operate on
      external user-defined arrays.
      As such, it is strictly up to the user to manage his/her own
      memory.
      Shared pointers are a great way to cause memory leaks, double-free
      bugs, and severe headaches.
      The bottom line is to remember that if you created a mess, it is
      your responsibility to clean it up.
\item Certain objects will allocate memory internally, and consequently
      will use more memory than others.
      This memory will only be freed when the appropriate {\tt delete()}
      method is invoked.
      Don't forget to clean up your mess!
%\item ...
\end{enumerate}


\subsection{Basic Life Cycle}
\label{section:data_structures:lifecycle}

Listed below is an example of the basic life cycle of a
{\tt iirfilt\_crcf} object (infinite impulse response filter with
complex float inputs/outputs, and real float coefficients).
The design parameters of the filter are specified in the {\em options}
section near the top of the file.
The {\tt iirfilt\_crcf} filter object is then created from the design
using the {\tt iirfilt\_crcf\_create()} method.
Input and output data arrays of type {\tt float complex} are allocated
and a loop is run which initializes each input sample and computes a
filter output using the {\tt iirfilt\_crcf\_execute()} method.
Finally the filter object is destroyed using the
{\tt iirfilt\_crcf\_destroy()} method,
freeing all of the object's internally allocated memory.
%
\input{listings/lifecycle.example.c.tex}
%
A more comprehensive example is given in the example file
{\tt examples/iirfilt\_crcf\_example.c},
located under the main \liquid\ project directory.

\subsection{Why C?}
\label{section:data_structures:why_C}
A commonly asked question is ``why C and not C++?''
The answer is simple: {\em portability}.
The project's aim is to provide a lightweight DSP library for
software-defined radio that does not rely on a myriad of dependencies.
While C++ is a fine language for many projects
(and theoretically runs just as fast as C),
it is not as portable to embedded platforms as C and typically has a
larger memory footprint.
Furthermore, the majority of functions simply perform complex operations on a
data sequence and do not require a high-level object-oriented programming
interface.
The significance of object-oriented programming is the techniques used,
not the languages describing it.

While a number of signal processing elements in \liquid\ use structures, these
are simply to save the internal state of the object.
For instance, a {\tt firfilt\_crcf} (finite impulse response filter) object
is just a structure which contains|among other things|the filter taps
(coefficients) and an input buffer.
This simplifies the interface to the user; one only needs to ``push'' elements
into the filter's internal buffer and ``execute'' the dot product when
desired.
This could also be accomplished with classes, a construct specific to C++ and
other high-level object-oriented programming languages;
however,
for the most part, C++ polymorphic data types and abstract base classes are
unnecessary for basic signal processing, and primarily just serve to reduce
the code base of a project at the expense of increased compile time and
memory overhead.
Furthermore, while C++ templates can certainly be useful for library
development
their benefits are of limited use to signal processing and can be circumvented
through the use of pre-processor macros at the gain of increasing the
portability of the code.
Under the hood, the C++ compiler's pre-processor expands templates and classes
before actually compiling the source anyway, so in this sense they are
equivalent to the second-order macros used in \liquid.

The C programming language has a rich history in system
programming|specifically targeting embedded applications|and is the
basis behind many well-known projects including the Linux
kernel~\cite{linux-kernel:web}
and the python programming language \cite{python:web}.
Having said this, high-level frameworks and graphical interfaces are
much more suited to be written in C++ and will beat an implementation in
C any day but lie far outside the scope of this project.

\subsection{Data Types}
\label{section:data_structures:data_types}
The majority of signal processing for SDR is performed at complex baseband.
Complex numbers are handled in \liquid\ by defining data type
{\tt liquid\_float\_complex} which is simply a place-holder for the
standard
C math type {\tt float complex} and C++ type {\tt std::complex<float>}.
{\em There are no custom/proprietary data types in \liquid!}%
\footnote{
    The only exception to this are the fixed-point data types,
    defined in the \liquidfpm\ library which hasn't been released yet,
    and even these data types are actually standard signed integers.}
Custom data types only promote lack of interoperability between
libraries requiring conversion procedures which slow down computation.
For those of you who like to dig through the source code might have
stumbled upon the {\tt typedef} macros at the beginning of the global 
header file {\tt include/liquid.h} which creates new complex data types
based on the compiler, 
(e.g. {\tt liquid\_complex\_float}).
While technically this code does define of a new type specification,
its purpose is for compatibility between compilers and programming
language
(see \S\ref{section:data_structures:c++} on C++ portability),
and is binary compatible with the standard C99 specification.
In fact, these data types are only used in the header file and should
not be used when programming.
For example, the following example program demonstrates the interface in
C:
%
\input{listings/nco.c.tex}
%

%Data bytes for digital communication (e.g. the ``message'' signal) are
%retained in arrays of {\tt unsigned char} for portability.

\subsection{Building/Linking with C++}
\label{section:data_structures:c++}
Although \liquid\ is written in C, it can be seamlessly compiled and linked
with C++ source files.
Here is a C++ example comparable to the C program listed in the previous
section:
%
\input{listings/nco.cc.tex}
%
It is important, however, to link the code with a C++ linker rather than
a C linker.
For example, if the above program ({\tt nco.cc}) is compiled with
{\tt g++} it must also be linked with {\tt g++}, viz
%
\begin{Verbatim}[fontsize=\small]
    $ g++ -c -o nco.cc.o nco.cc
    $ g++ nco.cc.o -o nco -lm -lc -lliquid
\end{Verbatim}
%

%\section{Complex Baseband}
%\label{section:complex_baseband}

\subsection{Learning by example}
\label{section:learning_by_example}
While this document contains numerous examples listed in the text, they are
typically condensed to demonstrate only the interface.
The {\tt examples/} subdirectory includes more extensive demonstrations and
numerous examples for all the signal processing components.
Many of these examples write an output file which can be read by
{\em octave}~\cite{octave:web} to display the results graphically.
For a brief description of each of these examples, see {\tt examples/README}.

