% 
% TUTORIAL : ofdmflexframe
%

\newpage
\section{Tutorial: OFDM}
\label{tutorial:ofdmflexframe}

In the previous tutorials we have created only the basic building blocks
for wireless communication.
We have also used the basic {\tt framegen64} and {\tt framesync64}
objects to transmit and receive simple framing data.
This tutorial extends...


%
% SUBSECTION : problem statement
%
\subsection{Problem Statement}
\label{tutorial:ofdmflexframe:problem}



%
% SUBSECTION : 
%
\subsection{Setting up the Environment}
\label{tutorial:ofdmflexframe:environment}

As with the other tutorials I assume that you are using {\tt gcc} to
compile your programs and link to appripriate libraries.
Create a new file {\tt ofdmflexframe.c} and include the headers
{\tt stdio.h},
{\tt stdlib.h},
{\tt math.h},
{\tt complex.h}, and
{\tt liquid/liquid.h}.
Add the {\tt int main()} definition so that your program looks like
this:
%
\input{tutorials/ofdmflexframe_init_tutorial.c.tex}
%
Compile and link the program using {\tt gcc}:
%
\begin{Verbatim}[fontsize=\small]
    $ gcc -Wall -o ofdmflexframe -lm -lc -lliquid ofdmflexframe.c
\end{Verbatim}
%
The flag ``{\tt -Wall}'' tells the compiler to print all warnings
(unused and uninitialized variables, etc.),
``{\tt -o ofdmflexframe}'' specifies the name of the output program is
``{\tt ofdmflexframe}'', and
``{\tt -lm -lc -lliquid}'' tells the linker to link the binary against
the math, standard C, and \liquid\ DSP libraries, respectively.
Notice that the above command invokes both the compiler and the linker
collectively.
%While it is usually preferred to build an intermediate object...
%
If the compiler did not give any errors, the output executable
{\tt ofdmflexframe} is created which can be run as
%
\begin{Verbatim}[fontsize=\small]
    $ ./ofdmflexframe
\end{Verbatim}
%
and should simply print ``{\tt done.}'' to the screen.
You are now ready to add functionality to your program.



%
% SUBSECTION : frame generator
%
\subsection{Creating the Frame Generator}
\label{tutorial:ofdmflexframe:framegen}

%
% SUBSECTION : frame synchronizer
%
\subsection{Creating the Frame Synchronizer}
\label{tutorial:ofdmflexframe:framesync}

%
% SUBSECTION : 
%
\subsection{Putting it All Together}
\label{tutorial:ofdmflexframe:xxx}

%
% SUBSECTION : 
%
\subsection{Final Program}
\label{tutorial:ofdmflexframe:completed}

%
\input{tutorials/ofdmflexframe_tutorial.c.tex}
%
