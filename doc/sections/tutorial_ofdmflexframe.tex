% 
% TUTORIAL : ofdmflexframe
%

\newpage
\section{Tutorial: OFDM Framing}
\label{tutorial:ofdmflexframe}

In the previous tutorials we have created only the basic building blocks
for wireless communication.
We have also used the basic {\tt framegen64} and {\tt framesync64}
objects to transmit and receive simple framing data.
This tutorial extends the work on the previous tutorials by introducing
a flexible framing structure that uses a parallel data transmission
scheme that permits arbitrary parametrization
(modulation, forward error-correction, payload length, etc.)
with minimal reconfiguration at the receiver.


%
% SUBSECTION : problem statement
%
\subsection{Problem Statement}
\label{tutorial:ofdmflexframe:problem}
% talking points:
%   * what is OFDM (briefly)
%   * the benefits of OFDM (briefly)
%   * current standards using OFDM
The framing tutorial (\S\ref{tutorial:framing}) loaded data serially
onto a single carrier.
Another option is to load data onto many carriers in parallel;
however it is desirable to do so such that bandwidth isn't wasted.
By allowing the ``subcarriers'' to overlap in frequency,
the system approaches the theoretical maximum capacity of the channel.
Several multiplexing schemes are possible,
but by far the most common is generically known as
orthogonal frequency divisional multiplexing (OFDM)
which uses a square temporal pulse shaping filter for each subcarrier,
separated in frequency by the inverse of the symbol rate.
This conveniently allows data to be loaded into the input of an inverse
discrete Fourier transform (DFT) at the transmitter and
(once time and carrier synchronized)
de-multiplexed with a regular DFT at the receiver.
For computational efficiency the DFT may be implemented with a fast
Fourier transform (FFT) which is mathematically equivalent
but considerably faster.
Furthermore, because of the cyclic nature of the DFT a certain portion
(usually on the order of 10\%)
of the tail of the generated symbol may be copied to its head before
transmitting;
this is known as the {\em cyclic prefix} which can eliminate
inter-symbol interference in the presence of multi-path channel
environments.
Carrier frequency and symbol timing offsets can be tracked and corrected
by inserting known {\em pilot subcarriers} in the signal at the
transmitter;
because the receiver knows the pilot symbols it can make an accurate
estimate of the channel conditions for each OFDM symbol.
%
% TODO : insert figures
%
As an example, the well-known Wi-Fi 802.11a standard uses OFDM with
64 subcarriers
(52 for data, 4 pilots, and 8 disabled for guard bands)
and a 16-sample cyclic prefix.
%
%For more details of OFDM I refer the interested reader to the following
%excellent publications...

In this tutorial we will create a simple pair of OFDM framing objects;
the generator ({\tt ofdmflexframegen}),
like the {\tt framegen64} object,
has a simple interface that accepts raw data in, frame samples out.
The synchronizer ({\tt ofdmflexframesync}),
like the {\tt framesync64} object,
accepts samples and invokes a callback function for each frame that it
detects,
compensating for sample timing and carrier offsets and multi-path
channels.
The framing objects can be created with nearly any even-length transform
(number of subcarriers),
cyclic prefix, and arbitrary null/pilot/data subcarrier allocation.%
\footnote{While nearly any arbitrary configuration is supported, the
          performance of synchronization is greatly dependent upon the
          choice of the number, type, and allocation of subcarriers.
          % See \S\ref{xxx} for a recommendation...
          }
Furthermore, the OFDM frame generator permits many different parameters
(e.g. modulation/coding schemes, payload length)
which are detected automatically at the receiver without any work on
your part.


%
% SUBSECTION : 
%
\subsection{Setting up the Environment}
\label{tutorial:ofdmflexframe:environment}

As with the other tutorials I assume that you are using {\tt gcc} to
compile your programs and link to appropriate libraries.
Create a new file {\tt ofdmflexframe.c} and include the headers
{\tt stdio.h},
{\tt stdlib.h},
{\tt math.h},
{\tt complex.h}, and
{\tt liquid/liquid.h}.
Add the {\tt int main()} definition so that your program looks like
this:
%
\input{tutorials/ofdmflexframe_init_tutorial.c.tex}
%
Compile and link the program using {\tt gcc}:
%
\begin{Verbatim}[fontsize=\small]
    $ gcc -Wall -o ofdmflexframe ofdmflexframe.c -lm -lc -lliquid
\end{Verbatim}
%
The flag ``{\tt -Wall}'' tells the compiler to print all warnings
(unused and uninitialized variables, etc.),
``{\tt -o ofdmflexframe}'' specifies the name of the output program is
``{\tt ofdmflexframe}'', and
``{\tt -lm -lc -lliquid}'' tells the linker to link the binary against
the math, standard C, and \liquid\ DSP libraries, respectively.
Notice that the above command invokes both the compiler and the linker
collectively.
%While it is usually preferred to build an intermediate object...
%
If the compiler did not give any errors, the output executable
{\tt ofdmflexframe} is created which can be run as
%
\begin{Verbatim}[fontsize=\small]
    $ ./ofdmflexframe
\end{Verbatim}
%
and should simply print ``{\tt done.}'' to the screen.
You are now ready to add functionality to your program.



%
% SUBSECTION : framing structure
%
\subsection{OFDM Framing Structure}
\label{tutorial:ofdmflexframe:structure}
%
In this tutorial we will be using the {\tt ofdmflexframegen} and
{\tt ofdmflexframesync} objects in \liquid\ which realize the
framing generator (transmitter) and synchronizer (receiver).
The OFDM framing structure is briefly described here
(for a more detailed description, see
\S\ref{module:framing:ofdmflexframe}).
The {\tt ofdmflexframe} generator and synchronizer objects
together realize a simple framing structure
for loading data onto a reconfigurable OFDM physical layer.
The generator encapsulates an 8-byte user-defined header
and a variable-length buffer of uncoded payload data
and fully encodes a frame of OFDM symbols ready for transmission.
The user may define many physical-layer parameters of the transmission,
including
  the number of subcarriers and their allocation (null/pilot/data),
  cyclic prefix length,
  forward error-correction coding,
  modulation scheme,
  and others.
The synchronizer requires the same number of subcarriers, cyclic prefix,
and subcarrier allocation as the transmitter, but can automatically
determine the payload length, modulation scheme, and forward
error-correction of the receiver.
Furthermore, the receiver can compensate for carrier phase/frequency and
timing offsets as well as multi-path fading and noisy channels.
The received data are returned via a callback function which includes
the modulation and error-correction schemes used
as well as certain receiver statistics such as the
received signal strength (\S\ref{module:agc}),
and error vector magnitude (\S\ref{module:modem:digital:evm}).

% TODO : include figures/diagrams here (?)


%
% SUBSECTION : frame generator
%
\subsection{Creating the Frame Generator}
\label{tutorial:ofdmflexframe:framegen}
%
The {\tt ofdmflexframegen} object can be generated with the
{\tt ofdmflexframegen\_create(M,c,p,props)} method which accepts four arguments:
%
\begin{itemize}
\item $M$ is an {\tt unsigned int} representing the total number of
    subcarriers
\item $c$ is an {\tt unsigned int} representing the length of the
    cyclic prefix
\item $\vec{p}$ is an $M$-element array of {\tt unsigned char} which
    gives the subcarrier allocation (e.g. which subcarriers
    are nulled/disabled, which are pilots, and which carry data).
    Setting to {\tt NULL} tells the {\tt ofdmflexframegen} object to use
    the default subcarrier allocation
    (see \S\ref{module:framing:ofdmflexframe:subcarrier_allocation} for
    details);
\item {\tt props} is a special structure called
    {\tt ofdmflexframegenprops\_s}
    which gives some basic properties including
    the inner/outer forward error-correction scheme(s) to use
    ({\tt fec0}, {\tt fec1}),
    and the modulation scheme ({\tt mod\_scheme}) and depth
    ({\tt mod\_depth}).
    The properties object can be initialized to its default by using
    {\tt ofdmflexframegenprops\_init\_default()}.
\end{itemize}
%
To begin, create a frame generator having 64 subcarriers with cyclic
prefix of 16 samples, the default subcarrier allocation, and
default properties as
%
\begin{Verbatim}[fontsize=\small]
    // create frame generator with default parameters
    ofdmflexframegen fg = ofdmflexframegen_create(64, 16, NULL, NULL);
\end{Verbatim}
%
%For now we will only set the payload length as a framing property,
%however the structure allows...
As with all structures in \liquid\ you will need to invoke the
corresponding {\tt destroy()} method when you are finished with the
object.

Now allocate memory for the header (8 bytes) and payload (120 bytes)
data arrays.
Raw ``message'' data are stored as arrays of type {\tt unsigned char} in
\liquid.
%
\begin{Verbatim}[fontsize=\small]
    unsigned char header[8];
    unsigned char payload[120];
\end{Verbatim}
%
Initialize the header and payload arrays with whatever values you wish.
%
Finally you will need to create a buffer for storing the frame samples.
Unlike the {\tt framegen64} object in the previous tutorial which
generates the entire frame at once,
the {\tt ofdmflexframegen} object generates each symbol independently.
%This reduces the amount of memory required and...
For this framing structure you will need to allocate $M+c$ samples of
type {\tt float complex} (for this example $M+c = 64+16 = 80$), viz.
%
\begin{Verbatim}[fontsize=\small]
    float complex buffer[80];
\end{Verbatim}
%
Generating the frame consists of two steps: assemble and write.
Assembling the frame simply involves invoking
the {\tt ofdmflexframegen\_assemble(fg,header,payload,payload\_len)}
method which accepts the frame generator object as well as the header
and payload arrays we initialized earlier.
Internally, the object encodes and modulates the frame, but does not
write the OFDM symbols yet.
To write the OFDM time-series symbols, invoke the
%{\tt ofdmflexframegen\_writesymbol(fg,buffer,*num\_written)}
{\tt ofdmflexframegen\_writesymbol()}
method.
This method accepts three arguments:
  the frame generator object,
  the output buffer we created earlier,
  and the pointer to an integer to indicate the number of samples that
  have been written to the buffer.
The last argument is necessary because not all of the symbols in the
frame are the same size (the first several symbols in the preamble do
not have a cyclic prefix).
Invoking the {\tt ofdmflexframegen\_writesymbol()} method repeatedly
generates each symbol of the frame
and returns a flag indicating if the last symbol in the frame has been
written.

Add the instructions to assemble and write a frame one symbol at a time
to your source code:
%
\begin{Verbatim}[fontsize=\small]
    // assemble the frame and print
    ofdmflexframegen_assemble(fg, header, payload, payload_len);
    ofdmflexframegen_print(fg);

    // generate the frame one OFDM symbol at a time
    int last_symbol=0;          // flag indicating if this is the last symbol
    unsigned int num_written;   // number of samples written to the buffer
    while (!last_symbol) {
        // write samples to the buffer
        last_symbol = ofdmflexframegen_writesymbol(fg, buffer, &num_written);

        // print status
        printf("ofdmflexframegen wrote %3u samples %s\n",
            num_written,
            last_symbol ? "(last symbol)" : "");
    }
\end{Verbatim}
%
That's it!
This completely assembles the frame complete with error-correction
coding, pilot subcarriers, and the preamble necessary for
synchronization.
You may generate another frame simply by
  initializing the data in your {\tt header} and {\tt payload} arrays,
  assembling the frame,
  and then writing the symbols to the buffer.
Keep in mind, however, that the buffer is overwritten each time you
invoke {\tt ofdmflexframegen\_writesymbol()},
so you will need to do something with the data with each iteration of
the loop.
%
Your program should now look similar to this:
%
\input{tutorials/ofdmflexframe_basic_tutorial.c.tex}
%
Running the program should produce an output similar to this:
%
\begin{Verbatim}[fontsize=\small]
    ofdmflexframegen:
        num subcarriers     :   64
          * NULL            :   14
          * pilot           :   6
          * data            :   44
        cyclic prefix len   :   16
        properties:
          * mod scheme      :   quaternary phase-shift keying (2 b/s)
          * fec (inner)     :   none
          * fec (outer)     :   none
          * CRC scheme      :   CRC (16-bit)
        payload:
          * decoded bytes   :   120
          * encoded bytes   :   122
          * modulated syms  :   488
        total OFDM symbols  :   23
          * S0 symbols      :   3 @ 64
          * S1 symbols      :   1 @ 80
          * header symbols  :   7 @ 80
          * payload symbols :   12 @ 80
        spectral efficiency :   0.5357 b/s/Hz
    ofdmflexframegen wrote  64 samples 
    ofdmflexframegen wrote  64 samples 
    ofdmflexframegen wrote  64 samples 
    ofdmflexframegen wrote  80 samples 
    ofdmflexframegen wrote  80 samples 
      ...
    ofdmflexframegen wrote  80 samples (last symbol)
    done.
\end{Verbatim}
%
Notice that the {\tt ofdmflexframegen\_print()} method gives a lot of
information, including
the number of null, pilot, and data subcarriers,
the number of modulated symbols,
the number of OFDM symbols,
and the resulting spectral efficiency.
%\footnote{The spectral efficiency is computed as the number of raw input
%          bytes to resulting generated samples...}
Furthermore, notice that the first three symbols have only 64 samples
while the remaining have 80;
these first three symbols are actually part of the preamble to help the
synchronizer detect the presence of a frame and estimate symbol timing
and carrier frequency offsets.
%Furthermore, because we aren't using any forward error correction...

%
% SUBSECTION : frame synchronizer
%
\subsection{Creating the Frame Synchronizer}
\label{tutorial:ofdmflexframe:framesync}
The OFDM frame synchronizer's purpose is to detect the presence of a
frame, correct for channel impairments (such as a carrier frequency
offset), decode the data (correct for errors in the presence of noise),
and pass the resulting data back to the user.
In our program we will pass to the frame synchronizer samples in the
buffer created by the generator, without adding noise, carrier frequency
offsets, or other channel impairments.
%
The {\tt ofdmflexframesync} object can be generated with the
{\tt ofdmflexframesync\_create(M,c,p,callback,userdata)} method which
accepts five arguments:
%
\begin{itemize}
\item $M$ is an {\tt unsigned int} representing the total number of
    subcarriers
\item $c$ is an {\tt unsigned int} representing the length of the
    cyclic prefix
\item $\vec{p}$ is an $M$-element array of {\tt unsigned char} which
    gives the subcarrier allocation
    (see \S\ref{tutorial:ofdmflexframe:framegen})
\item {\tt callback}
    is a pointer to your callback function which will be invoked each
    time a frame is found and decoded.
\item {\tt userdata}
    is a {\tt void} pointer that is passed to the callback function each
    time it is invoked.
    This allows you to easily pass data from the callback function.
    Set to {\tt NULL} if you don't wish to use this.
\end{itemize}
%
Notice that the first three arguments are the same as in the
{\tt ofdmflexframegen\_create()} method;
the values of these parameters at the synchronizer need to match those
at the transmitter in order for the synchronizer to operate properly.
%
When the synchronizer does find a frame, it attempts to decode the header
and payload and invoke a user-defined callback function.%
\footnote{a basic description of how callback functions work is given in
          the basic framing tutorial in
          \S\ref{tutorial:framing:framesync}.}
The callback function for the {\tt ofdmflexframesync} object has seven
arguments and looks like this:
%
\begin{Verbatim}[fontsize=\small]
    int ofdmflexframesync_callback(unsigned char *  _header,
                                   int              _header_valid,
                                   unsigned char *  _payload,
                                   unsigned int     _payload_len,
                                   int              _payload_valid,
                                   framesyncstats_s _stats,
                                   void *           _userdata);
\end{Verbatim}
%
The callback is typically defined to be {\tt static} and is passed to
the instance of {\tt ofdmflexframesync} object when it is created.
Here is a brief description of the callback function's arguments:
%
\begin{description}
\item[{\tt \_header}]
    is a pointer to the 8 bytes of decoded header data
    (remember that {\tt header[8]} array you created with the
    {\tt ofdmflexframegen} object?).
    This pointer is not static and cannot be used after returning from
    the callback function.
    This means that it needs to be copied locally before returning
    in order for you to retain the data.
\item[{\tt \_header\_valid}]
    is simply a flag to indicate if the header passed its cyclic
    redundancy check
    (``{\tt 0}'' means invalid, ``{\tt 1}'' means valid).
    If the check fails then the header data have been corrupted beyond
    the point that internal error correction can recover;
    in this situation the payload cannot be recovered.
\item[{\tt \_payload}]
    is a pointer to the decoded payload data.
    Like the header,
    this pointer is not static and cannot be used after returning from
    the callback function.
    Again, this means that it needs to be copied locally for you to retain the
    data.
    When the header cannot be decoded ({\tt \_header\_valid == 0})
    this value is set to {\tt NULL}.
\item[{\tt \_payload\_len}]
    is the length (number of bytes) of the payload array.
    When the header cannot be decoded ({\tt \_header\_valid == 0})
    this value is set to {\tt 0}.
\item[{\tt \_payload\_valid}]
    is simply a flag to indicate if the payload passed its cyclic
    redundancy check
    (``{\tt 0}'' means invalid, ``{\tt 1}'' means valid).
    As with the header,
    if this flag is zero then the payload almost certainly contains
    errors.
\item[{\tt \_stats}]
    is a synchronizer statistics construct that indicates some useful
    PHY information to the user (such as RSSI and EVM).
    We will ignore this information in our program, but it can be quite
    useful for certain applications.
    For more information on the {\tt framesyncstats\_s} structure, see
    \S\ref{module:framing:framesyncstats_s}.
\item[{\tt \_userdata}]
    is a {\tt void} pointer given to the
    {\tt ofdmflexframesync\_create()} method
    that is passed to this callback function and can represent anything
    you want it to.
    Typically this pointer is a vehicle for getting the header and
    payload data (as well as any other pertinent information)
    back to your main program.
\end{description}
%
This can seem a bit overwhelming at first, but relax!
The next version of our program will only add about 20 lines of code to
our previous program.

%
% SUBSECTION : 
%
\subsection{Putting it All Together}
\label{tutorial:ofdmflexframe:xxx}
First create your callback function at the beginning of the file, just
before the {\tt int main()} definition;
you may give it whatever name you like (e.g. {\tt mycallback()}).
For now ignore all the function inputs and just print a message to the
screen that indicates that the callback has been invoked,
and return the integer zero ({\tt 0}).
This return value for the callback function should always be zero
and is reserved for future development.
Within your {\tt main()} definition, create an instance of
{\tt ofdmflexframesync} using the {\tt ofdmflexframesync\_create()}
method, passing it
  64 for the number of subcarriers,
  16 for the cyclic prefix length,
  {\tt NULL} for the subcarrier allocation (default),
  {\tt mycallback}, and
  {\tt NULL} for the userdata.
Print the newly created synchronizer object to the screen if you like:
%
\begin{Verbatim}[fontsize=\small]
    ofdmflexframesync fs = ofdmflexframesync_create(64, 16, NULL, mycallback, NULL);
\end{Verbatim}
%
Within the {\tt while} loop that writes the frame symbols to the buffer,
invoke the synchronizer's {\tt execute()} method,
passing to it the frame synchronizer object you just created ({\tt fs}),
the buffer of frame symbols,
and the number of samples written to the buffer ({\tt num\_written}):
%
\begin{Verbatim}[fontsize=\small]
    ofdmflexframesync_execute(fs, buffer, num_written);
\end{Verbatim}
%
Finally, destroy the frame synchronizer object along with the frame
generator at the end of the file.
That's it!
Your program should look something like this:
%
\input{tutorials/ofdmflexframe_intermediate_tutorial.c.tex}
%
Compile and run your program as before and verify that your callback
function was indeed invoked.
Your output should look something like this:
%
\begin{Verbatim}[fontsize=\small]
    ofdmflexframegen:
        ...
    ofdmflexframesync:
        num subcarriers     :   64
          * NULL            :   14
          * pilot           :   6
          * data            :   44
        cyclic prefix len   :   16
    ***** callback invoked!
      header (valid)
      payload (valid)
    done.
\end{Verbatim}
%
Your program has demonstrated the basic functionality of the OFDM frame
generator and synchronizer.
The previous tutorial on framing (\S\ref{tutorial:framing}) added
a carrier offset and noise to the signal before synchronizing;
these channel impairments are addressed in the next section.
%
% The {\tt examples/ofdmflexframesync\_advanced\_example.c} includes a
% more sophisticated channel model including multi-path fading as well
% as carrier frequency offset.
%

%
% SUBSECTION : FINAL PROGRAM
%
\subsection{Final Program}
\label{tutorial:ofdmflexframe:final}

In this last portion of the OFDM framing tutorial, we will modify our
program to change the modulation and coding schemes from their default
values
as well as add channel impairments (noise and carrier frequency offset).
Information on different modulation schemes can be found in
  \S\ref{module:modem:digital};
information on different forward error-correction schemes and validity
checks cane be found in
  \S\ref{module:fec}.
%
To begin, add the following parameters to the beginning of your
{\tt main()} definition with the other options:
%
\begin{Verbatim}[fontsize=\small]
    modulation_scheme ms = LIQUID_MODEM_PSK8;   // payload modulation scheme
    fec_scheme fec0  = LIQUID_FEC_NONE;         // inner FEC scheme
    fec_scheme fec1  = LIQUID_FEC_HAMMING128;   // outer FEC scheme
    crc_scheme check = LIQUID_CRC_32;           // data validity check
    float dphi  = 0.001f;                       // carrier frequency offset
    float SNRdB = 20.0f;                        // signal-to-noise ratio [dB]
\end{Verbatim}
%
The first five options define which modulation, coding, and
error-checking schemes should be used in the framing structure.
The {\tt dphi} and {\tt SNRdB} are the carrier frequency offset
($\Delta\phi$) and signal-to-noise ratio (in decibels), respectively.
%
To change the framing generator properties, create an instance of the
{\tt ofdmflexframegenprops\_s} structure,
query the current properties list with
{\tt ofdmflexframegen\_getprops()},
override with the properties of your choice,
and then reconfigure the frame generator with
{\tt ofdmflexframegen\_setprops()}, viz.
%
\begin{Verbatim}[fontsize=\small]
    // re-configure frame generator with different properties
    ofdmflexframegenprops_s fgprops;
    ofdmflexframegen_getprops(fg, &fgprops);
    fgprops.check           = check;
    fgprops.fec0            = fec0;
    fgprops.fec1            = fec1;
    fgprops.mod_scheme      = ms;
    ofdmflexframegen_setprops(fg, &fgprops);
\end{Verbatim}
%
%It is {\em always} a good idea to call
%{\tt ofdmflexframegenprops\_init\_default()} on the object before
%setting the properties;
%this ensures that all properties within the structure are initialized
%with reasonable values.
%
Add this code somewhere after you create the frame generator, but before
you assemble the frame.

Adding channel impairments can be a little tricky.
We have specified the signal-to-noise ratio in decibels (dB) but need to
compute the equivalent noise standard deviation.
Assuming that the signal power is unity, the noise standard deviation is
just $\sigma_n = 10^{-\textup{SNRdB}/20}$.
%
The carrier frequency offset can by synthesized with a phase variable
that increases by a constant for each sample, $k$.
That is, $\phi_{k} = \phi_{k-1} + \Delta\phi$.
Each sample in the buffer can be multiplied by the resulting complex
sinusoid generated by this phase, with noise added to the result:
\[
    \textup{buffer}[k] \leftarrow \textup{buffer}[k] e^{j\phi_{k}} + \sigma_n(n_i + jn_q)
\]
Initialize the variables for noise standard deviation and carrier phase
before the {\em while} loop as
%
\begin{Verbatim}[fontsize=\small]
    float nstd = powf(10.0f, -SNRdB/20.0f); // noise standard deviation
    float phi = 0.0f;                       // channel phase
\end{Verbatim}
%
Create an inner loop (inside the {\em while} loop) that modifies the
contents of the buffer after the frame generator, but before the frame
synchronizer:
%
\begin{Verbatim}[fontsize=\small]
    // channel impairments
    for (i=0; i<num_written; i++) {
        buffer[i] *= cexpf(_Complex_I*phi); // apply carrier offset
        phi += dphi;                        // update carrier phase
        cawgn(&buffer[i], nstd);            // add noise
    }
\end{Verbatim}
%
Your program should look something like this:
%
\input{tutorials/ofdmflexframe_advanced_tutorial.c.tex}
%
Run this program to verify that the frame is indeed detected
and the payload is received free of errors.
For a more detailed program, see
{\tt examples/ofdmflexframesync\_example.c}
in the main \liquid\ directory;
this example also demonstrates setting different properties of the
frame, but permits options to be passed to the program from the command
line, rather than requiring the program to be re-compiled.
%
Play around with various combinations of options in the program, such as
increasing the number of subcarriers,
modifying the modulation scheme, decreasing the signal-to-noise ratio,
and applying different forward error-correction schemes.

\begin{enumerate}
\item What happens to the spectral efficiency of the frame when you
      increase the payload from 120 bytes to 400?%
      \footnote{{\em A:} the spectral efficiency increases from 0.5357
                to 0.8439 because the preamble accounts for less of
                frame (less overhead)}
      %
      when you decrease the cyclic prefix from 16 samples to 4?%
      \footnote{{\em A:} the spectral efficiency increases from 0.5357
                to 0.61686 because fewer samples are used for each OFDM
                symbol (less overhead)}
      %
      when you increase the number of subcarriers from 64 to 256?%
      \footnote{{\em A:} the spectral efficiency decreases from 0.5357
                to 0.400 because the preamble accounts for {\em more} of
                the frame (increased overhead).}
\item What happens when the frame generator is created with 64
      subcarriers and the synchronizer is created with only 62?%
      \footnote{{\em A:} the synchronizer cannot detect frame
                because the subcarriers don't match (different pilot
                locations, etc.}
      %
      when the cyclic prefix lengths don't match?%
      \footnote{{\em A:} the synchronizer cannot decode header because of
                symbol timing  mis-alignment.}
\item What happens when you decrease the SNR from 20~dB to 10~dB?%
      \footnote{{\em A:} the payload will probably be invalid because of
                too many errors.}
      %
      when you decrease the SNR to 0~dB?%
      \footnote{{\em A:} the frame header will probably be invalid
                because of too many errors.}
      %
      when you decrease the SNR to -10~dB?%
      \footnote{{\em A:} the frame synchronizer will probably miss the
                frame entirely because of too much noise.}
      %
\item What happens when you increase the carrier frequency offset from
      0.001 to 0.05?%
      \footnote{{\em A:} the frame isn't detected because the carrier
                offset is too large for the synchronizer to correct.
                Try decreasing the number of subcarriers from 64 to 32
                and see what happens.}
\end{enumerate}

