% 
% MODULE : equalization
%

\section{equalization}
\label{module:equalization}
adaptive equalizers: LMS, RLS, blind...

This section describes the functionality of two digital linear adaptive
equalizers implemented...

\subsection{System Description}
Suppose a known transmitted symbol sequence
$\vec{d} = \{ d_0, d_1, \ldots ,d_{N-1}\}$
which passes through an unknown channel filter $\vec{h}_n$ of length $q$.
The received symbol at time $n$ is therefore
\[
    y(n) = \sum\limits_{k=0}^{q-1}{h_n(k)d(n-k)} + w(n)
\]
where $w(n)$ represents white Gauss noise.
The adaptive linear equalizer attempts to use a finite impulse response (FIR)
filter $\vec{w}$ of length $p$ to estimate the transmitted symbol, using only
the received signal vector $\vec{y}$ and the known data sequence $\vec{d}$,
viz
\[
    \hat{d}(n) = \vec{w}_n^T \vec{y}(n)
\]
where $\vec{y}_n = [ y(n), y(n-1),\ldots, y(n-p+1) ]^T$.
Several methods for estimating $\vec{w}$ are known in the literature, and
typically rely on iteratively adjusting $\vec{w}$ with each input though a
recursion algorithm.
In this section we provide a very brief overview of two prevalent adaptation
algorithms;
for a more in-depth discussion we refer the interested reader to
\cite{Proakis:2001,Haykin:2002}.

\subsection{Least Mean-Squares}
The least mean-squares (LMS) algorithm adapts the coefficients of the filter
estimate using a steepest descent (gradient) of the instantaneous {\it a priori}
error.
The filter estimate at time $n+1$ follows the following recursion
\begin{equation}
\label{eq:lms:weight_update}
\vec{w}_{n+1} = \vec{w}_{n} - \mu \vec{g}_n
\end{equation}
where $\mu$ is the iterative step size, and
$\vec{g}_n$ the normalized gradient vector, estimated from the error signal
and the coefficients vector at time $n$.

\subsection{Recursive Least-Squares}
The recursive least-squares (RLS) algorithm attempts to minimize the
time-average weighted square error of the filter output, viz
\begin{equation}
c(\vec{w}_n) = \sum\limits_{i=0}^{n}{ \lambda^{i-n} \left| d(i)-\hat{d}(i)\right|^2 }
\end{equation}
where the forgetting factor $0<\lambda\leq 1$ which itroduces exponential
weighting into past data, appropriate for time-varying channels.
The solution to minimizing the cost function $c(\vec{w}_n)$ is achieved by
setting its partial derivatives with respect to $\vec{w}_n$ equal to zero.
The solution at time $n$ involves inverting the weighted cross correlation
matrix for $\vec{y}_n$, a computationally complex task.
This step can be circumvented through the use of a recursive algorithm which
attempts to estimate the inverse using the {\it a priori} error from the
output of the filter.
The update equation is
\begin{equation}
\label{eq:rls:weight_update}
\vec{w}_{n+1} = \vec{w}_n + \Delta_{n}
\end{equation}
where the correction factor $\Delta_{n}$ depends on $\vec{y}_n$ and $\vec{w}_n$,
and involves several $p \times p$ matrix multiplications.
The RLS algorithm provides a solution which converges much faster than the LMS
algorithm, however with a significant increase in computational complexity and
memory requirements.

\subsection{Results}
Both the LMS and RLS equalizers were implemented in the C programming
language using a minimalistic interface wrapper.
The performance of the two equalizers were compared by generating a channel
with an impulse response representing a strong line-of-sight (LoS) component
followed by random echoes.
Each was trained on 512 iterations of a known QPSK-modulated training sequence with learning
rate parameters $\mu=0.999$ and $\lambda=0.999$ for the LMS and RLS algorithms,
respectively.
A small amount of noise was inject after the channel filter (roughly 40dB
below the signal level) to demonstrate the robustness of the algorithms.
The results of two simulations are shown in
figures~\ref{fig:module:equalization:eqlms} and \ref{fig:module:equalization:eqrls};
the first demonstrating a 10-tap equalizer applied to the response of a 6-tap
channel while the second demonstrates a 28-tap equalizer for a 12-tap
channel.

The passband power spectral densities (PSD) of the channel and the equalizer outputs
are depicted in figures~\ref{fig:module:equalization:eqlms:psd} and \ref{fig:module:equalization:eqrls:psd}.
While nearly flat, it is important to realize that these algorithms minimize
a cost function defined as the square of the {\it a priori} filter output error,
and do not necessarily force the PSD to zero.
In both scenarios, the RLS equalizer had a lower error after training, and converged to its
error minimum much faster.



%-------------------- FIGURE: EQUALIZER EXAMPLE 1 --------------------
\begin{figure}[ht]
\centering
\mbox{
  \subfigure[PSD] {
      \includegraphics[trim = 15mm 0mm 15mm 0mm, clip, width=6cm]{figures.gen/equalizer_cccf_psd}
      \label{fig:module:equalization:eqlms:psd}
    } \quad
  \subfigure[constellation] {
      \includegraphics[trim = 15mm 0mm 15mm 0mm, clip, width=6cm]{figures.gen/equalizer_cccf_const}
      \label{fig:module:equalization:eqlms:constellation}
    } \quad
}
\mbox{
  \subfigure[taps] {
      \includegraphics[trim = 15mm 0mm 15mm 0mm, clip, width=6cm]{figures.gen/equalizer_cccf_taps}
      \label{fig:module:equalization:eqlms:taps}
    } \quad
  \subfigure[mean-squared error] {
      \includegraphics[trim = 15mm 0mm 15mm 0mm, clip, width=6cm]{figures.gen/equalizer_cccf_mse}
      \label{fig:module:equalization:eqlms:mse}
    } \quad
}
% trim = left bottom right top
\caption{{\tt eqlms}, {\tt eqrls} demonstration:
10-tap equalizer for a 6-tap channel}
\label{fig:module:equalization:eqlms}
\end{figure}


%-------------------- FIGURE: EQUALIZER EXAMPLE 2 --------------------
\begin{figure}[ht]
\centering
\mbox{
  \subfigure[PSD] {
      %\includegraphics[trim = 15mm 0mm 15mm 0mm, clip, width=6cm]{figures/equalizer_example2_psd}
      \label{fig:module:equalization:eqrls:psd}
    } \quad
  \subfigure[constellation] {
      %\includegraphics[trim = 15mm 0mm 15mm 0mm, clip, width=6cm]{figures/equalizer_example2_constellation}
      \label{fig:module:equalization:eqrls:constellation}
    } \quad
}
\mbox{
  \subfigure[taps] {
      %\includegraphics[trim = 15mm 0mm 15mm 0mm, clip, width=6cm]{figures/equalizer_example2_taps}
      \label{fig:module:equalization:eqrls:taps}
    } \quad
  \subfigure[mean-squared error] {
      %\includegraphics[trim = 15mm 0mm 15mm 0mm, clip, width=6cm]{figures/equalizer_example2_error}
      \label{fig:module:equalization:eqrls:mse}
    } \quad
}
% trim = left bottom right top
\caption{{\tt eqrls} demonstration: 10-tap equalizer for a 6-tap channel}
\label{fig:module:equalization:eqrls}
\end{figure}


