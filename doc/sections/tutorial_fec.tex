% 
% TUTORIAL : fec
%

\newpage
\section{Tutorial: Forward Error Correction}
\label{tutorial:fec}

This tutorial will demonstrate computation at the byte level by
introducing the forward error-correction (FEC) coding module.
Please note that \liquid\ only provides some very basic FEC
capabilities including some Hamming block codes and repeat codes.
While these codes are very fast and enough to get started,
they are not very efficient and add a lot of redunancy without providing
a strong level of correcting capabilities.
\liquid\ will use the convolutional and Reed-Solomon codes described in
{\em libfec} \cite{libfec:web} if installed on your machine.
%While not a requirement...
% maybe in the future these can be imported into liquid-dsp...

\subsection{Problem Statement}
\label{tutorial:fec:problem}
Digital communications over a noisy channel can be unreliable,
resulting in errors at the receiver.
Forward error-correction (FEC) coding adds redundancy to the original
data message that allows for some errors to be corrected at the
receiver.
The error-correction capability of the code is dependent upon many
factors, but is usually improved by increasing the amount of redundancy
added to the message.
The drawback to adding a lot of redundancy is that the communications
rate is decreased as the link must be shared among the important data
information as well as the redundant bits.
The benefit, however, is that the receiver has a better chance of
correcting the errors without having to request a retransmission of the
message.
Volumes of research papers and books have been written about the
error-correction capabilities of certain FEC encoder/decoder pairs
(codecs) and their performance in a variety of environments.
While there is far too much information on the subject to discuss here,
it is important to note that \liquid\ implements a very small subset of
simple FEC codecs, including several Hamming and repeat codes.
If the {\em libfec} \cite{libfec:web} library is installed when \liquid\
is configured this list extends to convolutional and Reed-Solomon codes.

In this tutorial you will create a simple program that will generate a
message, encode it using a simple Hamming(7,4) code, corrupt the encoded
message by adding an error, and then try to correct the error with the
decoder.
The length of the message, the encoding scheme, and the number of errors
can be changed... xxx


\subsection{Setting up the Environment}
\label{tutorial:fec:environment}

Create a new file {\tt fec.c} and open it with your favorite editor.
Include the headers {\tt stdio.h} and {\tt liquid/liquid.h}
and add the {\tt int main()} definition
so that your program looks like this:
%
\input{tutorials/fec_init_tutorial.c.tex}
%
Compile and link the program using {\tt gcc}:
%
\begin{Verbatim}[fontsize=\small]
    $ gcc -Wall -o fec -lm -lc -lliquid fec.c
\end{Verbatim}
%
The flag ``{\tt -Wall}'' tells the compiler to print all warnings
(unused and uninitialized variables, etc.),
``{\tt -o fec}'' specifies the name of the output program is
``{\tt fec}'', and
``{\tt -lm -lc -lliquid}'' tells the linker to link the binary against
the math, standard C, and \liquid\ DSP libraries, respectively.
Notice that the above command invokes both the compiler and the linker
collectively.
%While it is usually preferred to build an intermediate object...
%
If the compiler did not give any errors, the output executable {\tt fec}
is created which can be run as
\begin{Verbatim}[fontsize=\small]
    $ ./fec
\end{Verbatim}
%
and should simply print ``{\tt done.}'' to the screen.
You are now ready to add functionality to your program.

We will now edit the file to set up the basic simulation by generating a
message signal and counting errors as a result of channel effects.
The error-correction capabilities will be added in the next section.
First set up the simulation parameters: for now the only parameter will
be the length of the input message, denoted by the variable {\tt n}
({\tt unsigned int}) representing the number of bytes.
Initialize {\tt n} to 8 to reflect an original message of 8 bytes.
Create another {\tt unsigned int} variable {\tt k} which will represent
the length of the encoded message.
This length is equal to the original ({\tt n}) with the additional
redundancy.
For now set {\tt k} equal to {\tt n} as we are not adding FEC coding
until the next section.
This implies that without any redundancy, the receiver cannot correct
for any errors.

Message data in \liquid\ are represented as arrays of type
{\tt unsigned char}.
Allocate space for the original, encoded, and decoded messages as
{\tt msg\_org[n]},
{\tt msg\_enc[k]}, and
{\tt msg\_dec[n]}, respectively.
Initialize the original data message as desired.
For example, the elements in {\tt msg\_org} can contain
{\tt 0,1,2,...,n-1}.
Copy the contents of {\tt msg\_org} to {\tt msg\_enc}.
This effectively is a placeholder for forward error-correction which
will be discussed in the next section. % [reword]
Corrupt one of the bits in {\tt msg\_enc}
(e.g. {\tt msg\_enc[0] \^= 0x01;} will flip the least-significant bit in
the first byte of the {\tt msg\_enc} array),
and copy the results to {\tt msg\_dec}.
Print the encoded and decoded messages to the screen to verify that they
are not equal.
Without any error-correction capabilities, the receiver should see a
message different than the original because of the corrupted bit.
Count the number of bit differences between the original and decoded
messages.
\liquid\ provides a convenient interface for doing this and can be
invoked as
%
\begin{Verbatim}[fontsize=\small]
    unsigned int num_bit_errors = count_bit_errors_array(msg_org,
                                                         msg_dec,
                                                         n);
\end{Verbatim}
%
Print this number to the screen.
Your program should look similar to this:
%
\input{tutorials/fec_basic_tutorial.c.tex}
%
Compile the program as before, creating the executable ``{\tt fec}.''
Running the program should produce an output similar to this:
%
\begin{Verbatim}[fontsize=\small]
    original message:  [  8]  00 01 02 03 04 05 06 07
    decoded message:   [  8]  01 01 02 03 04 05 06 07
    number of bit errors received:      1 /  64
\end{Verbatim}
%
Notice that the decoded message differs from the original and that the
number of received errors is nonzero.


%
% SUBSECTION : creating the encoder
%
\subsection{Creating the Encoder/Decoder}
\label{tutorial:fec:codec}
% talking points
%   * Hamming(7,4) is a weak code
%   * liquid abstracts from gritty details
%   * 

\subsection{Final Program}
\label{tutorial:fec:completed}

The final program is listed below,
and a copy of the source is located in the {\tt doc/tutorials/}
subdirectory.
%
\input{tutorials/fec_tutorial.c.tex}
%
The output should look like...
%
\begin{Verbatim}[fontsize=\small]
    ...
\end{Verbatim}

For a more detailed program, see {\tt examples/fec\_example.c} in the
main \liquid\ directory.
Section~\ref{module:fec} describes \liquid's FEC module in detail.
Additionally, the {\tt packetizer} object extends the simplicity of the
{\tt fec} object by adding a cyclic redundancy check and two layers of
forward error-correction and interleaving, all of which can be
reconfigured as desired.
See {\tt examples/packetizer\_example.c} for a detailed example program
on how to use the {\tt packetizer} object.

