% 
% MODULE : sequence
%

\section{sequence}
\label{module:sequence}
linear feedback shift registers, complementary codes, etc.

\subsection{{\tt bsequence}, generic binary sequence}
\label{module:sequence:bsequence}
...
This is particularly useful in wireless communications for correlating long
bit sequences in seeking frame preambles and packet headers.
%This is also useful in the physical layer description 
The {\tt bsequence} object internally stores its sequence of bits as an array
of bytes which handles shifting values even faster than the {\tt window}
family of objects.

\begin{description}
\item[{\tt bsequence\_create(num\_bits)}]
    creates a {\tt bsequence} object of length {\tt num\_bits}, filled
    initially with zeros.
\item[{\tt bsequence\_destroy()}]
    destroys the object, freeing all internally-allocated memory.
\item[{\tt bsequence\_clear()}]
    resets the sequence to all zeros.
\item[{\tt bsequence\_print()}]
    prints the contents of the sequence to the screen.
\item[{\tt bsequence\_push()}]
    pushes a bit into the back (right side) of a binary sequence, and in turn
    drops the left-most bit.
    Only the right-most (least-significant) bit of the input is regarded.
\item[{\tt bsequence\_circshift()}]
    circularly shifts a binary sequence left, pushing the left-most bit back
    into the right-most position.
\item[{\tt bsequence\_correlate()}]
    runs a binary correlation of two {\tt bsequence} objects, and returns XXX.
\item[{\tt bsequence\_add()}]
    computes the binary addition of two sequences
\item[{\tt bsequence\_mul()}]
    computes the binary multiplication of two sequences
\item[{\tt bsequence\_accumulate()}]
    accumulates the {\tt 1}s in a binary sequence
\item[{\tt bsequence\_get\_length()}]
    returns the length of the sequence (number of bits)
\item[{\tt bsequence\_index()}]
    returns the bit at a particular index of the sequence
\end{description}


\subsection{{\tt msequence}, $m$-sequence}
\label{module:sequence:msequence}
The {\tt msequence} object in \liquid\ is really just a linear feedback shift
register (LFSR).


\subsection{complementary codes}
\label{module:sequence:ccodes}


