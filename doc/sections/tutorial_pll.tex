% 
% TUTORIAL : pll
%

\newpage
\section{Tutorial: Phase-Locked Loop}
\label{tutorial:pll}
This tutorial demonstrates the functionality of a carrier phase-locked
loop and introduces the {\tt iirfilt} object.
%
You will need on your local machine:
\begin{itemize}
\item the \liquid\ DSP libraries built and installed (see
      Section~\ref{section:installation})
\item a text editor such as {\tt vim} \cite{vim:web}
\item a C compiler such as {\tt gcc} \cite{gcc:web}
\item a terminal
\end{itemize}
%
The problem statement and a brief theoretical description of
phase-locked loops is given in the next section.
A walk-through of the source code follows.

\subsection{Problem Statement}
\label{tutorial:pll:problem}
Wireless communications systems up-convert the data signal with
a high-frequency carrier before transmitting over the air.
This transmitted signal is orthogonal to other signals so long as
their bandwidths don't overlap and can be recovered at the receiver by
mixing it back down to baseband.
%This is accomplished by mixing the complex baseband signal...
%with a complex sinusoid, viz
%\[
%    s(t) = \Re\Bigl\{ m(t)\exp\{j\omega_ct\} \Bigr\}
%\]
Many digital communications systems modulate information in the phase of
the carrier requiring the receiver to demodulate the signal coherently
in order to recover the original data message.
In this regard the receiver must synchronize its carrier oscillator to
that of the transmitter.
To put it simply, the receiver must lock on to the phase of
transmitter's carrier.
One of the key advantages to performing signal processing in software is
that the radio can operate at complex baseband.
% TODO : expand on this
% (see Section~\ref{section:complex_baseband}).

In this simulation, the received signal is simply a complex sinusoid
with an unknown initial carrier phase and frequency.
The carrer holds no information-bearing symbols and is simply a tone
whose frequency and phase represent the residual mismatch between the
transmitter and receiver.
The received signal $x$ at time step $k$ can be described as
%
\begin{equation}
\label{eqn:tutoral:pll:x}
    x_k = \exp\bigl\{ j(\theta + k\omega) \bigr\}
\end{equation}
%
where $j \triangleq \sqrt{-1}$ and
$\theta$ and $\omega$ represent the unknown initial carrier phase and
frequency offsets, respectively.
The receiver generates a complex sinusoid with a phase $\phi_k$ as the
phase difference between $x_k$ and $y_k$ and can be computed as
%
\begin{equation}
\label{eqn:tutoral:pll:y}
    y_k = \exp\bigl\{j\phi_k\bigr\}
\end{equation}
%
%which tries to minimize the difference between the phase of the ...
The phase error at time step $k$ is expressed as
%
\begin{equation}
\label{eqn:tutoral:pll:dphi}
    \Delta\phi_k = \arg\bigl\{ x_k y_k^* \bigr\}
\end{equation}
%
where $(^*)$ denotes complex conjugation.%
\footnote{
    Those who are savvy with communications techniques will
    appreciate that we are dealing in complex baseband and can easily
    compute the phase error estimate simply as the argument of the
    product of $x_k$ and $y_k$.
    Conventional PLLs which have operated strictly in the real domain
    multiply only the real components of $x_k$ and $y_k$ for a phase
    error estimate, assume that the loop filter rejects the
    high-frequency component, and make the approximation
    $\Delta\phi \approx \sin(\Delta\phi) = \sin(\phi-\hat{\phi})$
    for small phase errors.}
The goal of the receiver is to control $\phi_k$
(the phase of the output signal $y$ at time $k$)
to lock onto the input phase of $x$,
hence the name ``phase-locked loop.''
If the phase of the output sample $y_k$ is behind that of the input
($\Delta\phi > 0$) then $\phi$ needs to be advanced appropriately for
the next time step.
Conversely, if the phase of $y_k$ is ahead of the phase of $x_k$
($\Delta\phi < 0$) then the receiver need to retard $\phi$.

Without going into a great amount of detail, this control is
accomplished using a special filter within the loop.
This filter, known as a ``loop filter,'' is designed to reject
high-frequency noise and is described with the transfer function $H(z)$.
Specifically $H(z)$ is a 2$^{nd}$-order integrating low-pass recursive
filter with
a natural frequency $\omega_n$,
a damping factor $\zeta$, and
a loop gain $K$.
The natural frequency is the resonant frequency of $H(z)$ and for all
practical purposes is the filter's bandwidth.
Increasing $\omega_n$ permits the loop to track to the input signal
faster (reduces lock time), but also increases the amount of noise
passed through the loop.
Decreasing $\omega_n$ reduces this noise but also increases the loop's
acquisition time.
The damping factor $\zeta$ controls the stability of the filter and is
typically set to a value near $1/\sqrt{2} \approx 0.707$.
The loop gain $K$ is typically very large
(on the order of $1000$ or so).
For more detailed information on loop filter design the interested
reader is referred to Section~\ref{module:nco:pll}.
% TODO : reference iirdes_pll_...() as well?

The estimated phase error $\Delta\phi_k$ is filtered using $H(z)$
resulting in an output phase estimate $\phi_{k+1}$
which is used for the subsequent output sample $y_{k+1}$ as
%
\begin{equation}
\label{eqn:tutoral:pll:y1}
    y_{k+1} = \exp\bigl\{ j\phi_{k+1} \bigr\}
\end{equation}
%
% ALGORITHM : phase-locked loop
\begin{algorithm}[H]
\caption{Phase-locked Loop Control}
\label{alg:tutorial:pll}
%\algsetup{linenosize=\footnotesize}
%\algsetup{linenodelimiter=:}
\algsetup{indent=2em}
\begin{algorithmic}[1]
\STATE $\vec{x} \leftarrow \{x_0,x_1,x_2,\ldots\}$    \COMMENT{input array}
\STATE $\hat{\phi}_0 \leftarrow 0$          \COMMENT{initial output phase}

\FOR{$k=0,\,1,\,2,\,\ldots$}
    \STATE $y_k \leftarrow \exp\bigl\{ j\hat{\phi}_k \bigr\}$ \COMMENT{compute output sample}
    \STATE $\Delta\phi_k \leftarrow \arg\bigl\{ x_k y_k^* \bigr\}$ \COMMENT{phase detector}
    \STATE $\hat{\phi}_{k+1} \leftarrow \text{filter}(\Delta\phi_k)$ \COMMENT{update output phase estimate}
\ENDFOR
\end{algorithmic}
\end{algorithm}
%
A summary of the algorithm is given in Algorithm~\ref{alg:tutorial:pll}.
In the next section we will create a simple C program to simulate a
phase-locked loop with \liquid.


\subsection{Setting up the Environment}
\label{tutorial:pll:environment}

For this tutorial and others, I assume that you are using the GNU
compiler collection ({\tt gcc}) for compiling source and linking objects
\cite{gcc:web},
and that you have a familiarity with the C (or C++) programming
language.
Create a new file {\tt pll.c} and open it with your favorite editor.
Include the headers {\tt stdio.h}, {\tt complex.h}, {\tt math.h}, and
{\tt liquid/liquid.h} and add the {\tt int main()} definition
so that your program looks like this:
%
\input{tutorials/pll_init_tutorial.c.tex}
%
Compile and link the program using {\tt gcc}:
%
\begin{Verbatim}[fontsize=\small]
    $ gcc -Wall -o pll -lm -lc -lliquid pll.c
\end{Verbatim}
%
The flag ``{\tt -Wall}'' tells the compiler to print all warnings
(unused and uninitialized variables, etc.),
``{\tt -o pll}'' specifies the name of the output program is
``{\tt pll}'', and
``{\tt -lm -lc -lliquid}'' tells the linker to link the binary against
the math, standard C, and \liquid\ DSP libraries, respectively.
Notice that the above command invokes both the compiler and the linker
collectively.
%While it is usually preferred to build an intermediate object...
%
If the compiler did not give any errors, the output executable {\tt pll}
is created which can be run as
\begin{Verbatim}[fontsize=\small]
    $ ./pll
\end{Verbatim}
%
and should simply print ``{\tt done.}'' to the screen.
You are now ready to add functionality to your program.

We will now edit the file to set up the basic simulation but without
controlling the phase of the output sinusoid.
As such the output won't track to the input resulting in a significant
amount of phase error.
This simulation will operate one sample at a time and is organized into
three sections.
First, set up the simulation parameters: the initial phase and frequency
offsets ({\tt float}),
and number of samples to run ({\tt unsigned int}).
Next, initialize the complex input and output variables
({\tt x} and {\tt y}) to zero,
as well as the state of the phase error ({\tt phase\_error})
and output phase ({\tt phi\_hat}) estimates.
Finally, set up the computational loop which generates the input and
output samples, computes the phase error between them, and then prints
the results to the screen.
%
Edit {\tt pll.c} to set up the basic simulation:
%
\input{tutorials/pll_basic_tutorial.c.tex}
%
% DISCECTION:
The variables {\tt x} and {\tt y} are of type {\tt float complex} which
contains both real and imaginary components of type {\tt float}.
%\footnote{
%    If, for some reason, you prefer C++ over C you could use the type
%    {\tt std::complex<float>} instead.
%    See Section~\ref{xxx} for details.}
The function {\tt cexpf()} computes the complex exponential of its
argument which for a purely imaginary input $j\alpha$ is simply
$e^{j\alpha} = \cos\alpha + j\sin\alpha$.
% TODO : finish explanation

%
Compile and run the program as before.
The program should now output something like this:
%
\begin{Verbatim}[fontsize=\small]
      0 : phase =   0.00000000, error =   0.80000001
      1 : phase =   0.00000000, error =   0.81000000
      2 : phase =   0.00000000, error =   0.81999999
      3 : phase =   0.00000000, error =   0.82999998
      4 : phase =   0.00000000, error =   0.84000003
            ...
     35 : phase =   0.00000000, error =   1.14999998
     36 : phase =   0.00000000, error =   1.15999997
     37 : phase =   0.00000000, error =   1.17000008
     38 : phase =   0.00000000, error =   1.18000007
     39 : phase =   0.00000000, error =   1.19000006
    done.
\end{Verbatim}
%
Notice that because we aren't controlling the output phase yet
the error increases with the input phase.
In the next section we will design the loop filter to adjust the output
phase to lock onto the input signal given the phase error.

\subsection{Designing the Loop Filter}
\label{tutorial:pll:design}

Our program so far has not used any of the \liquid\ DSP libraries for
computation and has only relied on the standard C libraries for dealing
with complex math operations.
In this section we will introduce \liquid's {\tt iirfilt\_rrrf} object
to realize a recursive (infinite impulse response) filter with real
inputs, coefficients, and outputs.
Additionally we will use the function {\tt iirdes\_pll\_active\_lag()}
to design the coefficients for the PLL's filter,
specifically an ``active lag'' design.
While the explanation in this section is fairly long, relax!
We will only need to add about 15 lines of code to our program.
If you are eager to edit your program you may skip to
Section~\ref{tutorial:pll:completed}.

Digital representations of infinite impulse response (IIR) filters have
two sets of coefficients: feedback and feedforward.
In the digital domain the transfer function is a ratio of the
polynomials in $z^{-1}$ where the feedforward coefficients
$\vec{b} = \{b_0, b_1, b_2, \ldots, b_{N-1}\}$
are in the numerator and the feedback coefficients
$\vec{a} = \{a_0, a_1, a_2, \ldots, a_{M-1}\}$
are in the denominator.
Specifically, the transfer function is
%
\begin{equation}
    H(z) =
        \frac{
            b_0 + b_1 z^{-1} + b_2 z^{-2} + \ldots + b_{N-1}z^{-(N-1)}
        }{
            a_0 + a_1 z^{-1} + a_2 z^{-2} + \ldots + a_{M-1}z^{-(M-1)}
        }
\end{equation}
%
This transfer function means that the output of the filter is the linear
combination of the $N$ previous filter inputs
%($\vec{x} = \{x_0, x_1, x_2, \ldots, x_{N-1}\}$)
($\vec{x}$)
and $M-1$ previous filter outputs
%($\vec{y} = \{y_0, y_1, y_2, \ldots, y_{M-1}\}$),
($\vec{y}$),
viz
%
\begin{eqnarray}
%    y_{k} =
%        \frac{1}{a_0}
%        \Bigl(
%            b_0 x_k &+& b_1 x_{k-1} + \cdots + b_{N-1} x_{k-N}\\
%                    &-& a_1 y_{k-1} - \cdots - a_{M-1} y_{k-M}
%        \Bigr)
    y[k] =
        \frac{1}{a_0}
        \Bigl(
            b_0 x[k] &+& b_1 x[k-1] + \cdots + b_{N-1} x[k-N]\\
                     &-& a_1 y[k-1] - \cdots - a_{M-1} y[k-M]
        \Bigr)
\end{eqnarray}
%
Typically the number of feedback and feedforward coefficients are equal
($M=N$), and the coefficients themselves are normalized so that $a_0=1$.

\liquid\ implements IIR filters with the {\tt iirfilt\_xxxt} family of
objects where ``{\tt xxxt}'' denotes the type definition
(see Section~\ref{section:data_structures} for details).
In our example we will be using the {\tt iirfilt\_rrrf} object which
indicates that this is an IIR filter with real inputs, outputs, and
coefficients with precision of type {\tt float}.
The IIR filter objects in \liquid\ maintain their state
internally, storing the previous inputs and outputs in its internal
buffers.
Nearly every object in \liquid\ (filter or otherwise) has at least four
basic methods:
{\tt create()},
{\tt print()},
{\tt execute()}, and
{\tt destroy()}.
For our program we will need to create the filter object by passing to
it a vector of each the feedback and feedforward coefficients.
The infinite impulse response (IIR) filter we are designing is of order
two which means that $\vec{a}$ and $\vec{b}$ have three coefficients
each.

Generating the loop filter coefficients is fairly straightforward.
As stated before, the loop filter has parameters for
natural frequency $\omega_n$,
damping factor $\zeta$, and
loop gain $K$.
Furthermore the filter is 2$^{nd}$-order which means that it has three
coefficients each for $\vec{a}$ and $\vec{b}$.
\liquid\ provides a method for computing such a filter with the
{\tt iirdes\_pll\_active\_lag()} function
which accepts $\omega_n$, $\zeta$, and $K$ as inputs and generates the
coefficients in two output arrays.
The coefficients can be computed as follows:
%
\begin{Verbatim}[fontsize=\small]
    float wn = 0.1f;        // pll bandwidth
    float zeta = 0.707f;    // pll damping factor
    float K = 1000.0f;      // pll loop gain
    float b[3];             // feedforward coefficients array
    float a[3];             // feedback coefficients array
    iirdes_pll_active_lag(wn, zeta, K, b, a);
\end{Verbatim}
%
The life cycle of the IIR filter can be summarized as follows
%
\begin{Verbatim}[fontsize=\small]
    iirfilt_rrrf loopfilter = iirfilt_rrrf_create(b,3,a,3);
    float sample_in = 0.0f;
    float sample_out;
    {
        // repeat as necessary
        iirfilt_rrrf_execute(loopfilter, sample_in, &sample_out);
    }
    iirfilt_rrrf_destroy(loopfilter);
\end{Verbatim}
%
noting that the {\tt execute()} method can be repeated as many times as
necessary before the object is destroyed.

Using the code snippets above, modify your program to include the loop
filter to adjust the output signal's phase.
The input to the filter will be the {\tt phase\_error} variable, and its
output will be {\tt phi\_hat}.
Don't forget to destroy your filter object once the loop has finished
running.

\subsection{Final Program}
\label{tutorial:pll:completed}

The final program is listed below,
and a copy of the source is located in the {\tt doc/tutorials/}
subdirectory.
%
\input{tutorials/pll_tutorial.c.tex}
%
Compile the program as before, creating the executable ``{\tt pll}.''
Running the program should produce an output similar to this:
\begin{Verbatim}[fontsize=\small]
    iir filter [normal]:
      b :  0.32277358  0.07999840 -0.24277516
      a :  1.00000000 -1.99995995  0.99996001
      0 : phase =   0.25821885, error =   0.80000001
      1 : phase =   0.75852644, error =   0.55178112
      2 : phase =   1.12857747, error =   0.06147351
      3 : phase =   1.27319980, error =  -0.29857749
      4 : phase =   1.23918116, error =  -0.43319979
            ...
     35 : phase =   1.15999877, error =   0.00000751
     36 : phase =   1.17000139, error =   0.00000122
     37 : phase =   1.18000150, error =  -0.00000131
     38 : phase =   1.19000030, error =  -0.00000140
     39 : phase =   1.19999886, error =  -0.00000024
    done.
\end{Verbatim}
%
Notice that the phase error at the end of the output is very small.
The initial error (at $k=0$) is 0.8 which is the value of the
{\tt phase\_offset} parameter ot the beginning of the program.
Notice also that the difference in phase of the last several samples
(i.e. the difference between the phase at steps {\tt 38} and {\tt 39})
is approximately 0.1 which is the initial frequency offset that was
given in the beginning.
Play around with the input parameters, particularly the frequency offset
and the phase-locked loop bandwidth.
Increasing the PLL bandwidth ({\tt wn}) should reduce the resulting
phase error more quickly.
The downside of having a PLL with a large bandwidth is that when the
input signal has been corrupted by noise then the phase error estimate
is also noisy.
In this tutorial no noise term was introduced. % TODO : explain more!

