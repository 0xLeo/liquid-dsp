% 
% MODULE : experimental
%

\newpage
\section{experimental}
\label{module:experimental}
This section describes the experimental module.
This is a special module which is a placeholder for experimental objects
and interfaces which are not ready for release, have not been fully
tested, but might still be useful to certain developers.
By default its objects are not built unless the
{\tt --enable-experimental} flag is given to the {\tt configure} script.

\subsection{{\tt fbasc} (filterbank audio synthesizer codec)}
\label{module:experimental:fbasc}

The fbasc audio codec implements an AAC-like compression algorithm, using the
modified discrete cosine transform as a loss-less channelizer.  The resulting
channelized data are then quantized based on their spectral energy levels and
then packed into a frame which the decoder can then interpret. The result is a
lossy encoder (as a result of quantization) whose compression/quality levels
can be easily varied.

%                +------+ -->           -->  +-------+
%                |      | -->           -->  |       |
%   original     |      | -->  M-band   -->  |       |     reconstructed
%     time   ->  | MDCT |  .  quantizer  .   | iMDCT | ->      time
%    series      |      |  .             .   |       |        series
%                |      |  .             .   |       |
%                +------+ -->           -->  +-------+

Specifically, fbasc uses sub-band coding to allocate quantization bits to each
channel in order to minimize distortion of the reconstructed signal. Sub-bands
with higher variance (signal 'energy') are assigned more bits.  This is the
heart of the codec, which exploits several components typical of audio signals
and aspects of human hearing and perception:
\begin{enumerate}
\item The majority of audio signals (including music and voice) have a
      strong time-frequency localization; that is, they only occupy a small
      fraction of audible frequencies for a short duration.  This is
      particularly true for voiced signals (e.g. vowel sounds).
\item The human ear (and brain) tends to be quite forgiving of spectral
      compression and often cannot easily distinguish between neighboring
      frequency components.
\end{enumerate}

There are several benefits to using fbasc over other compression algorithms
such as CVSD (see src/audio/readme.cvsd.txt) and auto-regressive models, the
main being that the algorithm is theoretically lossless (i.e. perfect
reconstruction) as the bit rate increases.  As a result, the codec is limited
only by the quantization noise on each channel.

Here are some useful definitions, as used in the fbasc code:
%                   __________  __________
%                  /   MDCT   \/          \
%                 /   window  /\          ...
%            ____/       ____/  \____                                time
%    frame:  [----s0----][----s1----][----s2----][----s3----] ... --->
%                        |          |
%                      ->|          |<- symbol (length = M samples)

\begin{description}
\item[MDCT]
the modified discrete cosine transform is a lapped discrete cosine
transform which uses a special windowing function to ensure perfect
reconstruction on its inverse. The transform operates on $2M$ time-domain
samples (overlapped by $M$) to produce $M$ frequency-domain samples.
Conversely, the inverse MDCT accepts $M$ frequency-domain samples and
produces $2M$ time-domain samples which are windowed and then overlapped to
reconstruct the original signal.  For convenience, we may refer to $M$
time-domain samples as a 'symbol.'

\item[symbol]
one block of $M$ time-domain samples upon which the MDCT operates.
 
\item[channel]
one of the $M$ frequency-domain components as a result of applying the
MDCT.  This is somewhat equivalent to a discrete Fourier transform 'bin.'
Note than $M$ is equal to the number of channels in analysis.

\item[frame]
a set of MDCT symbols upon which the fbasc codec runs its analysis.
Because the codec uses time-frequency localization for its encoding, it is
necessary for the codec to gain enough statistical information about the
original signal without losing temporal stationarity. The codec typically
operates on several symbols, however, the exact number depends on the
application.
\end{description}

\subsubsection{Interface}
\begin{description}
\item[{\tt fbasc\_create()}]
        creates an fbasc encoder/decoder object, allocating memory as
        necessary, and computing internal parameters appropriately.
\item[{\tt fbasc\_destroy()}]
        destroys an fbasc encoder/decoder object, freeing internally-allocated
        memory.
\item[{\tt fbasc\_encode()}]
        encode a frame of data, storing the header and frame data separately.
        This separation allows the user to use different forward
        error-correction codes (if desired) to protect the header differently
        than the rest of the frame.  It is important to keep the two together,
        however, as the header is a description of how to decode the frame.
\item[{\tt fbasc\_decode()}]
        decodes a frame of data, generating the reconstructed time series.
\end{description}

\subsubsection{Useful properties}
\begin{itemize}
\item Because of the nature of the MDCT, frames will overlap by $M$ samples
      (one symbol).  This introduces a reconstruction delay of $M$ samples,
      noticeable at the decoder.
\end{itemize}

%The header contains the following data:
%    id      name                # bytes
%    --      ----------          ---------
%    fid     frame id            2
%    g0      nominal gain        1
%    bk      bit allocation      num_channels / 2
%    gk      gain allocation     num_channels / 2
%    --      ----------          ---------
%            total:              num_channels + 3
%
%Miscellaneous information
%
%Example:
%
%create() parameters:
%    num channels    =   64  (samples/symbol)
%    samples/frame   =   512
%    bytes/frame     =   256
%    ---------------------------
%derived values:
%    symbols/frame   =   [samples/frame] / [samples/symbol]  =   8
%    bytes/symbol    =   [bytes/frame]   / [symbols/frame]   =   32
%
%Each symbol must be encoded with an even number of bytes.

%
% gport
%
\subsection{gport}
\label{module:buffer:gport}
The {\tt gport} object implements a generic port to share data between
asynchronous threads.
The port itself is really just a circular (ring) buffer containing a
mutually-exclusive locking mechanism to allow processes running on independent
threads to access its data.
Because no other modules rely on the {\tt gport} object and because it
requires the pthread library, it is likely to be removed from \liquid\ in the
near future and likely put into another library, e.g. {\em liquid-threads}.

There are two ways to access the data in the {\tt gport} object: direct memory
access and indirect (copied) memory acess, each with distinct advantages and
distadvantages.
Regardless of which interface you use, the model is equivalent:
a buffer of data (initially empty) is created.
The {\it producer} is the method in charge of writing to the buffer (or
``producing'' the data).
The {\it consumer} is the method in charge of reading the data from the buffer
(or ``consuming'' it).
The producer and consumer methods can exist on completely separate threads,
and do not need to be externally synchronized.
The {\tt gport} object synchronizes the data between the ports.

\subsubsection{Direct Memory Access}
Using {\tt gport} via direct memory access is a multi-step process, equivalent
for both the producer and consumer threads.
For the sake of simplicity, we will describe the process for writing data to
the port on the producer side; the consumer process is identical.
%
\begin{enumerate}
\item the producer requests a lock on the port of a certain number of samples.
\item once the request is serviced, the port returns a pointer to an array of
      data allocated internally by the port itself.
\item the producer writes its data at this location, not exceeding the
      original number of samples requested.
\item the producer then unlocks the port, indicating how many samples were
      actually written to the buffer.
      This allows the consumer thread to read data from the buffer.
\item this process is repeated as necessary.
\end{enumerate}
%
Listed below is a minimal example demonstrating the direct memory access
method for the {\tt gport} object.
%
\input{listings/gport.direct.example.c.tex}
%

\subsubsection{Indirect/Copied Memory Access}
Indirect (or ``copied'') memory access appears similar...

\input{listings/gport.indirect.example.c.tex}

\subsubsection{Key differences between memory access modes}
While the direct memory access method provides a simpler interface--in the
sense that no external buffers are required--the user must take care in not
writing outside the bounds of the memory requested.
That is, if 256 samples are locked, only 256 values are available.
Writing more data will produce unexpected results, and could likely result in
a segmentation fault.
Furthermore, the buffer must wait until the entire requested block is
available before returning.
This could potentially increase the amount of time that each process is
waiting on the port.
Additionally, if one requests too many samples on both the producer and
consumer sides, the port could wait forever.
For example, assume one initially creates a {\tt gport} with 100 elements and
the producer initially writes 30 samples.
Immediately following, the consumer requests a lock for 100 samples which
isn't serviced because only 30 are available.
Following that, the producer requests a lock for 100 samples which isn't
serviced because only 70 are available.
This is a deadlock condition where both threads are waiting for data, and
neither request will be serviced.
The solution to this problem is actually fairly simple; the port should be
initially created as the sum of maximum size of the producer's and consumer's
requests.
That is, if the producer will at most ever request a lock on 50 samples and
the consumer will at most request a lock of 70 samples, then the port should
be initially created with a buffer size of 120.
This guarantees that the deadlock condition will never occur.

Alternatively one may use the indirect memory access method which guarantees
that the deadlock condition will never occur, even if the buffer size is 1 and
the producer writes 1000 samples while the consumer reads 1000.
This is because both the internal producer and consumer methods will write
the data as it becomes available, and do not have to wait internally until an
entire block of the requested size is ready.
This is the benefit of using the indirect memory access interface of the
{\tt gport} object.
Indirect memory access, however, requires the use of memory allocated
externally to the port.

It is important to stay consistent with the memory access mode used within a
thread, however mixed memory access modes can be used between threads on the
same port.
For example, the producer thread may use the direct memory access mode while
the consumer uses the indrect memory acess mode.

\subsubsection{Interface}
\label{module:buffer:gport:interface}

\begin{description}
\item[{\tt gport\_create()}]
    creates a new {\tt gport} object with an internal buffer of a certain
    length.
\item[{\tt gport\_destroy()}]
    destroys a {\tt gport} object, signaling {\it end of message} to any
    connected ports.
    %calling {\tt gport\_signal\_eom()}
\item[{\tt gport\_producer\_lock()}]
    locks a requested number of samples for producing, returning a {\tt void}
    pointer to the locked data array directly.
    Invoking this method can be thought of as asking the port to allocate a
    certain number of samples for writing.
    Special care must be taken by the user not to write more elements to the
    buffer than were requested.
    This function is a blocking call and waits for the data to become
    available or an {\it end of message} signal to be received.
    The data are locked until {\tt gport\_producer\_unlock()} is invoked.
    The number of unlocked samples does not have to match but cannot exceed
    those which are locked.
\item[{\tt gport\_producer\_unlock()}]
    unlocks a requested number of samples from the port.
    Use in conjunction with {\tt gport\_producer\_lock()}.
    Invoking this method can be thought of as telling the port ``I have
    written $n$ samples to the buffer you gave me earlier; release them to the
    consumer for reading.''
    The number of samples written to the port cannot exceed the initial
    request (e.g. if you request a lock for 100 samples, you should never try
    to unlock more than 100).
    There is no internal error-checking to ensure this.
    Failure to comply could result in over-writing data internally, and
    corrupt the consumer side.
\item[{\tt gport\_produce()}]
    produces $n$ samples to the port from an external buffer.
    This method is a blocking call and waits for the requested data to become
    available or an {\it end of message} signal to be received.
\item[{\tt gport\_produce\_available()}]
    operates just like {\tt gport\_produce()} except will write as many
    samples as are available when the function is called.
    Invoking this method is like telling the buffer ``I have $n$ samples, so
    write as many as you can right now.''
    It will always wait for at least one sample to become available and blocks
    until this condition is met.
\item[{\tt gport\_consumer\_lock()}]
    locks a requested number of samples for consuming, returning a {\tt void}
    pointer to the locked data array directly.
    Invoking this method can be thought of as asking the port to wait for a
    certain number of samples to be read.
    Special care must be taken by the user not to read more elements to the
    buffer than were requested.
    This function is a blocking call and waits for enough samples to become
    available or an {\it end of message} signal to be received.
    The data will be locked until {\tt gport\_consumer\_unlock()} is invoked.
    The number of unlocked samples does not have to match but cannot exceed
    those which are locked.
\item[{\tt gport\_consumer\_unlock()}]
    unlocks a requested number of samples from the port.
    Use in conjunction with {\tt gport\_consumer\_lock()}.
    Invoking this method can be though of as telling the port ``I have read
    $n$ samples from the buffer you gave me earlier; release them to the
    producer for writing.''
    The number of samples read from the port cannot exceed the initial
    request (e.g. if you request a lock for 100 samples, you should never try
    to unlock more than 100).
\item[{\tt gport\_consume()}]
    consumes $n$ samples from the port and writes to an external buffer.
    This method is a blocking call and waits for the requested data to become
    available or an {\it end of message} signal to be received.
\item[{\tt gport\_consume\_available()}]
    operates just like {\tt gport\_consume()} except will read as many samples
    as are available when the function is called.
    Invoking this method is like telling the buffer ``I have a buffer of $n$
    samples, so write to it as many as you can right now.''
    It will always wait for at least one sample to become available and blocks
    until this condition is met.
\item[{\tt gport\_signal\_eom()}]
    signals {\it end of message} to any connected {\tt gport}.
    This tells the port to stop waiting for data (on both the producer and
    consumer side) and return.
    This method prevents lock conditions where, e.g., the producer is waiting
    for several samples to become available, but the consumer has finished its
    process.
    This method is normally invoked only during {\tt gport\_destroy()}.
\item[{\tt gport\_clear\_eom()}]
    ({\it untested})
    clears the {\it end of message} signal.
\end{description}


\subsubsection{Problem areas}
When using the direct memory access method, the size of the data request
during lock is limited by the size of the port.
[[race/lock conditions?]]


%
% SUBMODULE : filter
%

% dds
\subsection{{\tt dds} (direct digital synthesizer)}

% qmfb
\subsection{{\tt qmfb} (quadrature mirror filter bank)}

