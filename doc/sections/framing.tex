% 
% MODULE : framing
%

\section{framing}
\label{module:framing}
The framing module contains objects and methods for packaging data into
manageable frames and packets.
For convention, \liquid\ refers to a ``packet'' as a group of binary
data bytes that need to be communicated over a wireless link.
A ``frame'' is a representation of the data once it has been properly
partitioned, encoded, and modulated before transmitting over the air.
Included in this module are the {\tt packetizer}, 
{\tt frame[gen|sync]64}, and {\tt flexframe[gen|sync]} objects which
greatly simplify over-the-air digital communication of data.

\subsection{{\tt packetizer}}
\label{module:framing:packetizer}

The liquid packetizer is a structure for abstracting multi-level forward
error-correction from the user.
The packetizer accepts a buffer of uncoded data bytes and adds a 32-bit
cyclic redundancy check (crc) before applying two levels of forward error-
correction and bit-level interleaving.  The user may choose any two 
supported FEC schemes (including none) and the packetizer object will
handle buffering and data management internally, providing a truly abstract
interface.  The same is true for the packet decoder which accepts an array
of possibly corrupt data and attempts to recover the original message using
the FEC schemes provided.  The packet decoder returns the validity of the
resulting CRC as well as its best effort of decoding the message.

The packetizer also allows for re-structuring if the user wishes to change
error-correction schemes or data lengths.  This is accomplished with the
{\tt packetier\_recreate()} method.

Here is a minimal example demonstrating the packetizer's most basic
functionality:
\input{listings/packetizer.example.c.tex}

See also: fec module, {\tt examples/packetizer\_example.c}

\subsection{{\tt frame64}, {\tt flexframe} (basic framing objects)}
\label{module:framing:frame}
The {\tt framegen64} and {\tt framesync64} objects implement a basic framing
structure for communicating packetized data over the air.

The {\tt flexframegen} and {\tt flexframesync} objects are similar to their
{\tt frame[gen|sync]64} counterparts, however extend functionality to include
a number of options in structuring the frame.

Both frames consist of six basic parts:

%\begin{tabular*}{0.95\textwidth}{l@{\extracolsep{\fill}}lll}
%\toprule
%{\it name}      & {\it \# symbols}  & {\it \# bytes}& {\it description}  \\
%\otoprule
%ramp up         & 64                & -             & BPSK ramp up sequence \\
%phasing pattern & 64                & -             & BSPK preamble phasing pattern \\
%p/n sequence    & 64                & -             & BPSK p/n synchronization sequence \\
%header          & 256               & 32            & QPSK, $r=1/2$-coded header \\
%payload         & 512               & 64            & QPSK, $r=1/2$-coded payload \\
%ramp down.      & 64                & -             & ramp down sequence \\
%\bottomrule
%\end{tabular*}

% 
% FIGURE: framing:structure
%
\begin{figure}
\centering
  \includegraphics[trim = 0mm 0mm 0mm 0mm, clip, width=16cm]{figures.pgf/framing_structure}
\caption{framing structure}
\label{fig:module:framing:structure}
\end{figure}


\begin{description}
\item[{\sf ramp/up}]
    gracefully increases the output signal level to avoid ``key clicking'' and
    reduce spectral side-lobes in the transmitted signal.
    Furthermore, it allows the receiver's automatic gain control unit to
    lock on to the incoming signal, preventing sharp transitions in its
    output.
\item[{\sf phasing pattern}]
    is BPSK pattern which flips phase for each transmitted symbol
    ({\tt~+1,-1,+1,-1,$\ldots$}).
    This sequence serves several purposes but primarily to help the receiver's
    symbol synchronization circuit to lock onto the proper timing phase.
    [This works] because the phasing pattern maximizes the number of symbol
    transitions [reword].
\item[{\sf p/n sequence}]
    is an $m$-sequence (see section~\ref{module:sequence}) exhibiting good
    auto- and cross-correlation properties.
    %The binary sequence is modulated using BPSK so that 
    This sequence aligns the frame synchronizers to the remainder of the
    frame, telling them when to start receiving and decoding the frame header,
    as well as if the phase of the received signal needs to be reversed.
    At this point, the receiver's AGC, carrier PLL, and timing PLL should all
    have locked.
    The p/n sequence is of length 64 for both the {\tt~frame64} and
    {\tt~flexframe} structures (63-bit $m$-sequence with additional padded
    bit).
\item[{\sf header}]
    is a fixed-length data sequence which contains a small amount of
    information about the rest of the frame.
    The headers for the {\tt~frame64} and {\tt~flexframe} structures are
    vastly different and are described independently.
\item[{\sf payload}]
    is the meat of the frame, containing the raw data to be transferred across
    the link.
    For the {\tt~frame64} structure, the payload is fixed at 64 bytes (hence
    its moniker), encoded using the Hamming~(7,4) code
    (section~\ref{module:fec}), and modulated using QPSK.
    The {\tt~flexframe} structure has a variable length payload and can be
    modulated using whatever schemes the user desires, however forward
    error-correction is executed externally.
    In both cases the synchronizer object invokes the callback upon receiving
    the payload.
\item[{\sf ramp/down}]
    gracefully decreases the output signal level as per ramp/up.
\end{description}

NOTE: while the {\tt flexframegen} and {\tt flexframesync} objects are
intended to be used in conjunction with one another, the output of
{\tt flexframegen} requires matched-filtering interpolation before the
{\tt~flexframesync} object can recover the data.


