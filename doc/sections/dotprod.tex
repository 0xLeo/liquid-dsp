% 
% MODULE : dotprod (vector dot product)
%

\newpage
\section{dotprod}
\label{module:dotprod}
Implements a vector dot product between two equally-sized vectors
$\vec{x} = \left[x_0,x_1,\ldots,x_{N-1}\right]^T$ and
$\vec{v} = \left[v_0,v_1,\ldots,v_{N-1}\right]^T$
\[
    \vec{x} \cdot \vec{v}   =
    \vec{x}^T \vec{v}       =
    \sum_{k=0}^{N-1}{ x(k) v(k) }
\]
A number of other modules rely on {\tt dotprod}, such as filtering and
equalization.

\subsection{Specific machine architectures}
\label{module:dotprod:arch}
The vector dot product has a complexity of $\ord(N)$ multiply-and-accumulate
operations.
Because of its prevalence in multimedia applications, a considerable amount of
research has been put into computing the vector dot product as efficiently as
possible.
Software-defined radio is no exception as basic profiling will likely
demonstrate that a considerable portion of the processor is spent computing
it.
Certain machine architectures have specific instructions for computing vector
dot products, particularly those which use a single instruction for
multiple data (SIMD) such as mmx, sse, AltiVec, etc.

\subsection{Usage}
\label{module:dotprod:usage}
There are effectively two ways to use the {\tt dotprod} module.
In the first and most general case, a vector dot product is computed on two
input vectors $\vec{x}$ and $\vec{v}$ whose values are not known
{\it a priori}.

In the second case, a {\tt dotprod} object is created around vector $\vec{v}$
which does not change (or rarely changes) throughout its life cycle.
This is the more convenient method for filtering objects which don't usually
have time-dependent coefficients.
Listed below is a simple interface example to the {\tt dotprod} module
object:
%
\input{listings/dotprod_rrrf.example.c.tex}
%
In both cases the {\tt dotprod} can be easily integrated with the
{\tt window} object (Section~\ref{module:buffer:window})
for managing input data and alignment.
There are three types of dot product objects and are listed in
Table~\ref{tab:dotprod:objects}.

% ------------ TABLE: DOTPROD OBJECT TYPES ------------
\begin{table*}
\caption{{\tt dotprod} object types}
\label{tab:dotprod:objects}
\centering
{\small
\begin{tabular*}{0.75\textwidth}{l@{\extracolsep{\fill}}lll}
\toprule
{\it precision} &
{\it input/output} &
{\it coefficients} &
{\it interface}\\\otoprule
%
{\it float} & real      & real      & {\tt dotprod\_rrrf} \\
{\it float} & complex   & complex   & {\tt dotprod\_cccf} \\
{\it float} & complex   & real      & {\tt dotprod\_crcf} \\\bottomrule
\end{tabular*}
}
\end{table*}%
% ------------------------

Here is a brief description of the {\tt dotprod} object interfaces.
While the types are described using the {\tt dotprod\_rrrf} object, the
same holds true for all other types.
\begin{description}
\item[{\tt dotprod\_run(h,x,n,y)}]
    executes a vector dot product between two vectors $\vec{h}$ and
    $\vec{x}$, each of length $n$ and stores the result in the output
    $y$.
    This is not a structured method and does not require creating a
    {\tt dotprod} object, however does not take advantage of SIMD
    instructions if available.
    Rather than speed, its intent is to provide a simple interface to
    demonstrate functional correctness.
\item[{\tt dotprod\_rrrf\_create(v,n)}]
    creates a {\tt dotprod} object with coefficients $\vec{v}$ of length
    $n$.
    %Certain machines can make use of special SIMD instructions, and
    %therefore the specific types of dot product
\item[{\tt dotprod\_rrrf\_recreate(q,v,n)}]
    recreates a {\tt dotprod} object with a new set of coefficients
    $\vec{v}$ with a (possibly) different length $n$.
\item[{\tt dotprod\_rrrf\_destroy(q)}]
    destroys a {\tt dotprod} object, freeing all internally-allocated
    memory.
\item[{\tt dotprod\_rrrf\_print(q)}]
    prints the object internals to the screen.
\item[{\tt dotprod\_rrrf\_execute(q,v,y)}]
    executes a dot product with an input vector $\vec{v}$ and stores the
    result in $y$.
\end{description}
