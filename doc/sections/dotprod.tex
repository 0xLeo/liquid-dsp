% 
% MODULE : dotprod (vector dot product)
%

\section{dotprod}
\label{module:dotprod}
Implements a vector dot product between two equally-sized vectors
$\vec{x} = \left[x_0,x_1,\ldots,x_{N-1}\right]$ and
$\vec{v} = \left[v_0,v_1,\ldots,v_{N-1}\right]$
\[
    \vec{x} \cdot \vec{v}   =
    \vec{x}^T \vec{v}       =
    \sum_{k=0}^{N-1}x_k v_k
\]
A number of other modules rely on {\tt dotprod}, such as filtering and
equalization.

\subsection{Specific machine architectures}
\label{module:dotprod:arch}
The vector dot product has a complexity of $\ord(N)$ multiply-and-accumulate
operations.
Because of its prevalence in multimedia applications, a considerable amount of
research has been put into computing the vector dot product as efficiently as
possible.
Software-defined radio is no exception as basic profiling will likely
demonstrate that a considerable portion of the processor is spent computing
it.
Certain machine architectures have specific instructions for computing vector
dot products... SIMD, mmx, sse, AltiVec, etc.

\subsection{Usage}
\label{module:dotprod:usage}
There are effectively two ways to use the {\tt dotprod} module.
In the first and most general case, a vector dot product is computed on two
input vectors $\vec{x}$ and $\vec{v}$ whose values are not known
{\it a priori}.

In the second case, a {\tt dotprod} object is created around vector $\vec{v}$
which does not change (or rarely changes) throughout its life cycle.
This is the more convenient method for filtering objects which don't usually
have time-dependent coefficients.

\input{listings/dotprod_rrrf.example.c.tex}

In both cases the {\tt dotprod} can be easily integrated with the
{\tt window} object for managing input data and alignment.
