% 
% MODULE : random
%

\newpage
\section{random}
\label{module:random}
The random module in \liquid\ includes a comprehensive set of
random number generators useful for simulation of wireless
communications channels,
particularly for generating noise as well as fading channels.
This includes the
uniform,
normal,
circular (complex) Gaussian,
Rice-$K$, and
Weibull distributions.


% 
% Uniform
%
\subsection{Uniform}
\label{module:random:uniform}
The uniform random variable generator in \liquid\ simply generates a
number evenly distributed in $[0,1)$.
Internally \liquid\ uses the standard {\tt rand()} method for generating
random integers and then divides by {\tt RAND\_MAX}, the maximum number
that can be generated.
The probability density function is defined as
%
\begin{equation}
\label{eqn:random:uniform:pdf}
    f_X(x) =
    \begin{cases}
        1 & \text{if $0 \leq x < 1$} \\
        0 & \text{else}.
    \end{cases}
\end{equation}
%
The uniform random number generator is the basis for generating most
other distributions in \liquid.

\begin{minipage}{0.5\textwidth}
  \includegraphics[trim = 5mm 0mm 5mm 0mm, clip, width=8cm]{figures.gen/random_histogram_uniform}
\end{minipage}
\begin{minipage}{0.5\textwidth}
  Uniform random number generator interface:
  \begin{Verbatim}[fontsize=\small]
  float randf();
  float randf_pdf(float _x);
  float randf_cdf(float _x);
  \end{Verbatim}
\end{minipage}


\subsection{Normal (Gaussian)}
\label{module:random:normal}
The normal (or Gauss) distribution has a probability density function
defined as
%
\begin{equation}
\label{eqn:random:normal:pdf}
    f_X(x;\sigma,\eta) =
        \frac{1}{\sigma \sqrt{2 \pi}}
        e^{-\left(x-\eta\right)^2/{2\sigma^2}}
\end{equation}
%
\liquid\ generates normal random variables using the Box-Muller method.
If $U_1$ and $U_2$ are uniform random variables with a distribution
defined by (\ref{eqn:random:uniform:pdf}), then
$X_1 = \sqrt{-2\ln(U_1)} \sin\left(2 \pi U_2\right)$ and
$X_2 = \sqrt{-2\ln(U_1)} \cos\left(2 \pi U_2\right)$
are independent normal random variables with a mean of zero and a unity
standard deviation ($X_1, X_2 \sim N(0,1)$).

\begin{minipage}{0.5\textwidth}
  \includegraphics[trim = 5mm 0mm 5mm 0mm, clip, width=8cm]{figures.gen/random_histogram_normal}
\end{minipage}
\begin{minipage}{0.5\textwidth}
  Normal (Gauss) random number generator interface:
  \begin{Verbatim}[fontsize=\small]
  float randnf();
  float randnf_pdf(float _x,
                   float _eta,
                   float _sigma);
  float randnf_cdf(float _x,
                   float _eta,
                   float _sigma);
  \end{Verbatim}
\end{minipage}


\subsection{Exponential}
\label{module:random:exponential}
The exponential distribution has a probability density function
defined as
%
\begin{equation}
\label{eqn:random:exponential:pdf}
    f_X(x;\lambda) = \lambda e^{-\lambda x}
\end{equation}
%
\liquid\ generates exponential random variables by inverting the
cumulative distribution function, viz
%
\begin{equation}
\label{eqn:random:exponential:cdf}
    F_X(x;\lambda) = 1 - e^{-\lambda x}
\end{equation}
%
Specifically if $U$ is uniform random variable with a distribution
defined by (\ref{eqn:random:uniform:pdf}) then
$X = -\ln U / \lambda$
%
has an exponential distribution defined by
(\ref{eqn:random:exponential:cdf}).

\begin{minipage}{0.5\textwidth}
  \includegraphics[trim = 5mm 0mm 5mm 0mm, clip, width=8cm]{figures.gen/random_histogram_exp}
\end{minipage}
\begin{minipage}{0.5\textwidth}
  Exponential random number generator interface:
  \begin{Verbatim}[fontsize=\small]
  float randexpf(float _lambda);
  float randexpf_pdf(float _x, float _lambda);
  float randexpf_cdf(float _x, float _lambda);
  \end{Verbatim}
\end{minipage}



% 
% Weibull
%
\subsection{Weibull}
\label{module:random:weibull}
The Weibull distrubtion has a probability denisty function defined by
%
\begin{equation}
\label{eqn:random:weibull:pdf}
    f_X(x;\alpha,\beta,\gamma) =
    \begin{cases}
        \frac{\alpha}{\beta}
        \left(
            \frac{x-\gamma}{\beta}
        \right)^{\alpha-1}
        \exp\Bigl\{
            -\left( \frac{x-\gamma}{\beta} \right)^\alpha
        \Bigr\}                         & \text{$x \ge \gamma$} \\
        0                               & \text{else}.
    \end{cases}
\end{equation}
%
where
$\alpha$ is the shape parameter,
$\beta$ is the scale parameter,
and
$\gamma$ is the threshold parameter.
%
\liquid\ generates Weibull random variables by inverting the cumulative
ditribution function, viz
%
\begin{equation}
\label{eqn:random:weibull:cdf}
    F_X(x;\alpha,\beta,\gamma) =
    \begin{cases}
        1 - \exp\left\{
            -\left(\frac{x-\gamma}{\beta}\right)^\alpha
        \right\} &                            x \ge \gamma \\
        0 &                                   \text{else}.
    \end{cases}
\end{equation}
%
Specifically if $U$ is uniform random variable with a distribution
defined by (\ref{eqn:random:uniform:pdf}) then
$X = \gamma + \beta\left[ \ln\left(1 - U\right) \right]^{1/\alpha}$
%
has a Weibull distribution defined by (\ref{eqn:random:weibull:cdf}).

\begin{minipage}{0.5\textwidth}
  \includegraphics[trim = 5mm 0mm 5mm 0mm, clip, width=8cm]{figures.gen/random_histogram_weib}
\end{minipage}
\begin{minipage}{0.5\textwidth}
  Weibull random number generator interface:
  \begin{Verbatim}[fontsize=\small]
  float randweibf(float _alpha,
                  float _beta,
                  float _gamma);
  float randweibf_pdf(float _x,
                      float _alpha,
                      float _beta,
                      float _gamma);
  float randweibf_cdf(float _x,
                      float _alpha,
                      float _beta,
                      float _gamma);
  \end{Verbatim}
\end{minipage}


% 
% Gamma
%
\subsection{Gamma}
\label{module:random:gamma}
The gamma distribution has a probability density function defined by
%
\begin{equation}
\label{eqn:random:gamma:pdf}
    f_X(x;\alpha,\beta) =
    \begin{cases}
        \frac{
            x^{\alpha-1}
        }{
            \Gamma(\alpha) \beta^\alpha
        }
        e^{-x / \beta}                  & x \ge 0 \\
        0                               & \text{else}.
    \end{cases}
\end{equation}
%


\begin{minipage}{0.5\textwidth}
  \includegraphics[trim = 5mm 0mm 5mm 0mm, clip, width=8cm]{figures.gen/random_histogram_gamma}
\end{minipage}
\begin{minipage}{0.5\textwidth}
  Gamma random number generator interface:
  \begin{Verbatim}[fontsize=\small]
  float randgammaf(float _alpha,
                   float _beta);
  float randgammaf_pdf(float _x,
                       float _alpha,
                       float _beta);
  float randgammaf_cdf(float _x,
                       float _alpha,
                       float _beta);
  \end{Verbatim}
\end{minipage}


% 
% Nakagami-m
%
\subsection{Nakagami-$m$}
\label{module:random:nakagamim}
The Nakagami-$m$ distribution is a versatile stochastic model for
modeling radio links \cite{Braun:1991} and has often been regarded as the
best distribution to model land mobile propagation due to its ability to
describe fading situations worse than Rayleigh, including one-sided
Gaussian \cite{Simon:1998}.
Empirical evidence regarding the efficacy the Nakagami-$m$ distribution
has on fading profiles been presented in \cite{Turin:1980, Suzuki:1977}.
Thus statistical inference of the Nakagami-$m$ fading parameters are of
interest in the design of adaptive radios such as optimized transmit
diversity modes \cite{Cavers:1999, Ko:2003} and adaptive modulation schemes
\cite{Catreux:2002}.
The Nakagami-$m$ probability density function is given by
\cite{Papoulis:2002}
%
\begin{equation}
\label{eqn:random:nakagamim:pdf}
    f_X(x;m,\Omega) =
    \begin{cases}
        \frac{2}{\Gamma(m)}
        \left( \frac{m}{\Omega} \right)^m
        x^{2m-1}
        e^{ -(m/\Omega)x^2}             & x \ge 0 \\
        0                               & \text{else}.
    \end{cases}
\end{equation}
%
where
$m \ge 1/2$ is the shape parameter and
$\Omega > 0$ is the spread parameter.
%
Nakagami-$m$ random numbers are generated from the gamma distribution.
Specifically if $R$ follows a gamma distribution defined by
(\ref{eqn:random:gamma:pdf}) with parameters $\alpha$ and $\beta$,
then $X=\sqrt{R}$ has a Nakagami-$m$ distribution with
$m=\alpha$ and $\Omega=\beta/\alpha$.


\begin{minipage}{0.5\textwidth}
  \includegraphics[trim = 5mm 0mm 5mm 0mm, clip, width=8cm]{figures.gen/random_histogram_nak}
\end{minipage}
\begin{minipage}{0.5\textwidth}
  Nakagami random number generator interface:
  \begin{Verbatim}[fontsize=\small]
  float randnakmf(float _m,
                  float _omega);
  float randnakmf_pdf(float _x,
                      float _m,
                      float _omega);
  float randnakmf_cdf(float _x,
                      float _m,
                      float _omega);
  \end{Verbatim}
\end{minipage}


%
% Rice-K
%
\subsection{Rice-$K$}
\label{module:random:ricek}
The Rice-$K$ multipath
channel models a fading envelope by assuming a line of sight (LoS)
component to the multipath elements summed at the receiver.
The complex path gain at a particular frequency consists of a fixed
(LoS) and fluctuating (diffuse) components.
When assuming a narrowband complex Gaussian stochasitc process, the
time-varying envelope will exhibit a Rice distribution where the $K$
factor is the power ratio of the LoS and diffuse components (often
referred to in dB) and thus is commonly used to describe fading
environments.
%
The Rice-$K$ distribution has a probability density function defined as
%
\begin{equation}
\label{eqn:random:ricek:pdf}
    f_R(r;K,\Omega) = 
        \frac{2(K+1)r}{\Omega}
        \exp\left\{-K-\frac{(K+1)r^2}{\Omega}\right\}
        I_0\left( 2r\sqrt{\frac{K(K+1)}{\Omega}} \right)
\end{equation}
%
where $\Omega=E\left\{R^2\right\}$ is the average signal power and $K$
is the fading factor (shape parameter).
\liquid\ generates Rice-$K$ random variables using two independent
normal random variables.
Specifically if
$X_0 \sim N(0,\sigma)$
and
$X_1 \sim N(s,\sigma)$
then
$R = \sqrt{X_0^2 + X_1^2}$
has follows a Rice-$K$ distribution defined by
(\ref{eqn:random:ricek:pdf})
where
$s      = \sqrt{\frac{\Omega K}{K+1}}$
and
$\sigma = \sqrt{\frac{\Omega}{2(K+1)}}$.


\begin{minipage}{0.5\textwidth}
  \includegraphics[trim = 5mm 0mm 5mm 0mm, clip, width=8cm]{figures.gen/random_histogram_rice}
\end{minipage}
\begin{minipage}{0.5\textwidth}
  Rice-$K$ random number generator interface:
  \begin{Verbatim}[fontsize=\small]
  float randricekf(float _m,
                   float _omega);
  float randricekf_pdf(float _x,
                       float _K,
                       float _omega);
  float randricekf_cdf(float _x,
                       float _K,
                       float _omega);
  \end{Verbatim}
\end{minipage}


% 
% Data scramble
%
\subsection{Data scrambler}
\label{module:random:data_scrambler}
Physical layer synchronization of received waveforms relies on independent and
identically distributed underlying data symbols.
If the message sequence, however, is too repetitive
(such as '{\tt 00000....}' or '{\tt 11001100....}')
and the modulation scheme is BPSK, the synchronizer probably won't be able to
recover the symbol timing because adjacent symbols are too similar.
This is a result of spectral correlation introduced which can prevent physical
layer synchronization techniques from tracking or even acquisition.
Having said that, certain patterns {\em are} beneficial to synchronization and
actually help the receiver track to the incoming signal, however these are
usually only introduced as a preamble to a frame or packet where the receiver
knows what to expect.
It is therefore imperative to increase the short-term entropy of the
underlying data to prevent the receiver from losing its lock on the signal.
The data scrambler routine attempts to ``whiten'' the data sequence with a bit
mask in order to achieve maximum entropy.

\subsubsection{interface}
The data scrambler has two methods, described here:
\begin{description}
\item[{\tt scramble\_data()}]
    takes an input sequence of data and scrambles the bits by applying a
    periodic mask.
    The first argument is a pointer to the input data array; the second
    argument is its length (number of bytes).
\item[{\tt unscramble\_data()}]
    takes an input sequence of data and unscrambles the bits by applying the
    reverse mask applied by {\tt scramble\_data()}.
    Just like {\tt scramble\_data()}, the first argument is a pointer to the
    input data array; the second argument is its length (number of bytes).
\end{description}

See {\tt examples/scramble\_example.c} for a full example of the interface.

