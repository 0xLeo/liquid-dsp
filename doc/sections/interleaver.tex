% 
% MODULE : interleaver
%

\section{interleaver}
\label{module:interleaver}

This section describes the functionality of the \liquid\ {\tt interleaver}
object.
In wireless communications systems, bit errors are often grouped together as a
result of multi-path fading, demodulator symbol errors, and synchronizer
instability.
Interleavers serve to distribute grouped bit errors evenly throughout a block
of data which aids certain forward error-correction (FEC) codes in their
decoding process (see section~\ref{module:fec} on error-correcting codes).
On the transmit side of the wireless link, the interleaver re-orders the bits
after FEC encoding and before modulation.
On the receiving side, the de-interleaver re-shuffles the bits to their
original position before attempting to run the FEC decoder.
The bit-shuffling order must be known at both the transmitter and receiver.

The {\tt interleaver} object operates by permuting indices on the input data
sequence.
The indices are computed during the {\tt interleaver\_create()} method and
stored internally.
At each iteration data bytes are re-shuffled using the permutation array.
Depending upon the properties of the array, multiple iterations should not
result in observing the original data sequence.
Shown below is a simple example where 8 symbols ($0,\ldots,7$) are re-ordered
using a random permutation.
The data at iteration 0 are the original data which are permuted twice.
% TODO : use graphic for this...
\begin{verbatim}
    forward
    permutation     iter[0]     iter[1]     iter[2]
    0 -> 6          0           6           1
    1 -> 4          1           4           3
    2 -> 7          2           7           5
    3 -> 0          3           0           6
    4 -> 3          4           3           0
    5 -> 2          5           2           7
    6 -> 1          6           1           4
    7 -> 5          7           5           2
\end{verbatim}
%
Reversing the process is as simple as computing the reverse permutation from
the input; this is equivalent to reversing the arrows in the forward
permutation
(e.g. the $2 \rightarrow 7$ forward permutation becomes the $7 \rightarrow 2$
reverse permutation).
\begin{verbatim}
    reverse
    permutation     iter[2]     iter[1]     iter[0]
    0 -> 3          1           6           0
    1 -> 6          3           4           1
    2 -> 5          5           7           2
    3 -> 4          6           0           3
    4 -> 1          0           3           4
    5 -> 7          7           2           5
    6 -> 0          4           1           6
    7 -> 2          2           5           7
\end{verbatim}
%
Notice that permuting indices only re-orders the bytes of data and does
nothing to shuffle the bits within the byte.
It is beneficial to FEC decoders to separate the bit errors as much as
possible.
Therefore, in addition to index permutation, \liquid\ also applies masks to
the data while permuting.
% TODO : explain more

Two options are available in \liquid\ for shuffling bits:
{\tt LIQUID\_INTERLEAVER\_BLOCK} and {\tt LIQUID\_INTERLEAVER\_SEQUENCE}.
These two methods are described here.

\subsection{{\tt LIQUID\_INTERLEAVER\_BLOCK} (block interleaving)}
\label{module:interleaver:block}
The block interleaver observes the block data as a matrix:
samples are read in by rows and out by columns.

% TODO : explain more...

\subsection{{\tt LIQUID\_INTERLEAVER\_SEQUENCE} ($m$-sequence interleaving)}
\label{module:interleaver:sequence}
This type of interleaving uses a special linear feedback shift register called
an $m$-sequence in order to compute its permutations.
The $m$-sequence has the special property that each symbol in the register is
both unique and pseudo-random.

% TODO : explain more...

See also {\tt msequence} (section~\ref{module:sequence}).

\subsection{interface}
\label{module:interleaver:interface}
The {\tt interleaver} object operates like most objects in \liquid; a

%-------------------- FIGURE: interleaver scatterplot --------------------
\begin{figure}
\centering
\mbox{
  \subfigure[$i=0$]
    {
      \includegraphics[trim = 15mm 0mm 15mm 0mm, clip, height=6cm]{figures.gen/interleaver_scatterplot_i0}
      \label{fig:interleaver:scatterplot:0}
    } \quad
  \subfigure[$i=1$]
    {
      \includegraphics[trim = 15mm 0mm 15mm 0mm, clip, height=6cm]{figures.gen/interleaver_scatterplot_i1}
      \label{fig:interleaver:scatterplot:1}
    } \quad
}
\mbox{
  \subfigure[$i=2$]
    {
      \includegraphics[trim = 15mm 0mm 15mm 0mm, clip, height=6cm]{figures.gen/interleaver_scatterplot_i2}
      \label{fig:interleaver:scatterplot:2}
    } \quad
  \subfigure[$i=3$]
    {
      \includegraphics[trim = 15mm 0mm 15mm 0mm, clip, height=6cm]{figures.gen/interleaver_scatterplot_i3}
      \label{fig:interleaver:scatterplot:3}
    } \quad
}
\caption{{\tt interleaver} (block) demonstration of a 64-byte (512-bit) array
with increasing number of iterations (interleaving depth)}
\label{fig:module:interleaver:scatterplot}
\end{figure}


