% 
% installation
%

\newpage
\section{Installation Guide}
\label{section:installation}
The \liquid\ DSP library can be easily built from source and is
available from several places.
The two most typical means of distribution are a compressed archive
(a {\em tarball}) and cloning the source repository.
Tarballs are generated with each stable release and are recommended for
users not requiring bleeding edge development.
Users wanting the very latest version (in addition to every other
version) should clone the \liquid\ Git repository.

\subsection{Building \& Dependencies}
\label{section:installation:building}
The \liquid\ signal processing library was intended to be universally
deployable to a number of platforms by eliminating dependencies on
external libraries and programs.
That being said, \liquid\ still does require a bare minimum build
environment to operate.
As such the library requires only the following:
%
\begin{itemize}
\item {\tt gcc}, the GNU compiler collection (or equivalent)
\item {\tt libc}, the standard C library
\item {\tt libm}, the standard math library (eventually will be phased
      out to optional)
\end{itemize}
%
While \liquid\ was designed to be portable,
requiring a minimal amount of dependencies,
the project will take advantage of other libraries if they are installed
on the target machine.
These optional packages are:
%
\begin{itemize}
\item {\tt fftw3} for computationally efficient fast Fourier transforms
\item {\tt libfec} for an extended number of forward error-correction
      codecs (including convolutional and Reed-Solomon)
\item {\tt liquid-fpm} (liquid fixed-point math library)
\end{itemize}
%
The build system checks to see if they are installed during the {\tt
configure} process and will generate an appropriate {\tt config.h} if
they are.

\subsection{Building from an archive}
\label{section:installation:build_from_tarball}
Download the compressed archive {\tt liquid-dsp-v.v.v.tar.gz} to your
local machine where {\tt v.v.v} denotes the version of the release
(e.g. {\tt liquid-dsp-\liquidversion.tar.gz}).
Check the validity of the tarball with the provided MD5 or SHA1 key.
%
% generate:
%   $ md5sum liquid-dsp-v.v.v.tar.gz > liquid-dsp.md5
%
% check:
%   $ md5sum --check liquid-dsp.md5
%
% You should see a message verifying the file:
%   liquid-dsp-v.v.v.tar.gz: OK
%
% If it fails, do no unpack.
Unpack the tarball
%
\begin{Verbatim}[fontsize=\small]
    $ tar -xvf liquid-dsp-v.v.v.tar.gz
\end{Verbatim}
%
Move into the directory and run the configure script and make the
library.
%
\begin{Verbatim}[fontsize=\small]
    $ cd liquid-dsp-v.v.v
    $ ./configure
    $ make
    # make install
\end{Verbatim}

\subsection{Building from the Git repository}
\label{section:installation:build_from_git}
Development of \liquid\ uses Git \cite{git:web}, a free and
open-source distributed version control system.
The benefits of Git over many other version control systems are
numerous and the list is too long to give here;
however one of the most important aspects is that each clone holds a
copy of the entire repository with a complete history and record of each
revision.
The main repository for \liquid\ is hosted online by {\em github}
\cite{github:web} and can be cloned on your local machine via
%
\begin{Verbatim}[fontsize=\small]
    $ git clone git://github.com/jgaeddert/liquid-dsp.git
\end{Verbatim}
%
Move into the directory and check out a particular tag using the
{\tt git checkout} command.%
\footnote{To list available tags run {\tt git tag -l}.}
Build as before with the archive, but with the additional bootstrapping
step.
%
\begin{Verbatim}[fontsize=\small]
    $ cd liquid-dsp
    $ git checkout v1.0.0
    $ ./boostrap.sh
    $ ./configure
    $ make
    # make install
\end{Verbatim}
%

%\subsection{Source code organization}
%In order to keep the project relatively organized, the source code is broken
%up into separate ``modules'' under the top {\tt src/} directory.

\section{Targets}
\label{section:installation:targets}
This section lists the specific targets in the main \liquid\ project.
A basic list can be printed by invoking ``{\tt make help}'' on the
command line.
This prints the following to the standard output:
%
\begin{Verbatim}[fontsize=\small]
  all       - build shared library (default)
  help      - print list of targets (see documentation for more)
  install   - installs the libraries and header files in the host system
  uninstall - uninstalls the libraries and header files in the host system
  check     - build and run autotest scripts
  bench     - build and run all benchmarks
  examples  - build all examples binaries
  sandbox   - build all sandbox binaries
  doc       - build documentation (doc/liquid.pdf)
  world     - build absolutely everything
  clean     - clean build (objects, dependencies, libraries, etc.)
  distclean - removes everything except the originally distributed files
\end{Verbatim}
%
The remainder of this section discusses some of the more important and
relevant targets.

%\subsection{Modules}
%\label{section:installation:targets:modules}
%Each module consists of a top-level included makefile, a {\tt README},
%the library source files, a set of test scripts, and a set of
%benchmarks.

\subsection{Examples ({\tt make examples})}
\label{section:installation:targets:examples}
All examples are built as stand-alone programs not build by the target
{\tt all} by default.
You may build all of the example binaries at one time by running
%
\begin{Verbatim}[fontsize=\small]
    make examples
\end{Verbatim}
%
Sometimes, however, it is useful to build one example individually.
This can be accomplished by directly targeting its binary
(e.g. ``{\tt make examples/modem\_example}'').
The example then can be run at the command line
(e.g. ``{\tt ./examples/modem\_example}'').

The examples are probably the best way to understand how each signal
processing element works.
Each example targets a specific functionality of \liquid,
such as FIR filtering, forward error correction, digital demodulation,
etc.
A number of the example programs when run will generate an output
{\tt .m} file which can be run directly in Octave~\cite{octave:web}.
This is particularly useful for visualizing filtering operations.
Most of the examples have a brief description at the top of the file;
these descriptions are also available in the {\tt examples/README} file
for convenience.
Some of the examples are experimental and will not be built by default
(see \S\ref{module:experimental}).

\subsection{Autotests ({\tt make check})}
\label{section:installation:targets:autotests}
Source code validation is a critical step in any software library,
particularly for verifying the portability of code to different processors and
platforms.
Packaged with \liquid\ are a number of automatic test scripts to validate the
correctness of the source code.
The test scripts are located under each module's {\tt tests} directory and
take the form of a C source file.
The testing framework operates similarly to CppUnit \cite{cppunit:web} and
cxxtest \cite{cxxtest:web}, however it is written in C.
The generator script {\tt scripts/autoscript} parses these header files looking for
the key ``{\tt void autotest\_}'' which corresponds to a specific test.
%Each test that is found is appended to a list
The script generates the header file {\tt autotest\_include.h} which
includes all the modules' test headers as well as several organizing
structures for keeping track of which tests have passed or failed.
The result is an executable file, {\tt xautotest}, which can be run to
validate the functional correctness of \liquid\ on your target platform.

\subsubsection{Macros}
Each module contains a number of autotest scripts which use pre-processor
macros for asserting the functional correctness of the source code.

\begin{description}
\item[{\tt CONTEND\_EQUALITY}$(x,y)$] asserts that $x==y$ and fails if
false.
\item[{\tt CONTEND\_INEQUALITY}$(x,y)$] asserts that $x$ differs from
$y$.
\item[{\tt CONTEND\_GREATER\_THAN}$(x,y)$] asserts that $x>y$.
\item[{\tt CONTEND\_LESS\_THAN}$(x,y)$] asserts that $x<y$.
\item[{\tt CONTEND\_DELTA}$(x,y,\Delta)$] asserts that $|x-y|<\Delta$
\item[{\tt CONTEND\_EXPRESSION}$(expr)$] asserts that some expression is
true.
\item[{\tt CONTEND\_SAME\_DATA}$(ptrA,ptrB,n)$] asserts that each of $n$
byte values in the arrays referenced by $ptrA$ and $ptrB$ are equal.
\item[{\tt AUTOTEST\_PASS}$()$] passes unconditionally.
\item[{\tt AUTOTEST\_FAIL}$(string)$] prints $string$ and fails
unconditionally.
\item[{\tt AUTOTEST\_WARN}$(string)$] simply prints a warning.
The autotest program will keep track of which tests elicit warnings and add
them to the list of unstable tests.
\end{description}

Here are some examples:
\begin{itemize}
\item[] {\tt CONTEND\_EQUALITY}(1,1) will {\it pass}
\item[] {\tt CONTEND\_EQUALITY}(1,2) will {\it fail}
\end{itemize}

\subsubsection{Running the autotests}
The result is an executable file named {\tt xautotest} which has several
options for running.
These options may be viewed with either the {\tt -h} or {\tt -u} flags (for
help/usage information).
%
\begin{Verbatim}[fontsize=\small]
    $ ./xautotest -h
    Usage: xautotest [OPTION]
    Execute autotest scripts for liquid-dsp library.
      -h,-u         display this help and exit
      -t[ID]        run specific test
      -p[ID]        run specific package
      -L            lists all scripts
      -l            lists all packages
      -x            stop on fail
      -s[STRING]    run all tests matching search string
      -v            verbose
      -q            quiet
\end{Verbatim}
%
Simply running the program without any arguments executes all the tests and
displays the results to the screen.
This is the default response of the target {\tt make check}.

\subsection{Benchmarks ({\tt make bench})}
\label{section:installation:targets:benchmarks}
Packaged with \liquid\ are benchmarks to determine the speed each signal
processing element can run on your machine.
You can build the benchmark program with {\tt make benchmark}, and view
the execution options with a {\tt -u} or {\tt -h} flag for usage/help
information:
%
\begin{Verbatim}[fontsize=\small]
    $ ./benchmark -h
    Usage: benchmark [OPTION]
    Execute benchmark scripts for liquid-dsp library.
      -h,-u         display this help and exit
      -v            verbose
      -q            quiet
      -e            estimate cpu clock frequency and exit
      -c            set cpu clock frequency (Hz)
      -n[COUNT]     set number of base trials
      -p[ID]        run specific package
      -b[ID]        run specific benchmark
      -t[SECONDS]   set minimum execution time (s)
      -l            list available packages
      -L            list all available scripts
      -s[STRING]    run all scripts matching search string
      -o[FILENAME]  export output
\end{Verbatim}
%
By default, running ``{\tt make bench}'' is equivalent to simply
executing the {\tt ./benchmark} program which runs all of the benchmarks
sequentially.
Initially the tool provides an estimate of the processor's clock
frequency;
while not necessarily accurate, this is necessary to gauge the relative
speed by which the benchmarks will run.
The tool will then estimate the number of trials so that each benchmark
will take between 50 and 500~ms to run.
Listed below is the output of the first several benchmarks:
%
\begin{Verbatim}[fontsize=\footnotesize]
  estimating cpu clock frequency...
  performed 67108864 trials in 650.0 ms
  estimated clock speed:   2.468 GHz
  setting number of trials to 246754
0: null
    0  : null                  :  23.59 M trials in 220.00 ms (107.212 M t/s,  22.00   cycles/t)
1: agc
    1  : agc_crcf              :   1.92 M trials in 270.00 ms (  7.093 M t/s, 337.50   cycles/t)
    2  : agc_crcf_squelch      :   1.92 M trials in 280.00 ms (  6.840 M t/s, 350.00   cycles/t)
    3  : agc_crcf_locked       :  15.32 M trials in 700.00 ms ( 21.887 M t/s, 109.38   cycles/t)
2: window
    4  : windowcf_n16          :   7.55 M trials in 260.00 ms ( 29.029 M t/s,  81.25   cycles/t)
    5  : windowcf_n32          :   7.55 M trials in 260.00 ms ( 29.029 M t/s,  81.25   cycles/t)
    6  : windowcf_n64          :   7.55 M trials in 270.00 ms ( 27.954 M t/s,  84.38   cycles/t)
    7  : windowcf_n128         :   7.55 M trials in 260.00 ms ( 29.029 M t/s,  81.25   cycles/t)
    8  : windowcf_n256         :   7.55 M trials in 260.00 ms ( 29.029 M t/s,  81.25   cycles/t)
3: dotprod_cccf
    9  : dotprod_cccf_4        :   1.89 M trials in 320.00 ms (  5.897 M t/s, 400.00   cycles/t)
    10 : dotprod_cccf_16       : 471.73 k trials in 320.00 ms (  1.474 M t/s,   1.60 k cycles/t)
    11 : dotprod_cccf_64       : 117.93 k trials in 300.00 ms (393.107 k t/s,   6.00 k cycles/t)
    12 : dotprod_cccf_256      :  29.48 k trials in 300.00 ms ( 98.267 k t/s,  24.00 k cycles/t)
4: dotprod_crcf
    13 : dotprod_crcf_4        :   1.89 M trials in  20.00 ms ( 94.347 M t/s,  25.00   cycles/t)
    14 : dotprod_crcf_16       : 471.73 k trials in  10.00 ms ( 47.173 M t/s,  50.00   cycles/t)
    15 : dotprod_crcf_64       : 117.93 k trials in   0.00 ps (    inf T t/s,   0.00 p cycles/t)
    16 : dotprod_crcf_256      :  29.48 k trials in  20.00 ms (  1.474 M t/s,   1.60 k cycles/t)
\end{Verbatim}
%
For this run the clock speed was estimated to be 2.468~GHz.
Benchmarks are sub-divided into {\em packages} which group similar
signal processing algorithms together.
For example, package {\tt 3} above refers to benchmarking the
{\tt dotprod\_cccf} object which computes the vector dot product
between two $n$-point arrays of complex floats.
Specifically, benchmark {\tt 11} refers to the speed of an $n=64$-point
dot product.
In this run the benchmarking tool computed approximately 117,930
64-point complex dot products in 300~ms
(about 393,107 trials per second).
For the estimated clock rate this means that the algorithm requires
approximately 6,000 clock cycles to compute
a single 64-point complex vector dot product.

\subsection{Documentation ({\tt make doc})}
\label{section:installation:targets:doc}
Specifically, ``{\tt make doc}'' builds this {\tt .pdf} file you're
reading right now.
The documentation requires a few additional packages to build from
scratch:
%
\begin{itemize}
\item {\tt pdflatex}, the \LaTeX engine responsible for making this
      document with all those pretty equations
\item {\tt bibtex}, the package for creating the bibliography
%\item {\tt tikz}, the professional...
\item {\tt gnuplot}, a program for plotting graphics
\item {\tt epstopdf}, conversion from {\tt .eps} to {\tt .pdf}, required
      for the figures created with {\tt gnuplot}
\item {\tt pygments}, the syntax highlighting engine responsible for
      generating all the fancy code listings given throughout this
      document.
      The command-line equivalent is called {\tt pygmentize}.
\end{itemize}
%

