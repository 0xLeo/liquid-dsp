% 
% installation
%

\newpage
\section{Installation Guide}
\label{section:installation}

The basic installation is as follows:
\begin{verbatim}
    $ ./reconf
    $ ./configure
    $ make
    # make install
\end{verbatim}
The {\tt install} target copies the global header file {\tt liquid.h} and the
shared object file {\tt libliquid.a} to the system directories.
You must have root access to run make with the {\tt install} target.
Additionally, the {\tt uninstall} target removes these files from the system
directory.

The additional makefile targets are listed here.

\section{Building \& Dependencies}
\label{section:installation:building}
Requirements:
\begin{itemize}
\item {\tt gcc}, the GNU compiler collection
\item {\tt libc}, the standard C library
\item {\tt libm}, the standard math library (eventually will be phased out to
optional)
\end{itemize}

\liquid\ was designed to be portable, and in doing so requires minimal dependencies to
build and run.
%Several well-known DSP packages (such as {\tt fftw} [cite] and {\it libfec} [cite])...
The project, however, will take advantage of other libraries if they are installed on the
target machine.
These optional packages are:
\begin{itemize}
\item {\tt fftw3}
\item {\tt libfec}
\item {\tt liquid-fpm} (liquid fixed-point math library)
\end{itemize}
The build system checks to see if they are installed during the {\tt configure} process
and will generate an appropriate {\tt config.h} if they are.

\subsection{Source code organization}
In order to keep the project relatively organized, the source code is broken
up into separate ``modules'' under the top {\tt src/} directory.

\section{Targets}
\label{section:installation:targets}

\subsubsection{Modules}
\label{section:installation:targets:modules}
Each module consists of a top-level included makefile, a {\tt README}, the
library source files, a set of test scripts, and a set of benchmarks.

\subsubsection{Examples ({\tt make examples})}
\label{section:installation:targets:examples}
All examples are built as stand-alone programs not build by the target
{\tt all} by default.
All examples can be built by invoking {\tt make examples}, or a specific
example can be build by invoking e.g. {\tt make examples/modem\_example}.

\subsection{Autotests ({\tt make check})}
\label{section:installation:targets:autotests}
Source code validation is a critical step in any software library,
particularly for verifying the portability of code to different processors and
platforms.
Packaged with \liquid\ are a number of automatic test scripts to validate the
correctness of the source code.
The test scripts are located under each module's {\tt tests} directory and
take the form of a C header file.
The testing framework operates similarly to CppUnit \cite{cppunit:web} and
cxxtest \cite{cxxtest:web}, however it is written in C (except for the
parser which, at the time of writing this, is in Python).
The python script {\tt autotest\_gen.py} parses these header files looking for
the key ``{\tt void autotest\_}'' which corresponds to a specific test.
%Each test that is found is appended to a list
The python script generates the header file {\tt autotest\_include.h} which
includes all the modules' test headers as well as several organizing
structures for keeping track of which tests have passed or failed.
The result is an executable file, {\tt xautotest}, which can be run to
validate the functional correctess of \liquid\ on your target platform.

\subsubsection{Macros}
Each module contains a number of autotest scripts which use pre-processor
macros for asserting the functional correctness of the souce code.

\begin{description}
\item[{\tt CONTEND\_EQUALITY}$(x,y)$] asserts that $x==y$ and fails if
false.
\item[{\tt CONTEND\_INEQUALITY}$(x,y)$] asserts that $x$ differs from
$y$.
\item[{\tt CONTEND\_GREATER\_THAN}$(x,y)$] asserts that $x>y$.
\item[{\tt CONTEND\_LESS\_THAN}$(x,y)$] asserts that $x<y$.
\item[{\tt CONTEND\_DELTA}$(x,y,\Delta)$] asserts that $|x-y|<\Delta$
\item[{\tt CONTEND\_EXPRESSION}$(expr)$] asserts that some expression is
true.
\item[{\tt CONTEND\_SAME\_DATA}$(ptrA,ptrB,n)$] asserts that each of $n$
byte values in the arrays referenced by $ptrA$ and $ptrB$ are equal.
\item[{\tt AUTOTEST\_PASS}$()$] passes unconditionally.
\item[{\tt AUTOTEST\_FAIL}$(string)$] prints $string$ and fails
unconditionaly.
\item[{\tt AUTOTEST\_WARN}$(string)$] simply prints a warning.
The autotest program will keep track of which tests elicit warnings and add
them to the list of unstable tests.
\end{description}

Here are some examples:
\begin{itemize}
\item[] {\tt CONTEND\_EQUALITY}(1,1) will {\it pass}
\item[] {\tt CONTEND\_EQUALITY}(1,2) will {\it fail}
\end{itemize}

\subsubsection{Running the autotests}
The result is an executable file named {\tt xautotest} which has several
options for running.
These options may be viewed with either the {\tt -h} or {\tt -u} flags (for
help/usage information).
\begin{verbatim}
    $ ./xautotest -h
    autotest options:
      -h,-u : prints this help file
      -t<n> : run specific test
      -p<n> : run specific package
      -L    : lists all autotests
      -l    : lists all packages
      -s    : stop on fail
      -v    : verbose
      -q    : quiet
\end{verbatim}
Simply running the program without any arguments executes all the tests and
displays the results to the screen.
The is the default response of the target {\tt make check}.

\subsection{Benchmarks ({\tt make bench})}
\label{section:installation:targets:benchmarks}
Packaged with \liquid\ are benchmarks to determine the speed each signal
processing element can run on your machine.
You can build the benchmark program with {\tt make benchmark}, and view the
execution options with a {\tt -u} or {\tt -h} flag for usage/help information:
\begin{verbatim}
    $ ./benchmark -h
    bench options:
      -h : prints this help file
      -e : estimate cpu clock frequency and exit
      -c : set cpu clock frequency (Hz)
      -n<num_trials>
      -p<package_index>
      -b<benchmark_index>
      -t<time> minimum execution time (ms)
      -l : lists available packages
      -L : lists all available benchmarks
      -v : verbose
      -q : quiet
      -o<output filename>
\end{verbatim}

\subsection{Targets summary}


