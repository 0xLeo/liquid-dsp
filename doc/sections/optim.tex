% optimization documentation
\section{optim (optimization)}
Newton-raphson, gradient, evolutionary algorithms, etc.

\subsection{Gradient search}
This module implements a steepest-descent search or gradient search.
Given a function $f$ which operates on a vector
$\vec{x} = \{x_0,x_1,\ldots,x_{N-1}\}$ of $N$ parameters,
the gradient search method seeks to find the optimum $\vec{x}$ which
minimizes $f$.
The gradient search is iterative, and adjusts $\vec{x}$ proportional to the
negative of the gradient of $f$ evaluated at the current location.
\[
    \Delta \vec{x}[n+1] = -\gamma[n] \nabla f(\vec{x}[n])
\]
where $\gamma[n]$ is the step size and
$\nabla f(\vec{x}[n])$ is the gradient of $f$ at $\vec{x}$, at the $n^{th}$
iteration.
The gradient is a vector field which points to the greatest rate of increase,
and is computed as
\[
    \nabla f(\vec{x}) = \left(
        \frac{f}{x_0},
        \frac{f}{x_1},
        \ldots,
        \frac{f}{x_{N-1}}
    \right)
\]
In most non-linear optimization problems, $\nabla f(\vec{x})$ is not known,
and must be estimated for each value of $\vec{x}[n]$ using the finite element
method.
The $k^{th}$ component of the gradient is approximated by
\[
    \frac{\partial f(\vec{x})}{\partial x_k} \approx 
    \frac{f(x_0,\ldots,x_k+\Delta,\ldots,x_{N-1}) - f(\vec{x})}{\Delta}
\]
An additional filtering operation is introduced on the step size to help
convergence, and therefore the updated vector at time $n+1$ is
\[
    \vec{x}[n+1] = \alpha \Delta\vec{x}[n+1] + (1-\alpha)\Delta\vec{x}[n]
\]
where $\Delta\vec{x}[0] = \vec{0}$.
In {\it liquid}, the gradient is normalized to unity (orthonormal).
Furthermore, $\gamma$ is slightly reduced each epoch.

