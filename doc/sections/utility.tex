% 
% MODULE : utility
%

\section{utility}
\label{module:utility}
The utility module contains useful functions, primarily for bit fast bit
manipulation.
This includes packing/unpacking byte arrays, counting ones in an integer,
computing a binary dot-product, and others.

\subsection{{\tt liquid\_pack\_bytes()},
            {\tt liquid\_unpack\_bytes()}, and
            {\tt liquid\_repack\_bytes()}}
\label{module:utility:pack_bytes}
Byte packing is used extensively in the
{\tt fec} (Section~\ref{module:fec}) and
{\tt framing} (Section~\ref{module:framing}) modules.
These methods resize symbols represented by various numbers of bits.
This is necessary to move between raw data arrays which use full bytes (eight
bits per symbol) to methods expecting symbols of different sizes.
In particular, the {\tt liquid\_repack\_bytes()} method is useful when one wants
to transmit a block of 64 data bytes using an 8-PSK modem which requires a
3-bit input symbol.
For example repacking two 8-bit symbols {\tt 00000000},{\tt 11111111} into six
3-bit symbols gives
{\tt 000},{\tt 000},{\tt 001},{\tt 111},{\tt 111},{\tt 100}.
Because 16 bits cannot be divided evenly among 3-bit symbols, the last symbol
is padded with zeros.

\subsection{miscellany}
\label{module:utility:misc}
This section describes the bit-counting methods which are used extensively
throughout \liquid, particularly the
{\tt fec} (Section~\ref{module:fec}) and
{\tt sequence} (Section~\ref{module:sequence}) modules.
Integer sizes vary for different machines;
when \liquid\ is initially configured (see Chapter~\ref{ch:installation}), the
size of the integer is computed such that the fastest method can be computed
without performing unnecessary loop iterations or comparisons.

\begin{description}
\item[{\tt liquid\_count\_ones(x)}]
    counts the number of {\tt 1}s that exist in the integer $x$.
    For example, the number {\tt 237} is represented in binary as
    {\tt 11101101}, therefore {\tt liquid\_count\_ones(237)} returns {\tt 6}.
\item[{\tt liquid\_count\_ones\_mod2(x)}]
    counts the number of {\tt 1}s that exist in the integer $x$, modulo 2; in
    other words, it returns {\tt 1} if the number of ones in $x$ is odd,
    {\tt 0} if the number is even.
    For example, {\tt liquid\_count\_ones\_mod2(237)} return {\tt 0}.
\item[{\tt liquid\_bdotprod(x,y)}]
    computes the binary dot-product between two itegers $x$ and $y$ as the sum
    of ones in $x \land y$, modulo 2 (where $\land$ is the logical `and'
    operation).
    This is useful in linear feedback shift registers
    (see Section~\ref{module:sequence:msequence} on $m$-sequences)
    as well as certain forward error-correction codes
    (see Section~\ref{module:fec:hamming} on Hamming codes).
    For example, the binary dot product between
    {\tt 10110011} and
    {\tt 11101110} is
    {\tt 1} because
    {\tt 10110011} $\land$ {\tt 11101110} $=$ {\tt 10100010} which has an odd
    number of {\tt 1}s.
\item[{\tt liquid\_count\_leading\_zeros(x)}]
    counts the number of leading zeros in the integer $x$.
    This is dependent upon the size of the integer for the target machine
    which is usually either two or four bytes.
\item[{\tt liquid\_msb\_index(x)}]
    computes the index of the most-significant bit in the integer $x$.
    The function will return {\tt 0} for $x=0$.
    For example if $x=129$ ({\tt 10000001}), the function will return {\tt 8}.
\end{description}


