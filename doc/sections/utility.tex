% 
% MODULE : utility
%

\newpage
\section{utility}
\label{module:utility}
The utility module contains useful functions, primarily for bit fast bit
manipulation.
This includes packing/unpacking byte arrays, counting ones in an integer,
computing a binary dot-product, and others.

\subsection{{\tt liquid\_pack\_bytes()},
            {\tt liquid\_unpack\_bytes()}, and
            {\tt liquid\_repack\_bytes()}}
\label{module:utility:pack_bytes}
Byte packing is used extensively in the
{\tt fec} (\S\ref{module:fec}) and
{\tt framing} (\S\ref{module:framing}) modules.
These methods resize symbols represented by various numbers of bits.
This is necessary to move between raw data arrays which use full bytes (eight
bits per symbol) to methods expecting symbols of different sizes.
In particular, the {\tt liquid\_repack\_bytes()} method is useful when one wants
to transmit a block of 64 data bytes using an 8-PSK modem which requires a
3-bit input symbol.
For example repacking two 8-bit symbols {\tt 00000000},{\tt 11111111} into six
3-bit symbols gives
{\tt 000},{\tt 000},{\tt 001},{\tt 111},{\tt 111},{\tt 100}.
Because 16 bits cannot be divided evenly among 3-bit symbols, the last symbol
is padded with zeros.

\subsection{{\tt liquid\_pack\_array()},
            {\tt liquid\_unpack\_array()}}
\label{module:utility:pack_array}
%
The {\tt liquid\_pack\_array()} and {\tt liquid\_unpack\_array()}
methods pack an array with symbols of arbitrary length.
These methods are similar to those in
\S\ref{module:utility:pack_bytes}
but are capable of packing symbols of any arbitrary length.
These are convenient for digital modulation and demodulation of a block
of symbols with different modulation schemes.
For example packing an array with five symbols
{\tt 1000},{\tt 011},{\tt 11010},{\tt 1},{\tt 000} yields two bytes:
{\tt 10000111,10101000}.
%
Here are the basic interfaces for packing and unpacking arrays:
\begin{Verbatim}[fontsize=\small]
  // pack binary array with symbol(s)
  void liquid_pack_array(unsigned char * _src,        // source array [size: _n x 1]
                         unsigned int _n,             // input source array length
                         unsigned int _k,             // bit index to write in _src
                         unsigned int _b,             // number of bits in input symbol
                         unsigned char _sym_in);      // input symbol

  // unpack symbols from binary array
  void liquid_unpack_array(unsigned char * _src,      // source array [size: _n x 1]
                           unsigned int _n,           // input source array length
                           unsigned int _k,           // bit index to write in _src
                           unsigned int _b,           // number of bits in output symbol
                           unsigned char * _sym_out); // output symbol
\end{Verbatim}
%
Listed below is a simple example of packing symbols of varying lengths
into a fixed array of bytes;
%
\input{listings/pack_array.example.c.tex}
%


\subsection{{\tt liquid\_lbshift()},
            {\tt liquid\_rbshift()}}
\label{module:utility:bshift}
Binary shifting.

{\tt liquid\_lbshift()}
\begin{Verbatim}[fontsize=\small]
    // input        : 1000 0001 1110 1111 0101 1111 1010 1010
    // output [0]   : 1000 0001 1110 1111 0101 1111 1010 1010
    // output [1]   : 0000 0011 1101 1110 1011 1111 0101 0100
    // output [2]   : 0000 0111 1011 1101 0111 1110 1010 1000
    // output [3]   : 0000 1111 0111 1010 1111 1101 0101 0000
    // output [4]   : 0001 1110 1111 0101 1111 1010 1010 0000
    // output [5]   : 0011 1101 1110 1011 1111 0101 0100 0000
    // output [6]   : 0111 1011 1101 0111 1110 1010 1000 0000
    // output [7]   : 1111 0111 1010 1111 1101 0101 0000 0000
\end{Verbatim}

{\tt liquid\_rbshift()}
\begin{Verbatim}[fontsize=\small]
    // input        : 1000 0001 1110 1111 0101 1111 1010 1010
    // output [0]   : 1000 0001 1110 1111 0101 1111 1010 1010
    // output [1]   : 0100 0000 1111 0111 1010 1111 1101 0101
    // output [2]   : 0010 0000 0111 1011 1101 0111 1110 1010
    // output [3]   : 0001 0000 0011 1101 1110 1011 1111 0101
    // output [4]   : 0000 1000 0001 1110 1111 0101 1111 1010
    // output [5]   : 0000 0100 0000 1111 0111 1010 1111 1101
    // output [6]   : 0000 0010 0000 0111 1011 1101 0111 1110
    // output [7]   : 0000 0001 0000 0011 1101 1110 1011 1111
\end{Verbatim}


\subsection{{\tt liquid\_lbcircshift()},
            {\tt liquid\_rbcircshift()}}
\label{module:utility:bcircshift}
Binary circular shifting.

{\tt liquid\_lbcircshift()}
\begin{Verbatim}[fontsize=\small]
    // input        : 1001 0001 1110 1111 0101 1111 1010 1010
    // output [0]   : 1001 0001 1110 1111 0101 1111 1010 1010
    // output [1]   : 0010 0011 1101 1110 1011 1111 0101 0101
    // output [2]   : 0100 0111 1011 1101 0111 1110 1010 1010
    // output [3]   : 1000 1111 0111 1010 1111 1101 0101 0100
    // output [4]   : 0001 1110 1111 0101 1111 1010 1010 1001
    // output [5]   : 0011 1101 1110 1011 1111 0101 0101 0010
    // output [6]   : 0111 1011 1101 0111 1110 1010 1010 0100
    // output [7]   : 1111 0111 1010 1111 1101 0101 0100 1000
\end{Verbatim}


{\tt liquid\_rbcircshift()}
\begin{Verbatim}[fontsize=\small]
    // input        : 1001 0001 1110 1111 0101 1111 1010 1010
    // output [0]   : 1001 0001 1110 1111 0101 1111 1010 1010
    // output [1]   : 0100 1000 1111 0111 1010 1111 1101 0101
    // output [2]   : 1010 0100 0111 1011 1101 0111 1110 1010
    // output [3]   : 0101 0010 0011 1101 1110 1011 1111 0101
    // output [4]   : 1010 1001 0001 1110 1111 0101 1111 1010
    // output [5]   : 0101 0100 1000 1111 0111 1010 1111 1101
    // output [6]   : 1010 1010 0100 0111 1011 1101 0111 1110
    // output [7]   : 0101 0101 0010 0011 1101 1110 1011 1111
\end{Verbatim}

\subsection{{\tt liquid\_lshift()},
            {\tt liquid\_rshift()}}
\label{module:utility:shift}
Byte-wise shifting.

{\tt liquid\_lshift()}
\begin{Verbatim}[fontsize=\small]
    // input        : 1000 0001 1110 1111 0101 1111 1010 1010
    // output [0]   : 1000 0001 1110 1111 0101 1111 1010 1010
    // output [1]   : 1110 1111 0101 1111 1010 1010 0000 0000
    // output [2]   : 0101 1111 1010 1010 0000 0000 0000 0000
    // output [3]   : 1010 1010 0000 0000 0000 0000 0000 0000
    // output [4]   : 0000 0000 0000 0000 0000 0000 0000 0000
\end{Verbatim}


{\tt liquid\_rshift()}
\begin{Verbatim}[fontsize=\small]
    // input        : 1000 0001 1110 1111 0101 1111 1010 1010
    // output [0]   : 1000 0001 1110 1111 0101 1111 1010 1010
    // output [1]   : 0000 0000 1000 0001 1110 1111 0101 1111
    // output [2]   : 0000 0000 0000 0000 1000 0001 1110 1111
    // output [3]   : 0000 0000 0000 0000 0000 0000 1000 0001
    // output [4]   : 0000 0000 0000 0000 0000 0000 0000 0000
\end{Verbatim}

\subsection{{\tt liquid\_lcircshift()},
            {\tt liquid\_rcircshift()}}
\label{module:utility:circshift}
Byte-wise circular shifting.

{\tt liquid\_lcircshift()}
\begin{Verbatim}[fontsize=\small]
    // input        : 1000 0001 1110 1111 0101 1111 1010 1010
    // output [0]   : 1000 0001 1110 1111 0101 1111 1010 1010
    // output [1]   : 1110 1111 0101 1111 1010 1010 1000 0001
    // output [2]   : 0101 1111 1010 1010 1000 0001 1110 1111
    // output [3]   : 1010 1010 1000 0001 1110 1111 0101 1111
    // output [4]   : 1000 0001 1110 1111 0101 1111 1010 1010
\end{Verbatim}

{\tt liquid\_rcircshift()}
\begin{Verbatim}[fontsize=\small]
    // input        : 1000 0001 1110 1111 0101 1111 1010 1010
    // output [0]   : 1000 0001 1110 1111 0101 1111 1010 1010
    // output [1]   : 1010 1010 1000 0001 1110 1111 0101 1111
    // output [2]   : 0101 1111 1010 1010 1000 0001 1110 1111
    // output [3]   : 1110 1111 0101 1111 1010 1010 1000 0001
    // output [4]   : 1000 0001 1110 1111 0101 1111 1010 1010
\end{Verbatim}

\subsection{miscellany}
\label{module:utility:misc}
This section describes the bit-counting methods which are used extensively
throughout \liquid, particularly the
{\tt fec} (\S\ref{module:fec}) and
{\tt sequence} (\S\ref{module:sequence}) modules.
Integer sizes vary for different machines;
when \liquid\ is initially configured (see Chapter~\ref{section:installation}), the
size of the integer is computed such that the fastest method can be computed
without performing unnecessary loop iterations or comparisons.

\begin{description}
\item[{\tt liquid\_count\_ones(x)}]
    counts the number of {\tt 1}s that exist in the integer $x$.
    For example, the number {\tt 237} is represented in binary as
    {\tt 11101101}, therefore {\tt liquid\_count\_ones(237)} returns {\tt 6}.
\item[{\tt liquid\_count\_ones\_mod2(x)}]
    counts the number of {\tt 1}s that exist in the integer $x$, modulo 2; in
    other words, it returns {\tt 1} if the number of ones in $x$ is odd,
    {\tt 0} if the number is even.
    For example, {\tt liquid\_count\_ones\_mod2(237)} return {\tt 0}.
\item[{\tt liquid\_bdotprod(x,y)}]
    computes the binary dot-product between two integers $x$ and $y$ as the sum
    of ones in $x \land y$, modulo 2 (where $\land$ is the logical `and'
    operation).
    This is useful in linear feedback shift registers
    (see \S\ref{module:sequence:msequence} on $m$-sequences)
    as well as certain forward error-correction codes
    (see \S\ref{module:fec:hamming} on Hamming codes).
    For example, the binary dot product between
    {\tt 10110011} and
    {\tt 11101110} is
    {\tt 1} because
    {\tt 10110011} $\land$ {\tt 11101110} $=$ {\tt 10100010} which has an odd
    number of {\tt 1}s.
\item[{\tt liquid\_count\_leading\_zeros(x)}]
    counts the number of leading zeros in the integer $x$.
    This is dependent upon the size of the integer for the target machine
    which is usually either two or four bytes.
\item[{\tt liquid\_msb\_index(x)}]
    computes the index of the most-significant bit in the integer $x$.
    The function will return {\tt 0} for $x=0$.
    For example if $x=129$ ({\tt 10000001}), the function will return {\tt 8}.
\end{description}


