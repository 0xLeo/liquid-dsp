% 
% MODULE : agc (automatic gain control)
%

\section{agc (automatic gain control)}
\label{module:agc}
Normalizing the incoming signal level is a critical step for many wireless
communications protocols, particularly in digital modulation schemes which
encode information in the signal amplitude (e.g. see {\tt MOD\_QAM}).
Furthermore, loop filters for tracking carrier and symbol timing are highly
sensitive to signal levels.
For these reasons and more does gain control play a crucial role in SDR.

\liquid\ implements automatic gain controlling with the {\tt agc\_xxxt}
object.
Operating one sample at a time, the {\tt agc} object makes an estimate
$\hat{e}$ of the signal energy and updates the internal gain $\hat{g}$,
applying it to the input to produce an output with the target energy
$\bar{e}$.

Problem areas: signals that have fluctuating gain...

Loop filter coefficients for a bandwidth $w$,
\[  \alpha = \sqrt{w}   \]
\[  \beta = 1 - \alpha  \]

\subsection{{\tt LIQUID\_AGC\_DEFAULT}}
The default {\tt agc} type is a... For a given input sample $x$ at time $n$,
the filtered signal energy estimate $\hat{e}$ is
\[  q(n) = \zeta\|x\|^2 + (1-\zeta)q(n-1)   \]
\[  \hat{e}(n) = \sqrt{q(n)}                \]
where $q(0)=0$ and $\zeta=0.1$ is a smoothing factor.%
\footnote{This smoothing factor is necessary to help prevent wild fluctuations
in $\hat{e}(n)$ which can occur if the input signal's instantaneous amplitude
has significant variation.}
The instantaneous ideal gain is
\[
    \bar{g}(n) = \bar{e}(n) / \hat{e}(n)
\]
and the filtered gain estimate is
\[
    \hat{g}(n+1) = \beta \hat{g}(n) + \alpha \bar{g}(n)
\]

\subsection{{\tt LIQUID\_AGC\_LOG}}
As in the default case, the log {\tt agc} type makes an estimate of the
instantaneous gain,
\[
    \bar{g}(n) = \bar{e}(n) / \hat{e}(n)
\]
however the loop control operates on the log of the gain error, viz
\[
    \gamma(n) = \ln\bigl( \bar{g}(n) / \hat{g}(n) \bigr)
\]
The {\tt agc} then updates its gain estimate according to the magnitude of the
error.
\[
    \hat{g}(n+1) = \hat{g}(n) e^{ \alpha \gamma(n) }
\]
Notice that when $\gamma(n)=0$, the gain is not updated
(i.e. $\hat{g}(n+1) = \hat{g}(n)$).


\subsection{{\tt LIQUID\_AGC\_EXP}}
The exponential gain update method does not...
The instantaneous output energy is
\[  \|y(n)\| = \hat{g}(n)\|x(n)\|      \]
and the gain error is
\[
    \gamma(n) = \|y(n)\| - \bar{e}
\]
The gain estimate is updated proportional to the gain error, viz
\[
    \hat{g}(n+1) =
        \hat{g}(n) \left(1 -
         \frac{
            \gamma(n)\sqrt{\alpha}
        }{
            \text{argmax}\left\{\bar{e}(n),\|y(n)\|\right\}
        }
    \right)
\]
The multiplier in to the right of the above equation is always positive
and proportional to the gain error.
Notice that when the error is zero ($\|y(n)\| = \bar{e}$),
$\hat{g}(n+1) = \hat{g}$

\subsection{Usage}
Basic usage:
\input{listings/agc.example.c.tex}

\begin{figure}
\centering
  \includegraphics[trim = 0mm 0mm 0mm 0mm, clip, width=13cm]{figures.gen/agc_squelch_pgf}
\caption{{\tt agc\_crcf} squelch}
\label{fig:module:agc:squelch}
\end{figure}


