% Audio documentation / sandbox
% 
% MODULE : audio
%

\newpage
\section{audio}
\label{module:audio}

\subsection{{\tt cvsd} (continuously variable slope delta)}
\label{module:audio:cvsd}
Continuously variable slope delta (CVSD) source encoding is used for data
compression of audio signals.
CVSD is a lossy compression whose quality is directly related to the sampling
frequency and is generally most practical for speech applications.
It is a form of delta modulation where $\Delta$ (the step size) is changed
continuously to minimize slope-overload distortion \cite[p. 131]{Proakis:2001}.
The output bit stream has a rate equal to that of the sampling frequency.
It is considered to be a moderate compromise between quality and complexity.

The algorithm attempts to dynamically adjust $\Delta$ value to track
to the input signal.
As with regular delta modulation algorithms,
if the decoded reference signal exceeds the input (the error signal is
negative), a binary {\tt 0} is sent and $\Delta$ is subtracted from the
reference, otherwise a binary {\tt 1} is sent and $\Delta$ is added.
However CVSD observes the previous $N$ transmitted bits are stored in memory.
$\Delta$ is increased by $\zeta$ if they are equal, and decreased otherwise.
This improves the dynamic range of the encoder over fixed-delta modulation
encoders.

The {\tt cvsd} module in \liquid\ allows the user to select both $\zeta$
as well as $N$, the number of repeated bits observed before $\Delta$ is
updated.
The combination of these values with the sampling rate yields a speech
compression algorithm with moderate quality.
A demonstration of the algorithm can be seen in
Figure~\ref{fig:module:audio:cvsd} where the encoder attempts to track to an
input sinusoid.
Notice that the encoder sometimes overshoots the reference signal.
This distortion results in degradations, particularly in the upper frequency
bands.

\begin{figure}
\centering
  \includegraphics[trim = 0mm 0mm 0mm 0mm, clip, width=13cm]{figures.gen/audio_cvsd}
\caption{{\tt cvsd} demonstration}
\label{fig:module:audio:cvsd}
\end{figure}

