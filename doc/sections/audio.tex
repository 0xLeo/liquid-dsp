% Audio documentation / sandbox
% 
% MODULE : audio
%

\newpage
\section{audio}
\label{module:audio}

\subsection{{\tt cvsd} (continuously variable slope delta)}
\label{module:audio:cvsd}
Continuously variable slope delta (CVSD) source encoding is used for data
compression of audio signals.
CVSD is a lossy compression whose quality is directly related to the sampling
frequency and is generally most practical for speech applications.
It is a form of delta modulation where $\Delta$ (the step size) is changed
continuously to minimize slope-overload distortion \cite[p. 131]{Proakis:2001}.
The output bit stream has a rate equal to that of the sampling frequency.
It is considered to be a moderate compromise between quality and complexity.

The algorithm attempts to dynamically adjust $\Delta$ value to track
to the input signal.
As with regular delta modulation algorithms,
if the decoded reference signal exceeds the input (the error signal is
negative), a binary {\tt 0} is sent and $\Delta$ is subtracted from the
reference, otherwise a binary {\tt 1} is sent and $\Delta$ is added.
However CVSD observes the previous $N$ transmitted bits are stored in memory.
$\Delta$ is increased by $\zeta$ if they are equal, and decreased otherwise.
This improves the dynamic range of the encoder over fixed-delta modulation
encoders.

The {\tt cvsd} module in \liquid\ allows the user to select both $\zeta$
as well as $N$, the number of repeated bits observed before $\Delta$ is
updated.
The combination of these values with the sampling rate yields a speech
compression algorithm with moderate quality.
A demonstration of the algorithm can be seen in
Figure~\ref{fig:module:audio:cvsd} where the encoder attempts to track to an
input sinusoid.
Notice that the encoder sometimes overshoots the reference signal.
This distortion results in degradations, particularly in the upper frequency
bands.

\begin{figure}
\centering
  \includegraphics[trim = 0mm 0mm 0mm 0mm, clip, width=13cm]{figures.gen/audio_cvsd}
\caption{{\tt cvsd} demonstration}
\label{fig:module:audio:cvsd}
\end{figure}

\subsection{{\tt fbasc} (filterbank audio synthesizer codec)}
\label{module:audio:fbasc}

The fbasc audio codec implements an AAC-like compression algorithm, using the
modified discrete cosine transform as a loss-less channelizer.  The resulting
channelized data are then quantized based on their spectral energy levels and
then packed into a frame which the decoder can then interpret. The result is a
lossy encoder (as a result of quantization) whose compression/quality levels
can be easily varied.

%                +------+ -->           -->  +-------+
%                |      | -->           -->  |       |
%   original     |      | -->  M-band   -->  |       |     reconstructed
%     time   ->  | MDCT |  .  quantizer  .   | iMDCT | ->      time
%    series      |      |  .             .   |       |        series
%                |      |  .             .   |       |
%                +------+ -->           -->  +-------+

Specifically, fbasc uses sub-band coding to allocate quantization bits to each
channel in order to minimize distortion of the reconstructed signal. Sub-bands
with higher variance (signal 'energy') are assigned more bits.  This is the
heart of the codec, which exploits several components typical of audio signals
and aspects of human hearing and perception:
\begin{enumerate}
\item The majority of audio signals (including music and voice) have a
      strong time-frequency localization; that is, they only occupy a small
      fraction of audible frequencies for a short duration.  This is
      particularly true for voiced signals (e.g. vowel sounds).
\item The human ear (and brain) tends to be quite forgiving of spectral
      compression and often cannot easily distinguish between neighboring
      frequency components.
\end{enumerate}

There are several benefits to using fbasc over other compression algorithms
such as CVSD (see src/audio/readme.cvsd.txt) and auto-regressive models, the
main being that the algorithm is theoretically lossless (i.e. perfect
reconstruction) as the bit rate increases.  As a result, the codec is limited
only by the quantization noise on each channel.

Here are some useful definitions, as used in the fbasc code:
%                   __________  __________
%                  /   MDCT   \/          \
%                 /   window  /\          ...
%            ____/       ____/  \____                                time
%    frame:  [----s0----][----s1----][----s2----][----s3----] ... --->
%                        |          |
%                      ->|          |<- symbol (length = M samples)

\begin{description}
\item[MDCT]
the modified discrete cosine transform is a lapped discrete cosine
transform which uses a special windowing function to ensure perfect
reconstruction on its inverse. The transform operates on $2M$ time-domain
samples (overlapped by $M$) to produce $M$ frequency-domain samples.
Conversely, the inverse MDCT accepts $M$ frequency-domain samples and
produces $2M$ time-domain samples which are windowed and then overlapped to
reconstruct the original signal.  For convenience, we may refer to $M$
time-domain samples as a 'symbol.'

\item[symbol]
one block of $M$ time-domain samples upon which the MDCT operates.
 
\item[channel]
one of the $M$ frequency-domain components as a result of applying the
MDCT.  This is somewhat equivalent to a discrete Fourier transform 'bin.'
Note than $M$ is equal to the number of channels in analysis.

\item[frame]
a set of MDCT symbols upon which the fbasc codec runs its analysis.
Because the codec uses time-frequency localization for its encoding, it is
necessary for the codec to gain enough statistical information about the
original signal without losing temporal stationarity. The codec typically
operates on several symbols, however, the exact number depends on the
application.
\end{description}

\subsubsection{Interface}
\begin{description}
\item[{\tt fbasc\_create()}]
        creates an fbasc encoder/decoder object, allocating memory as
        necessary, and computing internal parameters appropriately.
\item[{\tt fbasc\_destroy()}]
        destroys an fbasc encoder/decoder object, freeing internally-allocated
        memory.
\item[{\tt fbasc\_encode()}]
        encode a frame of data, storing the header and frame data separately.
        This separation allows the user to use different forward
        error-correction codes (if desired) to protect the header differently
        than the rest of the frame.  It is important to keep the two together,
        however, as the header is a description of how to decode the frame.
\item[{\tt fbasc\_decode()}]
        decodes a frame of data, generating the reconstructed time series.
\end{description}

\subsubsection{Useful properties}
\begin{itemize}
\item Because of the nature of the MDCT, frames will overlap by $M$ samples
      (one symbol).  This introduces a reconstruction delay of $M$ samples,
      noticeable at the decoder.
\end{itemize}

%The header contains the following data:
%    id      name                # bytes
%    --      ----------          ---------
%    fid     frame id            2
%    g0      nominal gain        1
%    bk      bit allocation      num_channels / 2
%    gk      gain allocation     num_channels / 2
%    --      ----------          ---------
%            total:              num_channels + 3
%
%Miscellaneous information
%
%Example:
%
%create() parameters:
%    num channels    =   64  (samples/symbol)
%    samples/frame   =   512
%    bytes/frame     =   256
%    ---------------------------
%derived values:
%    symbols/frame   =   [samples/frame] / [samples/symbol]  =   8
%    bytes/symbol    =   [bytes/frame]   / [symbols/frame]   =   32
%
%Each symbol must be encoded with an even number of bytes.

