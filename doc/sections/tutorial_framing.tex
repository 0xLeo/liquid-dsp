% 
% TUTORIAL : framing
%

\newpage
\section{Tutorial: Framing}
\label{tutorial:framing}

In the previous tutorials we have created only the basic building blocks
for wireless communication.
This tutorial puts them all together by introducing a very simple
framing structure for sending and receiving data over a wireless link.
In this context ``framing'' refers to the encapsulation of data into a
modulated time series at complex baseband to be transmitted over a
wireless link.
Conversely, ``packets'' refer to packing raw message data bytes with
forward error-correction and data validity check redundancy.


%
% SUBSECTION : problem statement
%
\subsection{Problem Statement}
\label{tutorial:framing:problem}

%To properly convey information over a wireless link...
For this tutorial we will be using the {\tt framegen64} and
{\tt framesync64} objects in \liquid.
As you might have guessed {\tt framegen64} is the frame generator object
on the transmit side of the link
and {\tt framesync64} is the frame synchronizer on the receive side.
%
Together these objects realize a
a very simple frame which encapsulates a 12-byte header and 64-byte
payload within a frame consisting of 640 symbols at complex baseband.
Conveniently the frame generator interpolates these symbols with a
matched filter to produce a 1280-sample frame at complex baseband,
ready to be up-converted and transmitted over the air.
This frame has a nominal spectral efficiency of 0.8 bits/second/Hz
(512 bits from 64 payload bytes assembled in 640 symbols).%
\footnote{
    For simplicity this computation of spectral efficiency
    neglects any exceess bandwidth of the pulse-shaping filter.}
This means that if you transmit with a symbol rate of 10kHz you should
expect to see a throughput of 8kbps if all the frames are properly
decoded.
On the receiving side,
raw samples at complex baseband are streamed to an instance of
the frame synchronizer which picks out frames and invokes a user-defined
callback function.
The synchronizer corrects for gain, carrier, and sample timing offsets
(channel impairments) in the complex baseband samples with a minimal
amount of pre-processing required by the user.
%
To help with synchronization, the frame includes a special preamble
which can be seen in the figure below.\\
%
% FIGURE: framing:structure
\includegraphics[width=\textwidth]{figures.pgf/framing_structure}\\
%
After up-conversion (mixing up to a carrier frequency) the frame is
transmitted over the link where the receiver mixes the signal back down to
complex baseband.
The received signal will be attenuated and noisy and typically degrades
with distance between the two radios.
Also, because receiver's oscillators run independent of the
transmitter's,
this received signal will have other impairments such as carrier
and timing offsets.
In our program we will be operating at complex baseband and will add the
channel impairments artificially.

The frame synchronizer's purpose is to correct for all of these
impairments (within limitations, of course) and attempt to detect the
frame and decode its data.
The framing preamble assists the synchronizer by introducing special
phasing sequences before any information-bearing symbols which aids in
correcting for carrier and timing offsets.
Without going into great detail, these sequences significantly increase
the probability of frame detection and decoding while adding a minimal
amount of overhead to the frame;
a small price to pay for increased data reliability!


%
% SUBSECTION : 
%
\subsection{Setting up the Environment}
\label{tutorial:framing:environment}

As with the other tutorials I assume that you are using {\tt gcc} to
compile your programs and link to appripriate libraries.
Create a new file {\tt framing.c} and include the headers
{\tt stdio.h},
{\tt stdlib.h},
{\tt math.h},
{\tt complex.h}, and
{\tt liquid/liquid.h}.
Add the {\tt int main()} definition so that your program looks like
this:
%
\input{tutorials/framing_init_tutorial.c.tex}
%
Compile and link the program using {\tt gcc}:
%
\begin{Verbatim}[fontsize=\small]
    $ gcc -Wall -o framing -lm -lc -lliquid framing.c
\end{Verbatim}
%
The flag ``{\tt -Wall}'' tells the compiler to print all warnings
(unused and uninitialized variables, etc.),
``{\tt -o framing}'' specifies the name of the output program is
``{\tt framing}'', and
``{\tt -lm -lc -lliquid}'' tells the linker to link the binary against
the math, standard C, and \liquid\ DSP libraries, respectively.
Notice that the above command invokes both the compiler and the linker
collectively.
%While it is usually preferred to build an intermediate object...
%
If the compiler did not give any errors, the output executable
{\tt framing} is created which can be run as
%
\begin{Verbatim}[fontsize=\small]
    $ ./framing
\end{Verbatim}
%
and should simply print ``{\tt done.}'' to the screen.
You are now ready to add functionality to your program.



%
% SUBSECTION : frame generator
%
\subsection{Creating the Frame Generator}
\label{tutorial:framing:framegen}
% talking points
%   * components to a frame
%       * header
%       * payload
%       * etc.
%   * options (m, beta)
The particular framing structure we will be using accepts a 12-byte
header and a 64-byte payload and assembles them into a frame
consisting of 1280 samples.
These sizes are fixed and cannot be adjusted for this framing
structure.%
\footnote{
    Alternatively, the {\tt flexframegen} and {\tt flexframesync}
    objects implement a dynamic framing structure which has many more
    options than the {\tt framegen64} and {\tt framesync64} objects.
    See Section~\ref{module:framing} for details.}
The purpose of the header is to conveniently allow the user a separate
control channel to be packaged with the payload.
For example, if your application is to send a file using multiple
frames, the header can include an identification number to indicate
where in the file it should be written.
Another application of the header is to include a destination node
identifier for use in packet routing for ad hoc networks.
Both the header and payload are assembled with a 16-bit cyclic
redundancy check (CRC) to validate the integrity of the received data
and encoded using the Hamming(12,8) code for error correction.
(see Section~\ref{module:fec} for more information on error detection
and correction capabilities in \liquid).
The encoded header and payload are modulated with QPSK and encapsulated
with a BPSK preamble.
Finally, the resulting symbols are interpolated using a square-root
Nyquist matched filter at a rate of 2 samples per symbol.
This entire process is handled internally so that as a user the only
thing you will need to do is call one function.

The {\tt framegen64} object can be generated with the
{\tt framegen64\_create()} method which accepts two arguments:
an {\tt unsigned int} and a {\tt float}
representing the matched filter's length (in symbols) and
excess bandwidth factor, respectively.
To begin, create a frame generator having a square-root Nyquist filter
with a delay of 3 and an excess bandwidth factor of 0.7 as
%
\begin{Verbatim}[fontsize=\small]
    framegen64 fg = framegen64_create(3, 0.7);
\end{Verbatim}
%
As with all structures in \liquid\ you will need to invoke the
corresponding {\tt destroy()} method when you are finished with the
object.
Now allocate memory for the header and payload data arrays,
remembering that they have lengths 12 and 64, respectively.
Raw ``message'' data are stored as arrays of type {\tt unsigned char} in
\liquid.
%
\begin{Verbatim}[fontsize=\small]
    unsigned char header[12];
    unsigned char payload[64];
\end{Verbatim}
%
Finally you will need to create a buffer for storing the frame samples.
For this framing structure you will need to allocate 1280 samples of
type {\tt float complex}, viz
%
\begin{Verbatim}[fontsize=\small]
    float complex y[1280];
\end{Verbatim}
%
Initialize the header and payload arrays with whatever values you wish.
All that is needed to generate a frame is to invoke the frame
generator's {\tt execute()} method:
%
\begin{Verbatim}[fontsize=\small]
    framegen64_execute(fg, header, payload, y);
\end{Verbatim}
%
That's it!
This completely assembles the frame complete with interpolation and is
ready for up-conversion and transmission.
To generate another frame simply write whatever data you wish to the
header and payload buffers, and invoke the {\tt framegen64\_execute()}
method again as done above.
If you wish, print the first few samples of the generated frame to the
screen (you will need to separate the {\em real} and {\em imaginary}
components of each sample).
%
\begin{Verbatim}[fontsize=\small]
    for (i=0; i<30; i++)
        printf("%3u : %12.8f + j*%12.8f\n", i, crealf(y[i]), cimagf(y[i]));
\end{Verbatim}
%
Your program should now look similar to this:
%
\input{tutorials/framing_basic_tutorial.c.tex}
%
Compile the program as before, creating the executable
``{\tt framing}.''
Running the program should produce an output similar to this:
%
\begin{Verbatim}[fontsize=\small]
framegen64 [m=3, beta=0.70]:
    ramp/up symbols     :   16
    phasing symbols     :   64
    p/n symbols         :   64
    header symbols      :   84
    payload symbols     :   396
    payload symbols     :   396
    ramp\down symbols   :   16
    total symbols       :   640
  0 :   0.00000000 + j*  0.00000000
  1 :   0.00000000 + j*  0.00000000
  2 :  -0.00011255 + j*  0.00000000
  3 :   0.00014416 + j*  0.00000000
  4 :   0.00040660 + j*  0.00000000
        ...
 25 :   0.04375378 + j*  0.00000000
 26 :   0.97077769 + j*  0.00000000
 27 :  -0.04032370 + j*  0.00000000
 28 :  -1.09209442 + j*  0.00000000
 29 :   0.03534408 + j*  0.00000000
done.
\end{Verbatim}
%
You might notice that the {\em imaginary} component of the samples in
the beginning of the frame are zero.
This is because the preamble of the frame is BPSK which has no imaginary
component at complex baseband.

%
% SUBSECTION : frame synchronizer
%
\subsection{Creating the Frame Synchronizer}
\label{tutorial:framing:framesync}
% talking points
%   * what framesync actually does
%       * gain control
%       * frame synchronization
%       * symbol timing recovery
%       * carrier phase recovery
%       * matched filtering
%       * demodulation
%       * frame decoding
%   * _coherent_ demodulation (eliminate phase ambiguity)
%   * what is a callback function?

As stated earlier the frame synchronizer's purpose is to detect the
presence of a frame, correct for the channel impairments, decode the
data, and pass it back to the user.
In our program we will simply pass to the frame synchronizer the samples
we generated in the previous section with the frame generator.
%In real system, however, the receiver does not know when the beginning
%of the frame is and so...
Furthermore, the hardware interface might pass the baseband samples to
the synchronizer in blocks much smaller than the length of a frame
(512 samples, for instance)
or even blocks much {\em larger} than the length of a frame
(4096 samples, for instance).
How does the synchronizer relay the decoded data back to the program
without missing any frames?
The answer is through the use of a callback function.

What is a callback function?
Put quite simply, a callback function is a function pointer
(a designated address in memory)
that is invoked during a certain event.
For this example the callback function given to the {\tt framesync64}
synchronizer object when the object is created
and is invoked whenever the synchronizer finds a frame.
This happens irrespective of the size of the blocks passed to the
synchronizer.
If you pass it a block of data samples containing four frames|several
thousand samples|then the callback will be invoked four times
(assuming that channel impairments haven't corrupted the frame beyond
the point of recovery).
You can even pass the synchronizer one sample at a time if you wish.

The {\tt framesync64} object can be generated with the
{\tt framesync64\_create()} method which accepts three pointers as
arguments:
%
\begin{Verbatim}[fontsize=\small]
    framesync64 framesync64_create(framesyncprops_s *   _props,
                                   framesync64_callback _callback,
                                   void *               _userdata);
\end{Verbatim}
%
%Here is a description of...
%
\begin{description}
\item[{\tt \_props}]
    is a construct that defines the specific properties of the frame
    synchronizer.
    This includes loop bandwidths for carrier, timing, and gain
    recovery, as well as squelch and equalizer control.
    You may pass the value {\tt NULL} to use the default parameters
    (recommended for now).
\item[{\tt \_callback}]
    is a pointer to your callback function which will be invoked each
    time a frame is found and decoded.
\item[{\tt \_userdata}]
    is a {\tt void} pointer that is passed to the callback function each
    time it is invoked.
    This allows you to easily pass data from the callback function.
    Set to {\tt NULL} if you don't wish to use this.
\end{description}
%
The {\tt framesync64} object has a callback function which has six
arguments and looks like this:
%
\begin{Verbatim}[fontsize=\small]
    int framesync64_callback(unsigned char *  _header,
                             int              _header_valid,
                             unsigned char *  _payload,
                             int              _payload_valid,
                             framesyncstats_s _stats,
                             void *           _userdata);
\end{Verbatim}
%
The callback is typically defined to be {\tt static} and is passed to
the instance of {\tt framesync64} object when it is created.
%
\begin{description}
\item[{\tt \_header}]
    is a pointer to the 12 bytes of decoded header data.
    This pointer is not static and cannot be used after returning from
    the callback function.
    This means that it needs to be copied locally for you to retain the
    data.
\item[{\tt \_header\_valid}]
    is simply a flag to indicate if the header passed its cyclic
    redundancy check
    (``{\tt 0}'' means invalid, ``{\tt 1}'' means valid).
    If the check fails then the header data most likely has been
    corrupted beyond the point that the internal error-correction code
    can recover; proceed with caution!
\item[{\tt \_payload}]
    is a pointer to the 64 bytes of decoded payload data.
    Like the header,
    this pointer is not static and cannot be used after returning from
    the callback function.
    Again, this means that it needs to be copied locally for you to retain the
    data.
\item[{\tt \_payload\_valid}]
    is simply a flag to indicate if the payload passed its cyclic
    redundancy check
    (``{\tt 0}'' means invalid, ``{\tt 1}'' means valid).
    As with the header,
    if this flag is zero then the payload most likely has errors in it.
    Some applications are error tolerant and so it is possible that the
    payload data are still useful.
    Typically, though, the payload should be discarded and a
    re-transmission request should be issued.
\item[{\tt \_stats}]
    is a synchronizer statistics construct that indicates some useful
    PHY information to the user.
    We will ignore this information in our program, but it can be quite
    useful for certain applications.
    For more information on the {\tt framesyncstats\_s} structure, see
    Section~\ref{module:framing:framesyncstats_s}.
\item[{\tt \_userdata}]
    Remember that {\tt void} pointer you passed to the {\tt create()}
    method?
    That pointer is passed to the callback and can represent just about
    anything.
    Typically it points to another structure and is the method
    by which the decoded header and payload data are returned to the
    program outside of the callback.
\end{description}
%
This can seem a bit overwhelming at first, but relax!
The next version of our program will only add about 20 lines of code.


%
% SUBSECTION : 
%
\subsection{Putting it All Together}
\label{tutorial:framing:xxx}
% talking points

First create your callback function at the beginning of the file, just
before the {\tt int main()} definition;
you may give it whatever name you like (e.g. {\tt mycallback()}).
For now ignore all the function inputs and just print a message to the
screen that indicates that the callback has been invoked,
and return the integer zero ({\tt 0}).
This return value for the callback function should always be zero
and is reserved for future development.
Within your {\tt main()} definition, create an instance of
{\tt framesync64} using the {\tt framesync64\_create()} method,
passing it a {\tt NULL} for the first and third arguments
(the properties and userdata constructs)
and the name of your callback function as the second argument.
Print the newly created synchronizer object to the screen if you like:
%
\begin{Verbatim}[fontsize=\small]
    framesync64 fs = framesync64_create(NULL,
                                        mycallback,
                                        NULL);
    framesync64_print(fs);
\end{Verbatim}
%
After your line that generates the frame samples
(``{\tt framegen64\_execute(fg, header, payload, y);}'')
invoke the synchronizer's {\tt execute()} method,
passing to it the frame synchronizer object you just created ({\tt fs}),
the pointer to the array of frame symbols ({\tt y}),
and the length of the array (1280):
%
\begin{Verbatim}[fontsize=\small]
    framesync64_execute(fs, y, 1280);
\end{Verbatim}
%
Finally, destroy the frame synchronizer object along with the frame
generator at the end of the file.
That's it!
Your program should look something like this:
%
\input{tutorials/framing_intermediate_tutorial.c.tex}
%
Compile and run your program as before and verify that your callback
function was indeed invoked.
Your output should look something like this:
%
\begin{Verbatim}[fontsize=\small]
    framegen64 [m=3, beta=0.70]:
        ramp/up symbols     :   16
        phasing symbols     :   64
        ...
    framesync64:
        agc signal min/max  :   -40.0 dB /  30.0dB
        agc b/w open/closed :   1.00e-03 / 1.00e-05
        sym b/w open/closed :   8.00e-02 / 5.00e-02
        pll b/w open/closed :   2.00e-02 / 5.00e-03
        samples/symbol      :   2
        filter length       :   3
        num filters (ppfb)  :   32
        filter excess b/w   :   0.7000
        squelch             :   disabled
        auto-squelch        :   disabled
        squelch threshold   :   -35.00 dB
        ----
        p/n sequence len    :   64
        payload len         :   64 bytes
    ***** callback invoked!
    done.
\end{Verbatim}
%
% ignore any lines pertaining to
%   framesync64/debug: results written to framesync64_internal_debug.m
As you can see, the {\tt framesync64} object has a long list of
modifiable properties pertaining to synchronization;
the default values provide a good initial set for a wide range of
channel conditions.
Duplicate the line of your code that executes the frame synchronizer.
Recompile and run your code again.
You should see the ``{\tt ***** callback invoked!}'' printed twice.

Your program has only demonstrated the basic functionality of the frame
generator and synchronizer under ideal conditions:
no noise, carrier offsets, etc.
The next section will add some channel impairments to stress the
synchronizer's ability to decode the frame.


%
% SUBSECTION : 
%
\subsection{Final Program}
\label{tutorial:framing:completed}

In this last section we will add some channel impairments to the frame
after it is generated and before it is received.
This will simulate non-ideal channel conditions.
Specifically we will introduce carrier frequency and phase offsets,
channel attenuation, and noise.
We will also add a frame counter and pass it through the {\em userdata}
construct in the frame synchronizer's {\tt create()} method to be passed
to the callback function when a frame is found.
Finally, the program will split the frame into pieces to emulate
non-contiguous data partitioning at the receiver.

To begin, add the following parameters to the beginning of your
{\tt main()} definition with the other options:
%
\begin{Verbatim}[fontsize=\small]
    unsigned int frame_counter = 0; // userdata passed to callback
    float phase_offset=0.3f;        // carrier phase offset
    float frequency_offset=0.02f;   // carrier frequency offset
    float SNRdB = 10.0f;            // signal-to-noise ratio [dB]
    float noise_floor = -40.0f;     // noise floor [dB]
\end{Verbatim}
%
The {\tt frame\_counter} variable is simply a number we will pass to the
callback function to demonstrate the functionality of the userdata
construct.
Make sure to initialize {\tt frame\_counter} to zero.
%
If you completed the tutorial on phase-locked loop design you might
recognize the {\tt phase\_offset} and {\tt frequency\_offset} variables;
these will be used in the same way to represent a carrier mismatch
between the transmitter and receiver.
%
The channel gain and noise parameters are a bit trickier and are set up
by the next two lines.
Typically the noise power is a fixed value in a receiver;
what changes is the received power based on the transmitter's power and
the gain of the channel;
however because theory dictates that the performance of a link is
governed by the ratio of signal power to noise power,
SNR is a more useful than defining signal amplitude and noise variance
independently.
The {\tt SNRdB} and {\tt noise\_floor} parameters fully describe the
channel in this regard.
The noise standard deviation and channel gain may be derived from these
values as follows:
%
\begin{Verbatim}[fontsize=\small]
    float nstd  = powf(10.0f, noise_floor*0.1f);
    float gamma = powf(10.0f, (SNRdB+noise_floor)*0.1f);
\end{Verbatim}
%
Add to your program
(after the {\tt framegen64\_execute()} line)
a loop that modifies each sample of the generated frame by introducing
the channel impairments.
%
\[
    y_i \leftarrow \gamma y_i e^{j(\theta + i\omega)} + \sigma n e^{j \pi u}
\]
%
where
$y_i$ is the frame sample at index $i$ ({\tt y[i]}),
$\gamma$ is the channel gain defined above ({\tt gamma}),
$\theta$ is the carrier phase offset ({\tt phase\_offset}),
$\omega$ is the carrier frequency offset ({\tt frequency\_offset}),
$\sigma$ is the noise standard deviation defined above ({\tt nstd}), and
$n$ and $u$ are random variables to represent noise with
normal and uniform distributions, repectively.
\liquid\ provides the {\tt randnf()} and {\tt randf()} methods to
generate real random numbers with Gauss and uniform distributions
respectively.
%
\begin{Verbatim}[fontsize=\small]
    y[i] *= gamma;
    y[i] *= cexpf(_Complex_I*(phase_offset + i*frequency_offset));
    y[i] += nstd * randnf() * cexpf(_Complex_I*M_PI*randf());
\end{Verbatim}
%
Check the program listed below if you need help.

Now modify the program to incorporate the frame counter.
First modify the piece of code where the frame synchronizer is created:
replace the last argument (initially set to {\tt NULL}) with the address
of our {\tt frame\_counter} variable.
For posterity's sake, this address will need to be type cast to
{\tt void*} (a void pointer) to prevent the compiler from complaining.
In your callback function you will reverse this process:
create a new variable of type {\tt unsigned int*}
(a pointer to an unsigned integer)
and assign it the {\tt \_userdata} argument type cast to
{\tt unsigned int*}.
Now dereference this variable and increment its value.
Finally print its value near the end of the {\tt main()} definition to
ensure it is being properly incremented.
Again, check the program below for assistance.

The last task we will do is push one sample at a time to the frame
synchronizer rather than the entire frame block to emulate
non-contiguous sample streaming.
To do this, simply remove the line that calls
{\tt framesync64\_execute()} on the entire frame
and replace it with a loop that calls the same function but with one
sample at a time.

The final program is listed below,
and a copy of the source is located in the {\tt doc/tutorials/}
subdirectory.
%
\input{tutorials/framing_tutorial.c.tex}
%
Compile and run the program as before.
The output of your program should look something like this:
%
\begin{Verbatim}[fontsize=\small]
    framegen64 [m=3, beta=0.70]:
        ramp/up symbols     :   16
        phasing symbols     :   64
        ...
    framesync64:
        agc signal min/max  :   -40.0 dB /  30.0dB
        agc b/w open/closed :   1.00e-03 / 1.00e-05
        ...
    ***** callback invoked!
      header (valid)
      payload (valid)
    received 1 frames
    done.
\end{Verbatim}
%
% extra credit
Play around with the initial options, particularly those pertaining to
the channel impairments.
Under what circumstances does the synchronizer miss the frame?
For example, what is the minimum SNR level that is required to reliably
receive a frame?
the maximum carrier frequency offset?

% additional notes:
The ``random'' noise generated by the program will be seeded to the same
value every time the program is run.
A new seed can be initialized on the system's time (e.g. time of day) to
help generate new instances of random numbers each time the program is
run.
To do so, include the {\tt <time.h>} header to the top of your file and
add the following line to the beginning of your program's {\tt main()}
definition:
%
\begin{Verbatim}[fontsize=\small]
    srand(time(NULL));
\end{Verbatim}
%
This will ensure a unique simulation is run each time the program is
executed.
For a more detailed program, see {\tt examples/framesync64\_example.c}
in the main \liquid\ directory.
Section~\ref{module:framing} describes \liquid's framing module in
detail.


% flexframe
While the framing structure described in this section provides a simple
interface for transmitting and receiving data over a channel,
its functionality is limited and isn't particularly spectrally
efficient.
\liquid\ provides a more robust framing structure which allows the use
of any linear modulation scheme, two layers of forward error-correction
coding, and a variable preamble and payload length.
These properties can be reconfigured for each frame to allow fast
adaptation to quickly varying channel conditions.
Furthermore, the frame synchronizer on the receiver automatically
reconfigures itself for each frame it detects to allow as simple an
interface possible.
The frame generator and synchronizer objects are denoted
{\tt flexframegen} and
{\tt flexframesync},
respectively,
and are described in Section~\ref{module:framing}.
A detailed example program {\tt examples/flexframesync\_example.c}
is available in the main \liquid\ directory.

